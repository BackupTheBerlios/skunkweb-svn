\chapter{The \texttt{AE} Package}
\label{aelib}
Mention configuration if used outside of SkunkWeb, specifically that
the default paths are different (since there is no SkunkRoot) 
\section{Components}
\subsection{\texttt{callComponent}}
\begin{funcdesc}{callComponent}{name, argDict, cache = 0,
                    defer = None, compType = DT_REGULAR,
                    srcModTime = None}
Return the output of calling a component.
\begin{argdesc}
\item[name] The path to the component.
\item[argDict] A dictionary of arguments to the component.
\item[cache] A boolean specifying that we should attempt to get the
output from cache, and failing that, execute it and write the output
to cache.
\item[defer] A boolean specifying that 
\begin{enumerate}
\item If a recently expired cached version of the component exists
give me that for now, and
\item Evaluate the component after the response has been sent and
write it to cache.
\end{enumerate}
Obviously, deferral only applies if the \texttt{cache} argument is
true.
\item[compType] \label{dtscore} Either \texttt{DT_REGULAR} meaning a
regular textual 
component, \texttt{DT_INCLUDE} meaning a textual component that runs in the
namespace of the calling component, or \texttt{DT\_DATA} for a data
component.  The \texttt{DT\_} 
contants are in the \verb!AE.Component! module.
\item[srcModTime] Modification time of the component source, if known.
\end{argdesc}
\end{funcdesc}

\subsection{The Component Stack}

\begin{datadesc}{componentStack} A list used as the component stack
(\verb!componentStack[0]! being the bottom of the stack).
\end{datadesc}
\begin{datadesc}{topOfComponentStack} The current top of the component
stack.  You should use \verb!componentStack[topOfComponentStack]! as
the top of the stack, not \verb!componentStack[-1]!, since if an
exception has occurred, \verb!componentStack[-1]! will point to the
stack frame where the exception occurred, and in the case where the
exception was handled, the stack may not have been cleaned up yet
\end{datadesc}
\begin{funcdesc}{resetComponentStack}{} Should be called after the top
level call to \texttt{callComponent} has returned.  It clears out the
component stack.
\end{funcdesc}


Attributes of the ComponentStackFrame
\begin{memberdesc}[string]{name} The name of the component being
executed.
\end{memberdesc}
\begin{memberdesc}[dictionary]{namespace} The namespace dictionary that
the component is being executed in.
\end{memberdesc}
\begin{memberdesc}[Executable]{executable} The execution object of the
component (from \verb!AE.Executables!).
\end{memberdesc}
\begin{memberdesc}[dictionary]{argDict}  The explict arguments passed to
the component.
\end{memberdesc}
\begin{memberdesc}[dictionary]{auxArgs} The auto-arguments passed to the
component.
\end{memberdesc}
\begin{memberdesc}[]{compType} The kind of component this is (data or
textual).  See Section \ref{dtscore}, page \pageref{dtscore}.
\end{memberdesc}

\section{Caching}
\NOTE{Compiled Memory Caching}

\subsection{Filesystem Interface}
\begin{funcdesc}{\_statDocRoot}{path}
If talking to a normal filesystem, this would return \verb!os.stat(path)!.
\end{funcdesc}
\begin{funcdesc}{\_getDocRootModTime}{path}
If talking to a normal filesystem, this would return 
\verb!os.stat(path)[stat.ST_MTIME]!. 
\end{funcdesc}
\begin{funcdesc}{\_readDocRoot}{path}
If talking to a normal filesystem, this would return 
\verb!open(path).read()!.
\end{funcdesc}

\subsection{The Compile Cache}
\begin{funcdesc}{\_getCompiledThing}{name, srcModTime, legend,
compileFunc, version, reconstituteFunc = None, unconstituteFunc =
None} 
A generic way to get and cache compiled things from cache/disk/memory,
returns the compiled thing.
\begin{argdesc}
\item[compileFunc]      takes the name and the source and returns compiled form
\item[unconstituteFunc] takes the object and produces a marshal-friendly form
\item[reconstituteFunc] takes the unmarshalled form and reconstitutes into
                 the object
\item[name]             is the documentRoot relative path to the thing
\item[srcModTime]       the modification time of the source, if known,
if not known, say \None.
\item[legend]           the label to use in debug messages
\item[version]          marhallable thing that if the one in the compile cache
                 doesn't match, we recompile
\end{argdesc}
\end{funcdesc}


\subsection{The Component Cache}
\begin{funcdesc}{getCachedComponent}{name, argDict, auxArgs, srcModTime = None}
returns cached component (or None),  source Mod time
\begin{argdesc}
\item[name] Name under which the cached output is stored.
\item[argDict] Dictionary of explicit component arguments.  
\item[auxArgs] Dictionary of automatic component arguments.
\item[srcModTime] Generally, modification time of source.
\end{argdesc}
\end{funcdesc}


\begin{funcdesc}{putCachedComponent}{name, argDict, auxArgs, out, cache_exp_time}
Puts the cached form of a component's output to storage.
\begin{argdesc}
\item[name]  Name under which the cached output is stored. 
\item[argDict] Dictionary of component arguments.  
\item[auxArgs]  Dictionary of automatic component arguments.
\item[out] The output of the component.
\item[cache_exp_time] The number of seconds past the epoch when the
cached version will expire.
\end{argdesc}
\end{funcdesc}


\begin{funcdesc}{clearCache}{name, arguments, matchExact = None}
Selectively removes cached components from cache.
\begin{argdesc}
\item[name] Name of the component you want to clear.
\item[arguments] Argument dictionary to match.
\item[matchExact] If true, will clear \strong{only} if the argument
dictionary supplied matches the argument dictionary used to store the
component exactly, otherwise, any cached version whose argument
dictionary is a superset of \texttt{arguments} will be cleared.
\end{argdesc}

\end{funcdesc}




\section{File Formats}
Serialized tuple of (\emph{item}, \emph{version}).  Cached compiled
objs use the \texttt{marshal} module for serialization and cached
component outputs use \texttt{cPickle} for serialization.

\subsection{Compiled Templates}
item is tuple of (\emph{filename}, \emph{python source text},
\emph{python code object}, \emph{metadata})

\subsection{Compiled Message Catalogs}
item is tuple of (\emph{message dict}, \emph{source file name})

\subsection{Compiled Python Code}
item is tuple of (\emph{python code object}, \emph{python source text})

\subsection{Cached components}
Generation of md5 hash (see cachekey.c). 

describe file layout and MD5 hash stuff
path\_to\_src/firstbyte/secondbyte/fullhash.(cache|key)

item is dict of 
\begin{argdesc}
\item[exp_time] Seconds past the epoch at which this cached
representation expires.
\item[defer_time] when using deferred rendering, the expired
component's lifetime is temporarily extended so that it gives the
process time to render it and write it back to the cache.  When this
is the case, this is the time at which it expires again, otherwise -1.
\item[output] Serialized form of component output.
\item[full_key] Serialized form of full cache key.
\end{argdesc}

\subsection{Component Cache Distribution}
Take last 16 bits of md5 hash and mod numServers.  Insert \# between
component cache root and normal path to cache.  If access to server
fails abnormally, don't try it again for failoverRetry seconds, and use
failoverComponentCacheRoot in meantime for only the section(s) of the
component cache that failed recently.


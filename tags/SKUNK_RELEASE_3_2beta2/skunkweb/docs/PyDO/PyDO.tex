% -*- latex -*-
\documentclass[titlepage]{manual}
\usepackage{makeidx}
\usepackage[nottoc]{tocbibind} % make so bib and ind are in toc
\title{PyDO --- Python Data Objects}
\author{Drew Csillag}
\release{1.0}
\setshortversion{1}
\makeindex

\begin{document}
%function argument list environment
\newcommand{\argdescitem}[1]{\hspace\labelsep
                                \normalfont\ttfamily #1\ }
\newenvironment{argdesc}{\begin{list}{}{
        \renewcommand{\makelabel}{\argdescitem}
}
}{\end{list}}

\newcommand{\doref}[1]{(see section \ref{#1}, page \pageref{#1})}
\maketitle
\ 
\vfill 

\noindent
Copyright (C) 2001 Andrew Csillag

\noindent
Permission is granted to make and distribute verbatim copies of
this manual provided the copyright notice and this permission notice
are preserved on all copies.

%%end titlepage stuff

\tableofcontents



\chapter{What is it?}
In short, PyDO allows you to simplify access to databases in a
comprehensible way.  Ok, now for a bit more detail.  SDS, PyDO's
predecessor made data modelling and access considerably easier than
doing the table design, SQL writing et. al. than doing it yourself.
The main problem was is that the database people had a hard time
understanding what it actually did under the hood, because, for them
(and rightly so) it was really important that it didn't do anything
stupid and allowed them to optimize it to death.  Since it was
difficult to explain to them how it did things, and it did constrain
them in meaningful ways, they didn't buy into it (and rightly so).
Basically, SDS traded ease of use for understanding, a tradeoff which
was it's undoing.

PyDO is meant as a way to give ultimate control to the database people
(if they want it) when it comes to database access and still be
relatively easy to use, but not as easy as SDS was. 

PyDO is also easy to configure, easy to see what goes on under the
hood and extremely lightweight (the PyDO.py file comes in currently at
616 lines of not-very-dense code).  Because the mapping is quite
thin, it is easy to explain how the mythical python expression:
\begin{verbatim}
SomeObject.getUnique(FOO=3)
\end{verbatim}
would yield the SQL query (when using the oracle driver)
\begin{verbatim}
SELECT COL1, COL2, COL3, FOO from TABLE where FOO = :p1 
\end{verbatim}

with \texttt{:p1} being bound to the integer 3 given it's definition.

Not only that, but you can override the way fetches and/or mutations
are done so that they don't necessarily even yeild SQL queries, in the
case that you want to do stored procedure access.  In general,
if you want to go direct to the database connection level to do
something, you can, and PyDO doesn't care, very much unlike SDS which
potentially could get very confused.

PyDO has no notion of a relation.  Relations are handled by the PyDO
data developer by using methods.  For example, if you have a Users
class and a Groups class, one would likely write a getGroups() method
on the Users object to fetch the Groups object associated with it.
PyDO does however provide convience functions to make implementing
relations simpler, specifically joinTable and joinTableSQL methods
which make many-to-many relations easier.  One-to-one and one-to-many
are are typically done making calls to \texttt{getUnique()} and
\texttt{getSome()} methods on the target class.

Unlike SDS, PyDO can also use more than one connection at a time.
Each data class defines a connection alias, which maps to a PyDO
connection string which subsequently maps to a database interface
instance (specific to the database type).  The connection alias
feature exists because you don't want to have to change all the
connect strings in your code to move them from the development
environment to the production environment, you just have to change the
connect string that the alias points to.


\chapter{How to Define a Data Class}
To define a PyDO data class, the first thing to do is inherit from the
PyDO base class.  From there, you define a series of class attributes
to configure the object.  

The connectionAlias attribute specifies the connection alias mentioned
above to determine which connection to use.

The table attribute specifies what database table this object maps
to.  Multiple dataclasses may point at the same database table.

The fields attribute is a tuple of two-tuples of column name (or field
name)/database type.  The case of the field name \emph{is} significant.
For all intents and purposes, use upper case unless the documentation
for your database driver says otherwise (none of them currently do, or
even have docs either for that matter).  If you have multiple data
classes pointing at the same database table, they need not specify the
same field tuples (they can though).

Data class instances are mutable unless you say \texttt{mutable = 0}
in the definition of your data class.

If you would like for fields in rows to be populated with values from
sequences (on databases that have named sequences, i.e. oracle) when
creating new rows, this can be done by specifying the sequenced
attribute as a dictionary of fieldname:sequence name pairs. 

If you would like to fetch fields that are populated via
the auto-increment feature of your database (if it has one, like
MySQL, oracle doesn't) on insert, this can be done by specifying the
auto_increment attribute as a dictionary of fieldname:autoincname
pairs.  In the case that autoincrement field fetches aren't named
(i.e. MySQL), just specify 1 as the autoincname, and beware then that
you can have only one item in the dictionary.

For some methods (i.e. \texttt{getUnique}, \texttt{delete}, \texttt{refresh}, etc.) that PyDO
has, it requires that it be able to obtain a unique row given a set of
column names/values.  The way to specify this is to set the unique
attribute on your data class to a list of strings or tuples of strings
(can mix and match) that identify that either this column (in the case
of a string) or this set of columns (tuple of strings) uniquely
identifies a row.  

Other attributes defined in a data class definition are ignored and
will not be present in the actual class.

If you want to add attributes into a data class instance, define the
\texttt{__init__} method (it will have no argument other than
\texttt{self}) and it can define whatever other instance attributes it
likes, although redefining data class attributes will have undefined
behavior (it might work *shrug*).

\section{Methods}
PyDO, like SDS supports methods.  Unlike SDS, though PyDO also has the
notion of static methods, methods that apply to the data class and not
an instance of the data class.

Defining a method is the same as defining regular python methods and
needs no explanation.

Defining a static method is merely a matter of defining the method
with static_ prepended to the method name.  In which case, the self
argument points to the data class and not the data class instance.

As you would expect, calling \texttt{self.method()} where method is
static is the same as calling \texttt{SelfsClass.method()}.

If you want to get a hold of the static method to be able to call it
from, say, a static method in the a subclass and have it be executed
in the class context of the subclass (not the superclass), use the
full \texttt{self.static_method()} form (\texttt{static_} prepended to
the static method name).

\section{Inheritance}
Inheritance is supported, albeit in a somewhat limited way.  Methods
(instance and static) are inherited as you would expect.  Fields in a
super class will be inherited into the subclass, where you can augment
the fields tuple or change the database type (by specifying the field,
but with a different database type).  This second form may or may not
be supported in future releases.  The unique is inherited, but setting
it will override, not augment, the super class' definition.
Inheriting from multiple PyDO classes is undefined as to the real
behavior.  It may work, but no guarantees.


\chapter{The Details}
\section{Connect Strings}
When calling \texttt{PyDO.PyDBI.DBIInitAlias}, you have to specify a connect
string.  If you are using PyDO from within the SkunkWeb server, use
the caching versions of the connect strings so that connections get
rolled back properly in the event of an error.  Obviously, if you 
dont have the pylib modules required for the caching versions, use
the direct methods.

For Oracle, they take one of two forms (either can optionally have
\texttt{|verbose} appended to them to log the sql executed by the
connection):
\begin{argdesc} 
\item[pydo:oracle:user|cache] uses the connection caching of the
\texttt{Oracle} pylib that is used by the \texttt{oracle} SkunkWeb
service.
\item[pydo:oracle:user/pw@host] use the \texttt{DCOracle} module
directly.
\end{argdesc}

For PostgreSQL, they also take one of two forms (either can optionally
have \texttt{:verbose} appended to them to log the sql executed by the
connection): 
\begin{argdesc}
\item[pydo:postgresql:user:cache] uses the connection caching of the
\texttt{PostgreSql} pylib that is used by the \texttt{postgresql}
SkunkWeb service.
\item[pydo:postgresql:normal\_postgresql\_connstr] use the
\texttt{pgdb} module directly.  In addition, if, in lieu of the host
portion of the normal PostgreSQL connection string you put
\texttt{host|port} instead, it will connect to the database listening
on the port \emph{port} instead of the default port.
\end{argdesc}

For MySQL, which hasn't been tested all that much, only the non-caching
version is currently available.  If there is demand I can do it, I just
don't have mysql installed anywhere convienient for me to test.
The form of PyDO/MySQL connect strings is:
\begin{verbatim}
pydo:mysql:normal\_mysql\_connect\_string
\end{verbatim}

\section{Inheritance}
%  * inheritance semantics from doc
All base class fields (columns) are inherited, subclasses can add fields 
and can only change inherited field types.

The \texttt{unique} and \texttt{connectionAlias} attributes are
inherited from left most, depth first class which defines them.

Static methods are inherited as static methods.

Instance methods are inherited as instance methods.

PyDO classes cannot inherit from non-PyDO classes.

The \texttt{_instantiable}, \texttt{sequenced} and
\texttt{auto_increment} attributes are not inherited.



\section{Data Class Details}
To be instantiable, class must define (or inherit) the
\texttt{connectionAlias}, table and fields attributes, or, can set the
_instantiable attribute to 1.  The overridability is there so, in the
case where you have your own fetching mechanisms (i.e. stored procs),
you can make the object instantiable even though it normally wouldn't
be (since no table for instance).

The fields member is tuple of (columnname, dbtype) pairs.
The unique attribute is a list of strings and/or tuple of strings.  If
a string, this says that this field is unique in the table, if a
tuple, this says these fields taken together are unique in the
table. 

You can make the dataclass instances immutable by defining the
\texttt{mutable} attribute as a false value (\texttt{None}, 0, empty
string, etc.)

For databases with named sequences, you can populate the value of an
field by defining the sequenced member as a dict of
\texttt{\{fieldname: sequence\_name\}} pairs, whereby if, on a call to
\texttt{new()}, the fields specified in sequenced are not present, the
values are fetched from the sequence(s) before insert and subsequently
inserted.

For databases with auto-increment fields, you can populate the value
of an field by defining the auto_increment member with a dict of
\texttt{\{fieldname: auto_increment\_name\}} pairs and the values will
be populated into the object after the insert is executed.  In the
case of MySQL, there can only be one auto-increment field per table,
so the auto_increment_name is needed, but it's value is irrelevant.

To define a static method (one that applies to the dataclass) define the 
method as 

\noindent
\texttt{def static\_}\emph{realmethodname}

Others are instance methods.

To get an unbound instance method, get
\emph{data\_class}.\emph{instance\_method}, to get a static method,
unbound from it's original data class (presumably called from a
sub-data-classe), use
\emph{data\_class}.static\_\emph{static\_method\_name}.

Attributes (static methods, data members, etc.) on data classes are accessible 
from instances.

\subsection{Data Class Attributes}
\begin{argdesc}
\item[_klass] name of the data class
\item[_baseClasses] tuple of super classes
\item[_staticMethods] dict of static methods
\item[_instanceMethods] dict of instance methods
\item[_rootClass] is the \texttt{_PyDOBase} root class metaclass
instance
\item[_instantiable] is this instantiable
\item[connectionAlias] connection alias string
\item[table] string naming the table
\item[mutable] are instances of this mutable?
\item[fieldDict] the dict of columnname: dbtype
\item[unique] list of candidate keys
\item[sequenced] dict of attrname: seq_name
\item[auto_increment] dict of attrname: auto_increment_name
\end{argdesc}

\subsection{Data Class Instance Attributes}
\begin{argdesc}
\item[_dataClass] the class which I'm an instance of
\item[_dict] dict of current row
\end{argdesc}

\subsection{Static Methods}

\begin{argdesc}
\item[getDBI()] gets database interface (see conn.readme)
\item[getColumns(qualified= None)] get column names (with table name
if qualified)
\item[getTable()] get table name
\item[_baseSelect(qualified = None)] get SELECT fragment to get rows
of object
\item[_matchUnique(kw)] returns an eligible candidate key based on
contents of \texttt{\_dict}
\item[_uniqueWhere(conn, kw)] generate a where clause from output of
\texttt{\_matchUnique}
\item[getUnique(**kw)] get a unique obj based on keyword args
\item[getSome(**kw)] get some objs based on keyword args
\item[new(refetch = None, **kw)] get new object based on kw args, if
refetch is true, refetch obj after insert
\item[_validateFields(dict)] does simple validation of fields on
insert
\item[commit()] causes the database connection of this object to
commit.
\item[rollback()]  causes the database connection of this object to
roll back.
\end{argdesc}

\subsection{Instance Methods}
\begin{argdesc}
\item[__init__()] can be used to prepopulate data object instance attributes.  Can have no arguments other than self.
\item[dict()] returns copy of dict representing current row
\item[updateValues(dict)] make the values in current dict "stick"
\item[delete()] delete current row
\item[refresh()] reload current object
\item[joinTable(thisAttrNames, pivotTable, thisSideColumns,
thatSideColumns, thatObject, thatAttrNames)] do cool m2m join
\item[joinTableSQL(thisAttrNames, pivotTable, thisSideColumns,
thatSideColumns, thatObject, thatAttrNames)] returns \texttt{sql} and
\texttt{value} list for \texttt{conn.execute} to do a m2m join but
doesn't execute it so you can do ordering or other stuff.
\end{argdesc}

FIXME scatterFetch stuff

\subsection{Dict-Stype Instance Methods}
PyDO object also obey a good majority of the dictionary interface.
They are:
\begin{verbatim}
__getitem__(item)
__setitem__(item, val)
items()
copy()
has_key(key)
key()
values()
get(item, default = None)
update(dict)
\end{verbatim}


\chapter{The Big Example}
An Oracle example:

Ok, for starters you've got a simple users table:

\begin{verbatim}
CREATE TABLE USERS (
    OID NUMBER,
    USERNAME VARCHAR(16),
    PASSWORD VARCHAR(16),
    CREATED DATE,
    LAST_MOD DATE
);
\end{verbatim}

And a sequencer for the \texttt{OID}:

\begin{verbatim}
CREATE SEQUENCE USERS_OID_SEQ;
\end{verbatim}

And you want to do stuff with it using PyDO.

\begin{verbatim}
from PyDO import *
DBIInitAlias('drew', 'pydo:oracle:drew/drew@drew')
class Users(PyDO):
    connectionAlias = 'drew'
    table = 'USERS'
    fields = (
        ('OID'     , 'NUMBER'),
        ('USERNAME', 'VARCHAR(16)'),
        ('PASSWORD', 'VARCHAR(16)'),
        ('CREATED' , 'DATE'),
        ('LAST_MOD', 'DATE')
        )
    sequenced = {
        'OID': 'USERS_OID_SEQ'
        }
    unique = [ 'OID', 'USERNAME' ]
\end{verbatim}

Ok, line-by-line, this is what this all means:

\begin{verbatim}
> from PyDO import *
\end{verbatim}

Import the contents of PyDO into your module namespace.  PyDO is
pretty clean and shouldn't pollute the namespace significantly as it
was designed to be imported this way, but if that irks you, doing a
regular \texttt{import PyDO} will also work (but you'll need to
adequately qualify things, obviously).

\begin{verbatim}
> DBIInitAlias('drew', 'pydo:oracle:drew/drew@drew')
\end{verbatim}

PyDO has a database driver library thingy.  It's not really meant for use 
outside of PyDO, but you can use it if you like, it's mainly there so the main
PyDO code doesn't have to care so much about the underlying database so much
in terms of things like: whether it support bind variables or not and etc.

The arguments to \texttt{DBIInitAlias} are: a connection alias name
(used as connectionAlias in your data classes), and a PyDO connect
string.  For oracle, the connect strings are of the form
\texttt{pydo:oracle:}\emph{user/password@inst}.

\begin{verbatim}
> class Users(PyDO):
\end{verbatim}

All dataclasses inherit directly or indirectly from the \texttt{PyDO}
base class.

\begin{verbatim}
> connectionAlias = 'drew'
\end{verbatim}

Used to select the database connection to use for this object.  You
can have more than one connection going at a time, so you need to
choose one (presumably the one that has the table you're going to
use).  In this case we're going to use the alias that we initialized
previously. 

\begin{verbatim}
> table = 'USERS'
\end{verbatim}

PyDO needs to know what table the rows will be coming from if it's
going to do anything, so we point it at the previously created
\texttt{USERS} table.

\begin{verbatim}
> fields = (
>     ('OID'     , 'NUMBER'),
>     ('USERNAME', 'VARCHAR(16)'),
>     ('PASSWORD', 'VARCHAR(16)'),
>     ('CREATED' , 'DATE'),
>     ('LAST_MOD', 'DATE')
>     )
\end{verbatim}

What you need to do here is associate the
column names from the table to their database type.  The case of the
column names *must* be the same as the native case of the database for
such things (specifically, the same case as what the database driver
returns on a describe of a query).  For most databases, this is
uppercase, same for the database type. 

\begin{verbatim}
> sequenced = {
>     'OID': 'USERS_OID_SEQ'
> }
\end{verbatim}

This says, if on a call to \texttt{new()} (described later),
\texttt{OID} is not specified, then fetch it from the sequence here
named.

\begin{verbatim}
> unique = [ 'OID', 'USERNAME' ]
\end{verbatim}

This is a list of candidate keys --- columns that uniquely identify a
row.  

\section{Using This New Data Class}
Assuming the aforecreated \texttt{USERS} table is empty, we need to
put some something in it before we start.

\begin{verbatim}
newUser = Users.new(USERNAME = 'drew', PASSWORD = 'foo', CREATED = SYSDATE,
                    LAST_MOD = SYSDATE)
\end{verbatim}

The new method inserts a new row into the table.  There is an optional
parameter, \texttt{refetch} which effectively calls the
\texttt{refresh} method (described below).  This is useful in the case
where you have a table with default values for columns and you want to
make references to the values with the defaults in place.

What this will do is: fetch a new \texttt{OID} from
\texttt{USERS_OID_SEQ} since \texttt{OID} wasn't specified above, and
subsequently insert a new row into the \texttt{USERS} table with the
\texttt{OID} and the values specified in the call to \texttt{new()}.

As you will notice, \texttt{SYSDATE} is a variable that translates to
the databases' idea of the current date (and time).


Now that we have a Users instance, we can examine it a bit more
closely.  PyDO subclass instances observe the python dictionary
interface.

For example:

\begin{verbatim}
>>> newUser['USERNAME']
'drew'
>>> newUser['OID']
1
>>> newUser.keys()
('OID', 'USERNAME', 'PASSWORD', 'CREATED', 'LAST_MOD')
\end{verbatim}


\section{Mutating Data Class Instances}

If you don't want your data class instances to be mutable (for
whatever reason), assign 1 to the \texttt{mutable} attribute in your class
definition.

To mutate an object, you use the dictionary-style mutation interface.
For example, to change the value of \texttt{USERNAME} in the current row:
\begin{verbatim}
>>> newUser['USERNAME'] = 'fred'
\end{verbatim}

What this will do is cause the following UPDATE query to be sent to the
connection.

\begin{verbatim}
UPDATE USERS SET USERNAME = :p1 WHERE OID = :p2
\end{verbatim}
(bind variables \texttt{:p1} = \texttt{'fred'} and \texttt{:p2 = 1})

One might ask: "how the hell did that happen?"  The answer is this:
it got the table from the table specified in the data class
description

\begin{verbatim}
> table = 'USERS'
\end{verbatim}

The attribute name is the item you assigned to.  The \texttt{OID =
:p2} part is a bit more interesting.  If you look above, you'll see:

\begin{verbatim}
> unique = ['OID', 'USERNAME']
\end{verbatim}

What PyDO does is this: it loops over the unique list and for each
item in the list is determines if it is a tuple or string.  If it's a
string, it's the name of an field that uniquely identifies a row in
the table (here \texttt{'OID'}).  If the current object has that
key-value pair, it stops having found an identifying field and so
composes the where clause.  If it is a tuple, it is a set of fields
that uniquely identify the row.  If all such fields are populated in
the current object, it will stop and compose the where clause from the
ANDing check of those fields in the current object.  In the case where
either there is no unique line specified or the key-value pairs aren't
defined in the current object, an exception saying "No way to get
unique row!" will be raised.

Ideally, if you want to update more than one field in your object in
one UPDATE query, you can use the \texttt{update} method (from the
dictionary interface) to accomplish this.

\begin{verbatim}
> newUser.update({'USERNAME': 'barney', PASSWORD='iF0rG0t'})
\end{verbatim}

This will update both column values in one UPDATE query.

You might now say, well, that's all fine an dandy, but to do this
correctly, I want to make sure that the \texttt{LAST_MOD} field gets
updated appropriately when people change the object!  Well, be at
rest, we can do that too.  Behind the scenes, the \texttt{__getitem__}
and update methods call an instance method updateValues that actually
does the hard work and we can override this to update
\texttt{LAST_MOD} as appropriate.

\section{Defining Instance Methods}

If we add the following method to the \texttt{Users} class definition, this
will do the trick:

\begin{verbatim}
    def updateValues(self, dict):
        if not dict.has_key['LAST_MOD']:
            dict = dict.copy()
            dict['LAST_MOD'] = SYSDATE
        return PyDO.updateValues(self, dict)
\end{verbatim}

All it does is say, if they didn't specify a value for
\texttt{LAST_MOD} (we assume here that if they specified it, they did
for good reason), we make a copy of the dict (in the case that they
still hold a reference to it, we don't want to screw it up) and set
\texttt{LAST_MOD} to \texttt{SYSDATE}, and subsequently call our
baseclasses version of \texttt{updateValues}.

This brings us to methods, of which there are two flavors, static and
instance.  Python itself doesn't have the notion of static methods,
but for certain applications (specifically PyDO), they can be
made available and incredibly useful.

To create a regular method, just write it as if everything was normal
in python land.  Nothing big to mention here.  It will apply to
instances of data classes and you can get an unbound verion by saying
\emph{class}.\emph{method} just as in regular Python.

\section{Defining Static Methods}

For static methods, you define your method as such:

\noindent
\texttt{def static_}\emph{mymethodname}\texttt{(self, }\emph{...whatever...}\texttt{):}

The \texttt{static_} prefix says "this is a static method".  The
\texttt{self} argument will point to the data class itself, not an
instance of the dataclass.  In the case that you want to call a super
classes static method on the current subclass, and in the context of
the subclass, you say:

\begin{verbatim}
   fooresult = MySuperClass.static_barmethod(self, baz, fred, barney)
\end{verbatim}

This is useful when you are overloading a static method in a subclass
but still want to call the superclass version (such as \texttt{new}).
This is very much unlike Java, which disallows this.  (don't know
about C++)

Your newly-defined method can then be called as

\noindent
\texttt{SomeClass.}\emph{mymethodname}\texttt{(}\emph{...whatever...}\texttt{)}
without the \texttt{static_} prefix to call the method statically.
For example, the call to the \texttt{new} method on the \texttt{Users}
object towards the beginning of this document is a static call (it's
defined as \texttt{static_new} in the \texttt{PyDO} base class).

NOTE: you cannot override a static method with an instance method or
vice versa.

Why this is useful is this: You want to make it so that you don't have
to specify the \texttt{CREATED} and \texttt{LAST_MOD} fields when
making a call to \texttt{new} since the caller shouldn't really have
to care and it can be taken care of automatically.  You can, if you
want to, enforce what fields they can or must set on a call to new.
For example: setting the \texttt{CREATED} and \texttt{LAST_MOD}
automatically and enforcing that \texttt{USERNAME} and
\texttt{PASSWORD} only are specified.

\begin{verbatim}
    def static_new(self, refetch = None, USERNAME, PASSWORD):
        return PyDO.static_new(self, refetch, USERNAME=USERNAME, 
                               PASSWORD=PASSWORD, CREATED=SYSDATE, 
                               LAST_MOD=SYSDATE)
\end{verbatim}
       
MAKE SURE TO USE THE STATIC UNBOUND VERSION WHEN CALLING YOUR
SUPERCLASS OR THE WRONG THINGS WILL LIKELY HAPPEN!

In the cases where PyDO is not your direct superclass, you might call
your superclass' \texttt{static_new} method instead.  On the other
hand, you may want to handle the new method entirely yourself.


\section{Relations and PyDO}
The way you do relations with PyDO is with methods.  For example, if
we had a \texttt{Files} class which had an field \texttt{OWNER_ID}
which was a foreign key to the \texttt{USERS} table, we could write a
method for the \texttt{Users} object like this (a one to many
relation):
\begin{verbatim}
    def getFiles(self):
        return Files.getSome(OWNER_ID = self['OID'])
\end{verbatim}

%FIXME, this paragraph sounds weird*
The \texttt{getSome} static method, given fields in the object will
generate a where clause with those fields and return a list of
objects, each of which representing one row.

If \texttt{Users} had an One to One relationship with
\texttt{Residence}, we could write a method to get it (presuming
\texttt{Residence}, again has a foreign key to the \texttt{USERS}
table in a column/field named \texttt{OWNER_ID}):

\begin{verbatim}
    def getResidence(self):
        return Residences.getUnique(OWNER_ID = self['OID'])
\end{verbatim}

The \texttt{getUnique} static method is similar to \texttt{getSome}
except that it will return only one row or \texttt{None}.  It uses the
\texttt{unique} attribute (here on the \texttt{Residences} object) to
determine how to get a unique row.  If you don't specify any
identifying rows, it will raise an exception saying "No way to get a
unique row", or in the case that it mysteriously finds more than one
row, will raise a similar exception.

To do Many To Many relations, things are a bit more interesting.
Since there may or may not be an object that represents the pivot
table (or linkage table) that links the two tables together, and you
probably wouldn't want to do the work to traverse all of them anyhow,
there is a \texttt{joinTable} method which simplifies the work.

Say there is a \texttt{Groups} entity and a table
\texttt{USERS_TO_GROUPS} which is the pivot table and has two columns,
\texttt{USER_ID} and \texttt{GROUP_ID} (foriegn keyed as appropriate).
You'd write a method \texttt{getGroups} as such:
\begin{verbatim}
    def getGroups(self):
        return self.joinTable('OID', 'USERS_TO_GROUPS', 'USER_ID',
                              'GROUP_ID', Groups, 'OID')
\end{verbatim}

What this will do is do the join across the \texttt{USERS_TO_GROUPS}
table to the table that the \texttt{Groups} object corresponds to.
The parameters (matched up to the arguments supplied above) are:

\begin{argdesc}
\item[thisAttributeNames]  \texttt{'OID'} attribute(s) in current object to join from
\item[pivotTable] \texttt{'USERS_TO_GROUPS'} pivot table name
\item[thisSideColumns] \texttt{'USER_ID'} column(s) that correspond to
the foriegn key column to \texttt{myAttributeName}.
\item[thatSideColumns] \texttt{'GROUP_ID'} column(s) that correspond
to the foriegn key column to \texttt{thatAttributeName}.
\item[thatObject] \texttt{Groups} the destination object.
\item[thatAttributeNames] \texttt{'OID'} see \texttt{thatSideColumns}.
\end{argdesc}

If in the case you want to do things like ordering and such on a many
to many relation, you can use the \texttt{joinTableSQL} function
(takes the same arguments) to get the sql and value list to use.  From
there you can add to the generated sql statement things like
\texttt{ORDER BY FOOTABLE.BAZCOLUMN} and such.  From there you use the
dbi's execute function to execute the query and subsequently construct
the objects.

\section{Refreshing an Object}
Ok, we've learned that the information in the current row has been
altered (by nefarious means - mwahahahah!  or not) in the database,
but we still hold the old information.  By calling the
\texttt{refresh()} method, it effectively does a \texttt{getUnique} on
itself and refreshes it's contents.  If the object no longer exists,
it raises an appropriate error.


\section{Deleting An object}
We're now done with this user object and want to dispose of it from
the database.  Using the \texttt{delete} method it issues an
appropriate DELETE query to the database.  Since it needs a unique
row, the usual things related to uniqueness mentioned above apply.


\section{Committing and Rollback}
Each PyDO object contains the two methods \texttt{commit()} and
\texttt{rollback()}.  These, in turn call the corresponding methods on
their database connection.  Normally there is little room for
confusion, but in the case of using multiple database connections
simultaneously, this could be a bit more confusing on what exactly is
getting committed.

\chapter{Tools}
ogenscript.py and pgenscript.py

\chapter{Adding Support for Another Database}
If you wish to add support for another database that is currently not
supported, you have to implement a class that follows the following
interface:
\begin{verbatim}
class interface:
    """don't actually inherit from me, this is just for documentation
    purposes"""
    def __init__(self, dbconnstr):
        """the dbconnstr is the conn str with the pydo:dbkind: bit
        chopped off"""

        self.bindVariables = 1 | 0 # 1 - I support bind variables, 0 - I don't
        pass

    def getConnection(self):
        """Get the actual database connection"""
        pass
    
    def bindVariable(self):
        """if you support bindVariables, return next bind variable name.
        suitable for direct inclusion into a sql query"""
        
    def sqlStringAndValue(self, val, attributeName, dbtype):
        """Returns a sql string and a value.  The literal is to be put into
        the sql query, the value should is put into the value list that is
        subsequently passed to execute().

        The reason for this is for example, using bind variables in the
        generated sql, you want to return a bind variable string and the
        value to be passed.  If doing such things requires state, you can
        clear it in resetQuery().

        """
        return "LITERAL", val

    def execute(self, sql, values, attributes):
        """Executes the statement with the values and does conversion
        of the return result as necessary.

        result is list of dictionaries, or number of rows affected"""
\end{verbatim}
\begin{verbatim}

    def convertResultRows(self, columnNames, attributeKinds, rows):
        """converts the result list into a list of dictionaries keyed
        by column name, and data type conversion specified by the
        attributeKinds dictionary (keyed by attribute, valued by database
        datatype).
        """
        
    def resetQuery(self):
        """Reset things like bind variable numbers if necessary before a query
        Need only if there is state between sqlLiterals because of bind
        variables, otherwise, don't need this.  Called before a query is
        executed.
        """

    def getSequence(self, name):
        """If db has sequences, this should return the sequence named name"""
        return 1

    def getAutoIncrement(self, name):
        """if things like mysql where can get the sequence after the insert"""
        return 1

    def typeCheckAndConvert(self, value, attributeName, attrDbType):
        """check values type to see that it is valid and subsequently
        do any conversion to value necessary to talk to the database with
        it, i.e. mxDateTime to database date representation"""
        
    def postInsertUpdate(self, PyDOObject, dict, isInsert):
        """to do anything needed after an insert or update of the values
        into table.  Specifically to handle cases like blobs where you
        insert/update with a new blob, but have to select for update and
        then deal with the blob post factum

        PyDOObject is the object being affected
        Dict is the dict of new values
        isInsert is a boolean stating whether or not this is an insert (true)
                 or an update (false).
        """
\end{verbatim}
Once that is done, an appropriate change to the \texttt{_driverConfig}
dictionary in the \texttt{PyDBI.py} file.

\cleardoublepage
\printindex
\end{document}

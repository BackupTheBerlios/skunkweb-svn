\documentclass{article}
\author{Frank Tegtmeyer, fte@pobox.com}
\title{Sending Mail with SkunkWeb}
\begin{document}
\maketitle

\begin{abstract}
This paper describes the use of the mail services included in SkunkWeb.
Sending mail seems to be easy but there are some technical and conventional
requirements that should be fullfilled. The incredible amount of web services
sending technically illegal mail or creating annoyances for users and mail
administrators is unbelievable.

You can do the world a favor by reading this document, following its
guidelines and thus avoiding all the trouble with mail services and
people.
\end{abstract}

\newpage
\tableofcontents
\newpage


\section{Social behaviour}

\subsection{The SPAM problem}

As you may know, SPAM is the most visible annoying thing regarding mail.
Of course your intention ist to get and keep loyal site users, so you
should do anything possible to avoid the slightest suspicion that your
messages sent through SkunkWeb are SPAM.

Here are some guidelines to keep your reputation of not being a spammer:

\begin{itemize}
  \item Always use ,,opt in''. Never use an address that comes from an
        untrusted or unverified source. The \textbf{intention} of the
	address owner to receive your mails \textbf{has to be verified}.
	An exception
	are mails that are technically ore procedural required in the
	case that there is already a relationship between the user and
	you.

	A short note here regarding opt-in: Some people, espacially those
	belonging to the ,,direct marketing guild'', use the term ,,double
	opt-in''. That's simply nonsense~- either the user opts in (with
	verification) or not. Don't trust anyone using the term double
	opt-in, most likely its a spammer (of course this will be denied
	by him).
  \item Never trade addresses~- not by providing them to others, not by
	accepting them from others.
  \item Always respect the users wish to not longer receive your mails.
        Provide a way for them to unsubscribe immediately and without
	your manual intervention.
  \item Handle complaints manually, not by an automated answer.
  \item Make your mails identifyable as coming from you. Possibly
        provide a reference to a webpage where the user may check the
	authenticity of your mail.
  \item Provide a contact address for complaints. Mails coming in at
        that address have to be handled of course.
  \item Handle bounces (non deliverable mails)~- you should unsubscribe
        addresses, that generate bounces everytime, after a while.
\end{itemize}

\subsection{Privacy and style}

Even in cases where the mail is not SPAM, people easily may get annoyed by it.
Always remember that your mails require attention and time of the receiver.
This time and attention is valuable, so don't waste it!

\begin{itemize}
  \item Use a good subject line. It's the key to users attention. It
        should give a summary of what the user has to expect from this mail.
  \item Use ,,inverse writing'' style~- start with a summary and then
        give the details for interested users (see Nielsen reference below).
  \item Use clear and concise language. Make your intention for sending
        this mail clear and, if possible, the mails value for the receiver.
	In any case there should be a value for the receiver or you will
	be recognized as an annoying factor.
  \item Provide a privacy statement. Explain what the users email address
        is used for and how it will be handled. Of course you must keep
	yourself bound to this statement. Never trade addresses.
  \item In case of HTML mails don't include web bugs. People don't like
        to be tracked. If you need such data, provide a way to get it from
	the users with their agreement. In general HTML mails should be
	avoided, they hurt security/privacy aware users and those with
	technically limited equipment (for example PDAs). Users that read
	mail offline will not get objects that are provided online
	anyway.
  \item Use an existing address as envelope sender\footnote{envelope senders
        are explained later}.
  \item In case of bouncing mails make the corresponding user aware of it
        in your web application.
	There may be a problem with its mail provider that
	the user should know about.
\end{itemize}

The usability expert Jakob Nielsen wrote about newsletters in his
,,alertbox''. This provides some valuable
insights into peoples behaviour and best practices for sending mail.
You can find the alertbox issue at\\
\texttt{http://www.useit.com/alertbox/20020930.html}.

\subsection{Technical correctness}

The number of technical incorrect implemented mailservices within web
applications is frustrating and it seems it gets worse with every new one.

Typically the ones that run these services don't know about the problems
or worse, they don't care. It's very frustrating for mail administrators
at the receiving sites to handle such situations.

In many cases the offending site is completely blocked as a countermeasure,
thus loosing the possibility to send any
mail to this mailserver again. Possibly the bad site is also reported
to some databases that collect information about such behaviour and
provide more administrators the possibility to block the site too.

It may be very hard to remove such database entries after they are created.
Once the services reputation is damaged, it takes a long time to improve
it again.

While Python and SkunkWeb save you from some errors that are common,
they still require you to put attention to some others.

Here are some errors that SkunkWeb/Python avoid:

\begin{itemize}
  \item bare linefeeds (see http://cr.yp.to/docs/smtplf.html for more
        information)
  \item missing Date: header
  \item missing, not required but very useful, message~id
  \item MIME encoding for ASCII (not a problem but not necessary)
  \item 8bit data without proper MIME headers
  \item sending with empty envelope sender
  \item sending with invalid envelope sender\\
        SkunkWeb uses the FromAddress if no other envelope sender
	is given. The FromAddress is possibly taken from the default
	value that is set in the sw.conf file or from the MailService module
	itself. The default value is ,,root@localhost'' which is not
	a valid envelope sender. You \textbf{have to} change this
	default value to a reachable address that is under your control.
\end{itemize}

Here are some problems that you have to avoid yourself:

\begin{itemize}
  \item encoding of multipart messages\\
        currently this is only supported via the ,,raw'' parameter
  \item encoding of character sets besides us-ascii and iso-8859-1
        is not supported (except through the ,,raw'' parameter)\\
        Because of compatibility reasons the email module is not
	used by SkunkWeb. The current limitations stem from the use
	of the rfc822, mimetools and mimify modules.
  \item use of invalid recipient addresses\\
        SkunkWeb does some basic syntax checks for the recipient addresses
	but this is a kind of ,,last fallback''. Always use valid,
	verified addresses.
  \item sending to addresses that do not or no longer exist\\
        Always ensure that you handle bounces, either by hand or, much
	better, automated. Addresses that are not reachable for a longer
	period must be removed from your database. If possible, the
	user should be made aware of that fact inside the web application.
\end{itemize}


\section{Internet mail basics}

While there are many resources that explain Internet mail, there
are only a few good ones that are complete \textbf{and} correct.
That's why this section is included here. It is also meant as a
reference that is quickly accessible.

\subsection{Simple Mail Transport Protocol (SMTP)}

Electronic mail is older than the Internet. It is possible to use
any transport mechanism that one can think of to move mail from the
sender to the receiver. Prominent mechanisms are UUCP (Unix to Unix Copy)
and SMTP. On the Internet nearly always the Simple Mail Transport
Protocol is used.

The protocol is text based and easy to debug. It is also easy to be used
for forgery. There is no authentication, no verification, no
encryption and no security at all. There are some enhancements that try to
improve the security of the protocol but they are strictly optional and
cannot be relied on.

The transport of messages is often done in one step from the sender
to the receiver. It is possible that the message travels through
multiple systems before reaching its destination. This builds a
,,store and forward'' system that moves the message closer to the
recipient in every step.

The content of the message is irrelevant to SMTP.
There are some requirements though - for example it is assumed that the
content is line oriented and that these lines don't exceed a given
length. Nevertheless good SMTP implementations don't have any problems
with overlong lines.

While the mail is travelling through the net, the recipients on the
envelope may require different transport paths. If this is the case, the
message (content) is copied and gets the necessary subset of the
recipients. This means that a mail system may get one message with
ten recipients and, for example, creates three copies which then are delivered
to three different target systems with three, five and two recipients on it.

To get more and complete information about SMTP, look at\\
\texttt{http://cr.yp.to/smtp.html}

\subsection{Headers and envelope}

SMTP implements the concept of messages that are moved
inside envelopes~- exactly like the postal service does with letters.
The SMTP message has exactly one sender on its envelope but may have
many recipients on it. Like explained above, the message may be copied
and the recipients divided into groups that have different transport
paths. At the destination system the message always has only
one envelope sender and one envelope recipient for final delivery.

While transport of the message through the net is mostly done over SMTP,
the final delivery is always a local mechanism like creating a file or
appending to a file.

What consequences has the use of envelopes?

First it is important to recognize that the headers of the message don't
have anything to do with the envelope. In fact the whole content of the
message is irrelevant for the delivery mechanisms. The message is simply
treated as a block of data with no special meaning.

To recall the postal service example: If you write two letters to two
different persons, prepare the envelopes and then put each letter into the
wrong envelope they will go to the wrong persons. The postal service has
no way to check the content of the letters to correlate them with the
envelope. The service is simply not interested (or at least should not
be) about what is inside the envelope.

The same is true for electronic mail. The transport mechanisms only look
for the envelope. They don't look into the headers and don't touch the
message itself. As always here are exceptions too. Sometimes mails are
converted from one encoding to another, although this should not be necessary
anymore on the Internet. Also every delivery step prepends the headers
with a Received: header that is a trace of the delivery path of the
message. An analogy would be the stamps on conventional envelopes that every
step in the postal delivery leaves. The Received: headers are also counted to
defend against mail loops. If they are above a limit (typically 20), the
message is bounced.

The second consequence is the handling of bounces (undeliverable mails).
As with the postal service (at least in Germany) the message is returned
to the envelope sender, if it is not deliverable. If your envelope sender
is wrong, \textbf{your} bounces will annoy some innocent victim.

If your envelope sender is an unreachable address itself, the system
that generates the bounce (the postal service)
is unable to return the message somewhere. In this case the message
(letter) will either be destroyed silently or handed over to a responsible
person for further inspection. This person will get annoyed of course too,
because it has to take action on errors that \textbf{you} made. In most cases
these situations are cleared by the victims with blocking the offending
addresses, or worse, servers.

Because this type of error (invalid envelope sender) is used by spammers to
hide their identity and avoid all the bounces that their incorrect address
pools generate, many sites do intensive checks on envelope data. If they see
not reachable envelope senders, your mail will be blocked.

Of course you want to get your messages through to the interested web users,
so it is not in your interest to have wrong or undeliverable envelope
senders.

The standard envelope sender address that skunkweb defines is
root@localhost (also defined in sw.conf) which is not an address that
makes bounces deliverable. The MailServices module
now refuses to send messages with this envelope sender to force
you to change the value.


\subsection{Interfaces for injecting mail}
\label{sec:interfaces}.

There are two main interfaces for injecting mail on a Unix system.

The first is use of the SMTP protocol. The client or program connects
to the SMTP port of the host, hands over the message and then closes the
connection. This type of injection is supported by SkunkWeb too. The
mechanism is called ,,relay'' in SkunkWeb and set through the ,,MailMethod''
configuration variable. Additionally you have to set the ,,MailHost''
variable.

The second common interface is a call to the sendmail binary. Nearly all
Unix systems still ship with the dangerous sendmail program, although
there are some very good alternatives now. If other systems are used,
they mostly provide a replacement, that behaves like the sendmail
program. This is true at least for qmail and postfix.
Mails are piped to the standard input of the sendmail binary,
while the envelope sender and
the envelope receivers are given as arguments on the commandline.
Please notice that
this puts some constraints on the number of recipients, because they have
to fit into the available buffer for the sendmail commandline.
Consult your manuals or experiment to know the limits of your system.
All up to 100 receivers should be no problem.\\
Setting the envelope sender on the commandline produces a nasty security
warning with sendmail. You have to add the user running SkunkWeb to the
trusted users in sendmail to avoid this.\\
The sendmail variant is called ,,sendmail'' in SkunkWeb. It has to be set
through the ,,MailMethod'' configuration variable. There is an additional
variable ,,SendmailCommand'' that should not be changed. It points to the
used sendmail binary.

SkunkWeb also supports a third interface. Its called ,,qmail'' and has also
to be set through the ,,MailMethod'' configuration variable. At the moment
it works exactly like the sendmail interface. In the future it will support
variable envelope senders that can be used to track bounces per recipient
and per message. The messages are injected to the standard input of the
qmail-inject program, envelope sender and recipients are set as arguments
on the commandline. The same restrictions as for the sendmail method
apply (except the security warning).

In case your qmail binaries don't live under /var/qmail/bin, the path
to the qmail-inject binary may be set through the configuration variable
,,QmailInject''.


\section{Usage patterns}

Sending mail is not only a technical problem. Depending on the task
that should be done by sending mail there are some things that have to
be thought about before. Problems for your reputation, for the security
of your service or for performance may arise if you blindly implement
some kind of mail sending service.

Especially dangerous are all services that enable a user to send mail
to another one through your service. Such services are subject to abuse
and should contain some damage limiting restrictions or should be avoided
generally.

\subsection{Administrative mails}

Administrative mails are the ones with the least problems. They are
required by procedures (for example password change of a user) or
triggered by some state of the web application. They should be sent
only when really necessary, or when they are requested by the user.

Of course the given guidelines for mail apply here too. In general mails
should be as polite as possible while giving instructions or informations.
In English this should not be a big problem, but there are languages
that use different words depending on the relationship between the
,,talkers''. German for example uses ,,Du'' for a more personal touch and
,,Sie'' for official style or between persons not knowing each other.
The most heavy differences, to my knowledge, has Japanese, which
establishes many layers of differentiation. In any case where ,,the system''
or ,,the machine'' talks to the user, the machine should be seen as slave
of the user. This has to be reflected by the used language.
A survey of users of a german BBS network showed that people like
to get the more official ,,Sie'' within system messages even if in general
the personal ,,Du'' was used between users of the network. As always
your users culture may be different.

\subsection{Newsletters and Mailinglists}

Newsletters and mailinglists are generally better implemented outside
of SkunkWeb. A web application is simply not the right tool for that
task. SkunkWeb may be used to implement a frontend for the management
of newsletters and mailinglists, but the sending process itself should
be done through available mailinglist managers (ezmlm, MailMan,
Majordomo, \ldots) or, if they don't fit, through cron jobs.

It's possible to prepare a newsletter through SkunkWeb. The newsletters
properties may be stored in a database. The sending itself should be done
later with a cronjob which gets the necessary data from the database
and reports failure or success also through the database.

If you don't listen to that advice, at least watch out for the limits
of your sending interface. This is discussed in section
\ref{sec:interfaces}.

\subsection{Feedback}

Feedback forms don't require much attention besides setting a valid
envelope sender and receiver (see below for a common security desaster).
Because you (or your customers) receive
the messages, the used form and language is completely up to you.
If you like funky messages, use them until they begin to annoy you
(nothing annoys more than to receive the same gag twenty times a day).

You should use different recipients for different types of feedback.
The webmaster is for sure not the right receiver for complaints about
long delivery times for goods or incorrect statements in articles that
are posted by the marketing department.

There is a potential security desaster (that is unfortunately very common)
with feedback forms. Never, I repeat, \textbf{never} give the recipients
email address as parameter to the script that handles the users form
input!

What are the problems with this approach? 

First you may want to keep the recipient addresses secret, for example
to keep them SPAM free. It may seriously impact productivity if critical
addresses are invaded by SPAM. Besides that it's simply annoying and you
may delete important mail from users by accident while deleting SPAM.

Second this practice makes your web server a tool for spammers and
other people that don't want to be traced. If they recognize that your
application sends mail on their behalf, they will use it. A prominent
software package that enables this misuse is formmail from Matt's script
archive that is still widely used today.

To avoid this problem, you may hardcode the receiver in your script. If
the script is used from many places in your application and needs to be
usable for different receivers, code them in a table and use the codes
in your forms. If a request uses a nonexistent code, show an error
message. The code table can be in the script code itself or in some
configuration area but never in the document space that is presented to
users.

If you see HTML like the following, the alarm bells in your head should
ring loudly:

\begin{scriptsize}
\begin{verbatim}
<form action="/sendmail" method="POST">
<input type="hidden" name="receiver" value="service@example.com" />
...
</form>
\end{verbatim}
\end{scriptsize}

The following is o.k. It should be noted that the value can be
anything, as long as it is used as a key to the real address.

\begin{scriptsize}
\begin{verbatim}
<form action="/sendmail" method="POST">
<input type="hidden" name="receiver" value="11" />
...
</form>
\end{verbatim}
\end{scriptsize}


\subsection{,,Send to a Friend''}

This is the most problematic application of electronic mail in web
services because the sender and the receiver of the messages are under
control of the visiting web user. Often they are used for postcard
services or to send interesting articles to other people.

To avoid misuse of such a service, the following countermeasures
should be established:

\begin{description}
  \item[envelope sender] ~ \\
       The envelope sender should be set to an address that is under
       your control. If possible it should encode the given receiver
       so that a script is able to extract the receiver address.
       A bouncing receiver should not be accepted as receiver again
       (at least for a while). Clients that (within a given time)
       repeatedly send mails to bouncing addresses should not be
       allowed to use the service again.
       The clients may be identified by a cookie or other data.
  \item[local blacklists] ~ \\
       Users should be able to enter their address into local blacklists
       that the sender and receiver are checked against. The entries
       may have a limited lifetime. Of course the addresses have to be
       verified by the system to avoid blacklisting of users that want
       to use the service.\\
       Clients that repeatedly try to use blacklisted addresses should
       not be allowed to use the service again.

\end{description}


\section{The sendmail function}
\subsection{Basic parameters}
\subsection{Extended parameters}
\subsection{Exceptions}

\section{The sendmail tag in STML}
\subsection{Parameters}
\subsection{Formatting}

\section{Self formatted mails (raw parameter)}
\subsection{Preparation with email module}
\subsection{Required headers}
\subsection{Sending attachments}

\section{Handling bounces}
\subsection{Qmail}
\subsection{Postfix}
\subsection{Sendmail}
\subsection{Exim}
\subsection{Manual handling}

\section{Example for a feedback form}
\subsection{STML}
\subsection{Python}

\end{document}

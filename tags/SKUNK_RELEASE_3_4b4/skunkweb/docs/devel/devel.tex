%The Developer guide could break down
%into subsections for each service and talk about what's available from
%STML/Python and what the APIs are.  Then a section on how to write a new
%service.

\documentclass{manual}
\usepackage{makeidx}
\usepackage[nottoc]{tocbibind} % make so bib and ind are in toc
\title{The SkunkWeb Developers Guide}
\author{Drew Csillag and Jacob Smullyan}
\release{3.3}
\setshortversion{3}
\makeindex

\begin{document}

%my commands here
\newcommand{\Location}{\texttt{Location}}
\newcommand{\PAR}{\texttt{par}}
\newcommand{\mskunkroot}[1]{\textit{SkunkRoot}\texttt{#1}}
\newcommand{\swpython}{\texttt{swpython}}
\newcommand{\connection}{\texttt{Connection}}
\newcommand{\NOTE}[1]{\textbf{\Large #1}}
\newcommand{\None}{\texttt{None}}
\newcommand{\nolocation}{[N]}
\newcommand{\service}[1]{\texttt{#1}\index{\texttt{#1}}}

%function argument list environment
\newcommand{\argdescitem}[1]{\hspace\labelsep
                                \normalfont\ttfamily #1\ }
\newenvironment{argdesc}{\begin{list}{}{
        \renewcommand{\makelabel}{\argdescitem}
}
}{\end{list}}

%begin titlepage stuff
\maketitle
\ 
\vfill 
\noindent
This file documents the SkunkWeb Web Application Framework.

\noindent
Copyright (C) 2002 Andrew T. Csillag, Jacob Smullyan

\noindent
Permission is granted to make and distribute verbatim copies of
this manual provided the copyright notice and this permission notice
are preserved on all copies.

%%end titlepage stuff

\tableofcontents

\chapter{Introduction}
This document is intended for the SkunkWeb developer that needs to get
more into the inner guts of the SkunkWeb server.  This document will
describe the inner architecture of the server, the file formats used,
the core APIs, the APIs of foundation services,
as well as how to
write SkunkWeb services.

It is assumed that you are familliar with the SkunkWeb server,
i.e. you don't need to be told that what \texttt{<:component
foo.comp:>} would do, and that you know Python well.

\chapter{Services}
The crux of most of what SkunkWeb does lies in services, of which
there are four kinds.
\begin{description}
\item[Provider] Provides direct functionality, but few, if any API functions.
\item[Linkage] Mainly links in some external (from \texttt{pylibs})
code, and may provide functionality of it's own.
\item[API] Provides little, if any functionality on it's own, but
provides API functions to do various things.  
\item[Foundation] Services that form the foundation that services of other types are built upon.
\end{description}

\section{Provider Services}
Many of the provider services are protocol adaptors, those being the
\service{aecgi}, \service{fcgiprot}, \service{scgi}, \service{httpd}
and \service{remote}, whose purpose is merely to provide support for a
network protocol.

There are others in this category.  These are the services that
provide direct functionality and really don't have any API to speak of.

The \service{basicauth} service, while deprecated, provides the
ability to do basicauth (i.e. browser-based) authentication.  It has
an API inasmuch setting of the
\texttt{REMOTE_USER}\index{\texttt{REMOTE_USER}} and
\texttt{REMOTE_PASSWORD}\index{\texttt{REMOTE_PASSWORD}} is an API.
The newer \service{auth} service provides the functionality that the
\service{basicauth} service provides and more and is the replacement
now that \service{basicauth} is deprecated.

The \service{rewrite} service allows for some pre-request URL
rewriting.  In addition, named groups in the matched regular
expression will be added to \texttt{CONNECTION.args} if the
appropriate configuration variable is set.  See the SkunkWeb
Operations Manual for details.

The \service{userdir} service allows you to provide simple SkunkWeb
services to the users on your machine.  If the URL starts with
\texttt{\~}\emph{username}, it will serve the pages from the directory
named by the \texttt{userDirPath} configuration variable
(\texttt{public_html} by default) in their home directory.

The \service{extcgi} service allows SkunkWeb to directly handle CGI
programs.  This is useful in the event that you either aren't running
Apache (or some other SkunkWeb compatible server), or that your
configuration is such that having your main HTTP server handle it is
either not aesthetically pleasing or just plain annoying.

The \service{pycgi} service also allows SkunkWeb to directly handle
CGI programs, but \service{pycgi} only works for CGI programs that are
Python scripts.  It handles them in such a way that, unlike
\service{extcgi}, does not require a call to \texttt{fork}, so should
provide better performance in the case that you are running a Python
CGI program.

\section{Linkage Services}
Linkage services mainly exist to link other stuff into skunkweb.
\begin{description}
\item[mysql] Loads the \index{MySQL} MySQL pylib library to do
connection caching.  See Appendix \ref{Database APIs}.
\item[postgresql] Loads the \index{PostgreSQL} PostgreSql pylib
library to do connection caching.  See Appendix \ref{Database APIs}.
\item[oracle] Loads the Oracle \index{Oracle} pylib library to do
connection caching.  See Appendix \ref{Database APIs}.
\item[ae_component] Loads the AE pylib library that is the component
rendering infrastructure, see Appendix \ref{aelib}.
\item[templating] Hooks up the \service{ae_component} service up to
the \service{web} service to make templates web accessible.
\item[psptemplate] Hooks up the psp pylib to the AE infrastructure.
\end{description}


\section{API Services}
The API services are those that could not (easily) be arranged to
otherwise be linkage services, as they are more tightly integrated
because of the tasks they perform, and they provide some direct
functionality also.

       auth
       product 
       remote\_client
       sessionHandler

\subsection{The \service{auth} Service}
\begin{verbatim}
class authorizer:
    def __init__(self, ......):
    """
    The ...... will be filled with the contents of
    Configuration.authAuthorizerCtorArgs when this object is instantiated.
    """

    def checkCredentials(self, conn):
    """
    Examine the connection however you see fit to see if the
    connection has the credentials needed to view the page
    return true to accept, false to reject
    """

    def login(self, conn, username, password):
    """
    Take the username and password, validate them, and if they are
    valid, give the connection credentials to validate them in the
    future (i.e. make it so checkCredentials() returns true).  If your
    authorizer is doing basicauth-style authentication, this method is
    not required.
    """

    def logout(self, conn):
    """
    Remove any credentials from the browser.  This method is not required
    for methods (specifically, basicauth) that do not have the concept of
    logging out.
    """

    def authFailed(self, conn):
    """
    If the connections credentials are rejected, what to do.  For
    this, unless you are using a basic-auth means, inheriting from
    RespAuthBase is probably sufficient (it will bring up a login
    page), otherwise inheriting from BasicAuthBase and providing an
    appropriate validate function is sufficient.
    """

    #this method is required only if using one of the base classes that
    #require it
    def validate(self, username, password):
    """
    Return true if the username/password combo is valid, false otherwise
    """


\end{verbatim}
\begin{verbatim}
class RespAuthBase:
    """
    This is a base class for those authorizers that need to use a login page.
    That's pretty much any of them, with the exception of BasicAuth, as it
    doesn't need it since the browser "puts up a login page".

    The loginPage ctor arg is the page to show.  It *can* be in the protected
    area.
    """


class AuthFileBase: #base class for those that auth against a basicauth file
    """
    Base class for authorizers that will use a simple basicauth file to
    validate user/password combinations

    The authFile ctor argument is the file that we will validate against
    """


class BasicAuthBase: #base class that does basicauth
    """
    This class does browser based basicauth authentication where it pops up
    the login box by itself.

    This is a mixin class and is not usable by itself.  You must provide a
    way to validate (by subclassing BasicAuthBase and defining a validate
    function) the username and password that is obtained."""

class CookieAuthBase: #class that does basic cookie authentication
    """
    This class implements a simple cookie based authentication.  Depending
    on how you want to do things, you may not (and probably don't) want to
    have the username and password encoded directly in the cookie, but
    definitely *do* use the armor module to protect whatever you do decide to
    put into the cookie as it will make them virtually tamperproof.

    As with BasicAuthBase, thisis a mixin class and is not usable by
    itself.  You must provide a way to validate (by subclassing CookieAuthBase
    and defining a function) the username and password that is obtained.
    """

\end{verbatim}
\begin{verbatim}
class SessionAuthBase: #class to do auth using sessions
    """
    This class uses the session object (you must have the sessionHandler
    service loaded) to handle authentication.  This is also probably not 
    something that you would really want to use in a production setup as:
    a) you probably don't want to store the password in the session.  Not that
       it's inherently evil, but it's one less thing you have to worry about
       leaking in the event that something gets screwed up.
    b) checking the credentials might just be to check if the username is set
       since all of the session info (minus the session key) is stored on
       the server and not the browser, we don't have to worry as much that
       somebody fooled around with us

    But what is here is usable as a starting point and will run.

    As with BasicAuthBase, thisis a mixin class and is not usable by
    itself.  You must provide a way to validate (by subclassing SessionAuthBase
    and defining a function) the username and password that is obtained.

    """
\end{verbatim}



\subsection{The \service{product} Service}
The \service{product} service is a service to support SkunkWeb
products, which can be installed as archive files (zip, tar, or tgz),
or as directories in the SkunkWeb product directory.  This service
requires that the documentRootFS be a MultiFS (at least prior to
scoping); the assumption is that MultiFS will become the default fs.

The product loader is configured to load by default all product
archives in the product directory, but can be configured to load any
arbitrary subset.

The loader opens the MANIFEST file in each archive, looks for
dependencies stated therein, and loads any products therein listed,
raising an error for circular or missing dependencies.  For each load,
it adds mount points to the MultiFS for the docroot and, by means of
an import hook in the vfs which permits python modules to be imported
from the vfs, adds the libs (if any) to sys.path.  Services specified
in the MANIFEST are then imported.

The MANIFEST is essentially an init file for the product which
integrates it into the SkunkWeb framework.  It must contain the
product version, and can also contain product dependencies and
services (python modules that use SkunkWeb hooks, to be imported at
product load time).  See manifest.py for details.

By default, a product's docroot will be mounted at
products/<product-name>, relative to the SkunkWeb documentRoot.  This
can be altered by modifying Configuration.defaultProductPath, which
will affect all products, or by adding the product-name to the
Configuration.productPaths mapping.

This service will also contain a utility for creating products, creating
the MANIFEST file, byte-compiling the python modules, and creating the
archive file.

\subsection{The \service{remote\_client} Service}
The \service{remote\_client} service exists to provide access to
components exposed on other SkunkWeb servers via the
\texttt{AE.Component.callComponent} function, and by extension the
\texttt{<:component:>} tag.  What this service does is make it so that if you specify the component name as \texttt{swrc://}\emph{host}\texttt{:}\emph{port/path/to/component.comp}, it will call the component \emph{/path/to/component.comp} by contacting a (presumably) existing SkunkWeb installation that has the \service{remote} service configured to listen on \emph{host}\texttt{:}\emph{port}.

\subsection{The \service{sessionHandler} Service}

The \service{sessionHandler} service adds three functions to the
\texttt{CONNECTION} \index{\texttt{CONNECTION}} object that the
top-level component sees.

The three functions are:
\begin{argdesc}
\item[getSession(create=0, **cookieParams)] if create is true, will
create (and return) a new session (in the cookie, with extra paraters
in \texttt{cookieParams}), otherwise will return an existing session, if it
exists, otherwise returns \texttt{None}.
\item[getSessionID(create=1)] get session id from cookie, or create it if \texttt{create} is true, otherwise returns \texttt{None}.
\item[removeSession()] kills the session cookie.
\end{argdesc}



\section{Foundation Services}
Foundation services are those which are used to build other services,
but in and of themselves, provide no directly useable functionality.

\subsection{The \service{requestHandler} Service}
The \service{requestHandler} service is used as the basis of all of
the network request based services (directly, or indirectly) and
handles the communication to the client (in whatever form that may
take).

PreemptiveResponse (responseData) -- respdata will be sent via
protocol.marshalResponse

from requestHandler/requestHandler.py
Hook sequences

BeginSession  Just after connection is gotten, scoped for IP,PORT or UNIXPATH (job, sock, sessionDict)

while 1:
    InitRequest     after the request data has been received (job, reqdata, sessiondict)
    HandleRequest   called to actually handle the request (job, reqdata, sessiondict)
    PostRequest     called after response sent (job, reqdata, sessionDict)
    CleanupRequest  right after PostRequest (job, reqdata, sessionDict)

EndSession called when session ends.



has DocumentTimeout stuff

requestHandler.requestHandler.addRequestHandler(handler -- should be
    subclass (or provide the same interface as)
    requestHandler.protocol.Protocol, ports)


requestHandler.protocol.Protocol

\begin{verbatim}
class Protocol:
    """
    abstract class for protocols used in handling request and response
    """

    def marshalRequest(self, socket, sessionDict):
        '''
        should return the marshalled request data
        '''
        raise NotImplementedError

    def marshalResponse(self, response, sessionDict):
        '''
        should return response data
        '''
        raise NotImplementedError

    def marshalException(self, exc_text, sessionDict):
        '''
        should return response data appropriate for the current exception.
        '''
        raise NotImplementedError

    def keepAliveTimeout(self, sessionDict):
        '''
        how long to keep alive a session.  A negative number will terminate the
        session.
        '''
        return -1
\end{verbatim}


\subsection{The \service{web} Service}

from web/protocol.py

--CONNECTION object is constructed from reqHandler.reqdata
HaveConnection (job, conn, sessionDict)
--Configuration is scoped
PreHandleConnection (job, conn, sessionDict)
HandleConnection (job, conn, session)

has Redirect exception

Contains the CONNECTION definition

\chapter{Templating Languages}
\begin{description}
\item[STML] SkunkWeb Template Markup Language
\item[PSP] Python Server Pages
\item[Cheetah] The Cheetah templating system (http://www.cheetahtemplate.org)
\item[Python] Not really a templating language, but included for
completeness
\end{description}

\section{Common to All}
\texttt{CONNECTION} exists in all of them (in toplevel) in the global
namespace.  URL (GET and POST) arguments show up in
\texttt{CONNECTION.args}.  Can import SkunkWeb API modules (might be
services).  Can be components (regular and data), cacheable, etc.
Component args show up in global namespace.

\section{PSP Specifics}

Only $<$\% and \%$>$ for now, no $<$\%= \%$>$ pairs ... yet?  No other
PSP-style tags either.

\section{Cheetah Specifics}

Precompiled Cheetah templates are executed via the normal Python code
execution path.  Direct support for Cheetah templates without
compilation (i.e. being able to go to
http://foo.bar/a_cheetah_doc.tmpl) is nonpresent in SkunkWeb right
now, but if there is demand, it could, of course, be added.

For Cheetah templates, \texttt{\$var} won't fetch item from global
namespace by default, so put this at the top of your Cheetah templates
and you'll be good to go.
\begin{verbatim}
#silent self._searchList.append(globals())
\end{verbatim}
Caching in Cheetah templates probably doesn't work in SkunkWeb, or if
it does, is not as effective as SkunkWeb caching because SkunkWeb does
not use threads.  If SkunkWeb did use threads, then they would be
equally effective in a single-machine environment.  In a multi-machine
environment, SkunkWeb caching is more effective because the cache can
be shared between the machines as it is disk-based.

\chapter{Writing a Service}
   Hooks
   reload gotchas

\chapter{Writing a New STML Tag}
   empty tags
   block tags
   debugging and vicache.py
   metadata

\chapter{Adding a Templating Type}

\begin{enumerate}
\item templating compiler (preferably) or interpreter
\item cached compiled vsn fetcher, or interpretable form instantiator
\item AE Executable
\item mime.types entries
\item adding applicable mime types to AE.Executables.executableByTypes, perhaps overriding those for text/html and text/plain
\item add to interpretMimeTypes somehow
\end{enumerate}


%%%%%%%%%%%%%%%%%%%%%%%%%%%%%%%%%%%%%%%%%
%\chapter{Core Services}
%The core services that SkunkWeb supplies are (intentionally) fairly minimal. 
%The services provided are:
%\begin{itemize}
%\item The reading of configuration file(s)
%\item The loading of Python modules or packages
%\item A logging service
%\item A multi-process TCP server framework
%\item Two hooks, one for pre-\texttt{fork} initialization and one for
%post-\texttt{fork} initialization.
%\item An interactive Python prompt (a.k.a. \texttt{swpython})
%\end{itemize}
%
%\section{Configuration}
%Configuration proceeds by first checking to see if a \verb!-c! or
%\verb!--config-files=! option was specified (as a colon delimited list
%of configuration files to read), if not, a configuration-time
%hardcoded default is used.  A namespace is set up with a small amount
%of preloaded stuff \footnote{The \texttt{Scope} ``directives'' is loaded and the default \texttt{SkunkRoot} is set}
%(and the configuration files executed (they're
%Python programs really) in this namespace in the order given on the
%command line (if given).
%
%\subsection{The \texttt{SkunkWeb.Configuration} ``Module''}
%
%The current configuration is available via the
%\texttt{SkunkWeb.Configuration} ``module''.  In reality, it is not a
%real Python module, but a class instance that has been set up to be
%importable.  The implementation of the ``module'' is the
%\texttt{ScopeableConfig} class in  \verb!pylibs/ConfigLoader.py!.
%(\texttt{ScopeableConfig} is itself a \texttt{scopeable.Scopeable} subclass
%with a backwards compatibility layer that will probably be removed soon.)
%
%\begin{methoddesc}{mergeDefaults}{dict} Function taking any number of dictionary
%arguments, and/or keyword arguments. For each key/value in each dictionary, (with 
%any keyword arguments comprising a final dictionary), if the current
%configuration already has a variable with that name, it is skipped,
%otherwise a the configuration variable with that name is set to the
%corresponding value. 
%\end{methoddesc}
%
%\begin{methoddesc}{update}{dict} Updates the root configuration with the
%contents of the argument (argument should be a dictionary-like
%object). 
%\end{methoddesc}
%\begin{methoddesc}{push}{} Overlay configuration with the contents
%of the argument (should be a dictionary-like object). 
%\end{methoddesc}
%\begin{methoddesc}{pop}{} Undo the last overlay.
%\end{methoddesc}
%\begin{methoddesc}{trim}{} Undo all overlays, resulting in just the
%root configuration. 
%\end{methoddesc}
%\begin{methoddesc}{mash}{} No argument function that returns a
%dictionary whose contents are the current view of the configuration.
%\end{methoddesc}
%\begin{methoddesc}{mashSelf}{} Sets the root configuration to
%\verb![self.mash()]!. 
%\end{methoddesc}
%
%Since the configuration is read prior to the services being loaded,
%the configuration is available to services at their time of import.
%
%The configuration variables show up as attributes in the module
%\verb!SkunkWeb.Configuration!.  Thus, the configuration
%variable \texttt{debugFlags} would appear as
%\verb!SkunkWeb.Configuration.debugFlags!.
%
%\section{Service Loading}
%The loading of services is done by the normal Python import facility.
%This being the case, the directory(ies) that the service
%modules/packages live in \strong{must} be in \texttt{sys.path}.  When
%SkunkWeb is installed, the services shipped with SkunkWeb are placed
%in a directory that is added to \texttt{sys.path} by the bootloader
%automatically.  If you decide to move them, or add services of your
%own (which should be located elsewhere so that future upgrades won't
%accidentally wipe it out), you can add paths to \texttt{sys.path} in
%the configuration file like so:
%\begin{verbatim}
%import sys
%sys.path.append('/some/other/path')
%\end{verbatim}
%
%Even though a service sounds like a big deal, services are merely
%Python modules and/or packages that are loaded 
%just after the configuration has been read in.  Conventionally though,
%service packages/modules assume presence of the SkunkWeb environment
%(i.e. configuration information, hooks, etc.) and plug into the
%SkunkWeb evironment in some way.
%
%
%\section{Logging}
%Logging is done by the \verb!SkunkWeb.LogObj! module.  The main
%functions of interest are:
%
%\begin{funcdesc}{DEBUG}{debugFlag, message} Takes two arguments, a debug flag (explained below) and a
%message string.  If the flag masked against \verb!Configuration.debugFlags!
%results in a non-zero value, the message is logged to the debug flag.
%\end{funcdesc}
%\begin{funcdesc}{DEBUGIT}{debugFlag} Take a single argument, a debug flag.  If the flag
%masked against \verb!Configuration.debugFlags! 
%results in a non-zero value, returns 1, else 0.  Useful for doing
%computation for debug messages that you don't want to do if you aren't
%going to log said message.
%\end{funcdesc}
%\begin{funcdesc}{LOG}{message} Takes one argument, the message to log.  Used for logging
%normal, but significant events.
%\end{funcdesc}
%\begin{funcdesc}{WARN}{message} Takes one argument, the message to log.  Used for logging
%events that require attention.  Logs to the error log file. 
%\end{funcdesc}
%\begin{funcdesc}{ERROR}{message} Takes one argument, the message to log.  Used for logging
%erroneous events.  Logs to the error log file.
%\end{funcdesc}
%\begin{funcdesc}{ACCESS}{message} Takes one argument, the message to log.  Used mainly by
%the \texttt{AE} library to log access events, but is available for
%general use.  Logs to the access log.
%\end{funcdesc}
%\begin{funcdesc}{logException}{} Logs the current exception to the error log.
%\end{funcdesc}
%
%
%If, when SkunkWeb is started, standard out and/or standard error are
%not TTY's (i.e. hooked to a terminal), they will be redirected to the
%\texttt{LOG} nad \texttt{ERROR} functions, respectively.
%
%\section{The \texttt{Server} Object}
%The \texttt{Server} object handles the routing of TCP connections to
%appropriate handlers, presumably installed during the loading of
%Python modules, to handle whatever protocol is spoken over that TCP
%connection. 
%
%In the \texttt{SkunkWeb.Server} module there is a single function of
%interest: 
%\begin{funcdesc}{addService}{sockaddr, func}
%Opens up a socket and listens for connections, and when connections
%arrive calls \texttt{func} (with the \texttt{accept()}ed socket to
%handle them. 
%\begin{argdesc}
%\item[sockaddr] An address specfier of one of two forms:
%\begin{enumerate}
%\item \texttt{('TCP', 'hostname', port)} --- bind to the TCP address
%\texttt{host}/\texttt{port} where \texttt{port} is an integer.  To
%bind to all addresses on host, specify the empty string as the
%hostname.
%\item \texttt{('UNIX', '/path/to/socket' [,
%\textit{socket_permissions} ])} --- bind to the Unix socket located at
%\verb!/path/to/socket! and optionally set the permissions of the
%socket to \texttt{\textit{socket_permissions}}.
%\end{enumerate}
%\item[func] A function to call (with the socket) when a connection
%comes in.
%\end{argdesc}
%\end{funcdesc}
%
%\section{Hook Objects}
%Hook objects are used in SkunkWeb to provide extensibility.  They
%allow you to setup functions to be called at specific points in
%processing.  They can be viewed as a list of functions to call 
%(the \texttt{Hook} class is actually a subclass of \texttt{UserList})
%that calls the functions in the list, in order, until one of them
%returns a non-\None\ value.  The return value of calling a hook then,
%is the return value of the function that returned a non-\None\ value,
%or if all of the calls returned \None, the return value is \None.
%
%\section{Initialization Hooks}
%
%\begin{datadesc}{ServerStart}  Called after services have been imported but
%before the server \texttt{fork()}s.
%\end{datadesc}
%\begin{datadesc}{ChildStart}
%Called in the child process just after the \texttt{fork()}
%\end{datadesc}
%
%
%%%%%%%%%%%%%%%%%%%%%%%%%%%%%%%%%%%%%%%%%
%\chapter{The \texttt{web} Service}
%\index{services!\texttt{web}}
%The \texttt{web} service, is short provides base request phase
%handling (extendable/configurable via hooks), an \connection\ object
%and socket connectivity to 
%Apache (via \texttt{mod\_skunkweb}).
%
%
%\section{Protocol}
%The protocol between the web listener (presumably Apache, but could be
%something else) is as follows:
%\begin{enumerate}
%\item Client connects to server at some prearranged port
%(\texttt{WebListenPorts}).
%\item Server sends one byte (the byte value should be 0)
%\item Client sends 10 byte ascii length of request (e.g. a length of
%twenty would be sent as the ascii string \verb*!        20! (a 
%\verb*! ! is a blank space)).
%\item Client sends Python marshalled form of a dictionary whose
%key/value pairs are:
%\begin{description}
%\item[\texttt{'stdin'}] the request body (in CGI-land, it's standard
%input)
%\item[\texttt{'headers'}] a dictionary of HTTP headers
%\item[\texttt{'env'}] the CGI environment variables.
%\end{description}
%\item Server sends 10 byte ascii length of request --- as in (3).
%\item Server sends full response (headers and body).
%\item Both sides close connection.
%\end{enumerate}
%
%\section{Hooks Exposed}
%
%\begin{datadesc}{InitRequest} Called with the raw request data as received from
%Apache.  If the return value is a tuple of
%(\textit{HTTP\_RESPONSE\_CODE}, string),  request processing stops and
%the string (which is assumed to contain headers also) is sent back as
%the response.
%\end{datadesc}
%\begin{datadesc}{HaveConnection} Called after the \connection\ object has been
%constructed but before the location specific configuration has been
%overlayed.
%Argument is the \connection\ object.  If the
%return value is an HTTP response code, the \connection\  object
%is assumed to be complete and a response is sent.
%\end{datadesc}
%\begin{datadesc}{PreRequest} Called after the location specific configuration has been
%overlayed.  Argument is the \connection\ object.  If the
%return value is an HTTP response code, the \connection\  object
%is assumed to be complete and a response is sent.
%\end{datadesc}
%\begin{datadesc}{HandleRequest} Called after the \texttt{PreRequest}
%hook. Argument is the \connection\  object.  If the
%return value is an HTTP response code, the \connection\  object
%is assumed to be complete and a response is sent.  If no response code
%is returned in the calling of this hook, a \texttt{'NoHandler'} string
%exception is raised and processing proceeds to the
%\texttt{RequestFailed} hook.
%\end{datadesc}
%\begin{datadesc}{PostRequest} Called after the response has been sent. Arguments
%are the \connection\  object and the raw request data.  If the
%exception occurs before the \connection\  has been constructed,
%the connection argument is \texttt{None}, otherwise the raw request
%data argument is \texttt{None}.
%\end{datadesc}
%\begin{datadesc}{CleanupRequest} Called after the \texttt{PostRequest} hook. Arguments
%are the \connection\  object and the raw request data.  If the
%exception occurs before the \connection\  has been constructed,
%the connection argument is \texttt{None}, otherwise the raw request
%data argument is \texttt{None}.
%\end{datadesc}
%\begin{datadesc}{RequestFailed} Called if there is an exception before a response
%has been sent.  It should return the text of a full response
%(i.e. headers and body).  After this hook returns (even if by a raised
%exception), the \texttt{PostRequest} and \texttt{CleanupRequest}
%handlers are called.
%\end{datadesc}
%
%\section{The \connection\  Object}
%The \connection\ object is used to obtain information about the
%request and to produce a response.  As such, it contains things like
%the CGI arguments sent from forms, and things like the request and
%response headers.  The attributes of the \connection\ object are as
%follows:
%
%\begin{memberdesc}[dictionary]{\_initRequestData}
% the raw data that the \connection\ object is
%constructed from.  Is the same data that the \texttt{InitRequest} hook
%is called with.
%\end{memberdesc}
%\begin{memberdesc}[integer]{\_status}
% The HTTP response code to send back to Apache.  If not
%otherwise set via call to \texttt{setStatus()}, is 200.
%\end{memberdesc}
%\begin{memberdesc}[dictionary]{env}
% The CGI environment set by Apache.
%\end{memberdesc}
%\begin{memberdesc}[string]{\_stdin}
%The request body text sent by the browser to Apache.
%\end{memberdesc}
%\begin{memberdesc}[HeaderDict]{requestHeaders} The HTTP headers of the request.
%\end{memberdesc}
%\begin{memberdesc}[Cookie]{requestCookie} The cookie sent by the browser.
%\end{memberdesc}
%\begin{memberdesc}[HeaderDict]{responseHeaders} The HTTP headers to be
%sent in the response. 
%\end{memberdesc}
%\begin{memberdesc}[Cookie]{responseCookie} The cookie to be sent in
%the response.
%\end{memberdesc}
%\begin{memberdesc}[File]{\_output} A file-like object that contains
%the body of the response. 
%\end{memberdesc}
%\begin{memberdesc}[Browser]{browser} An object with three attributes
%that describes the user agent as sent in the \texttt{User-Agent} HTTP header.  Those attributes are:
%
%\begin{memberdesc}[string]{version} the version of the client browser
%\end{memberdesc}
%\begin{memberdesc}[string]{lang} the language of the client browser
%\end{memberdesc}
%\begin{memberdesc}[string]{name} the name of the client browse (e.g. Mozilla)
%\end{memberdesc}
%\end{memberdesc}
%\begin{memberdesc}[dictionary]{args} A dictionary of CGI variables.
%\end{memberdesc}
%\begin{memberdesc}[string]{uri} The URI of the request.
%\end{memberdesc}
%
%\begin{methoddesc}{redirect}{url}  A function taking a URL that sets the appropriate response
%variables to enact a redirect.  
%\end{methoddesc}
%\begin{methoddesc}{setContentType}{contentType}  Convienience function taking a single argument
%to set the \texttt{Content-Type} header.  Equivalent to:
%\begin{verbatim}
%connObj.responseHeaders['Content-Type'] = arg
%\end{verbatim}
%\end{methoddesc}
%\begin{methoddesc}{write}{s} Function taking a string to be written into the request body.
%\end{methoddesc}
%\begin{methoddesc}{setStatus}{statusCode} Set the HTTP response code to the integer argument.
%\end{methoddesc}
%\begin{methoddesc}{response}{} Returns a string containing the full text of the response.
%\end{methoddesc}
%
%%%%%%%%%%%%%%%%%%%%%%%%%%%%%%%%%%%%%%%%%
%\chapter{The \texttt{templating} Service}
%\index{services!\texttt{templating}}
%The main raison d'\^etre of the \texttt{templating} service is to
%integrate the \texttt{AE} library (see Appendix \ref{aelib}, page
%\pageref{aelib}).  The \texttt{AE} library, in short, is responsible
%for the component rendering, caching and message catalog facilities
%that are generally the reason for using SkunkWeb in the first place.
%
%
%\section{API}
%The API to the \texttt{templating} service is actually quite minimal,
%as most of the interesting bits are in the \texttt{AE} library proper.
%
%\subsection{\texttt{templating.MailServices.sendmail}}
%
%In the \texttt{templating.MailServices} module, there is one function
%of interest:
%
%\begin{funcdesc}{sendmail}{to_addrs, subj, msg, from_addr =
%Configuration.FromAddress} 
%The general-purpose sendmail function, which is called
%    by the STML \texttt{<:sendmail:>} tag, or directly by Python code.
%
%\begin{argdesc}
%\item[to\_addrs] should be a list or tuple of email address
%    strings. 
%\item[subj] must be a string, although it may be empty.
%\item[msg] is a string containing the body of the message; it can
%    be empty. 
%\item[from\_addr] is a single mail address string; it 
%    defaults to the value of the \texttt{FromAddress} configuration variable.
%\end{argdesc}
%This function returns nothing on success, and raises a
%\texttt{MailError} on any mail failure. 
%
%\end{funcdesc}
%
%
%
%\subsection{\texttt{templating.UrlBuilder} Functions}
%
%\begin{funcdesc}{url}{path, queryargs, text = None, kwargs = {},
%noescape = None, url = None}
%
%Returns either the url part of the request (\texttt{http://path}) or
%(if \texttt{text} is not \None) the complete link:
%\verb!<A href="http://path">text</A>!. 
%
%If the \texttt{url} argument is given, it is used in place of
%\texttt{path} and \texttt{query}.
%
%\begin{argdesc}
%\item[path] Path to the target document.
%\item[queryargs] Any HTTP GET style CGI arguments.
%\item[text] Text to appear between the open and close \texttt{<A>} tags.
%\item[kwargs] Any additional arguments to the open \texttt{<A>} tag
%(e.g. \texttt{onClick}).
%\item[noescape] Boolean specifying that the path should not be
%properly escaped.
%\item[url] As an alternate to \texttt{path}, the full absolute URL to the
%target document.
%\end{argdesc}
%\end{funcdesc}
%
%
%\begin{funcdesc}{image}{path, queryargs = None, kwargs={}, noescape = None}
%
%Generate an HTML \texttt{<IMG>} tag.  Arguments are the same as those
%for \texttt{url}.
%\end{funcdesc}
%
%
%\begin{funcdesc}{form}{path, kwargs = {}, noescape = None}
%Generate an HTML \texttt{<FORM>} tag.  Arguments are the same as those
%for \texttt{url}.
%\end{funcdesc}
%
%
%\begin{funcdesc}{redirect}{path = None, url = None, queryargs = {},
%kwargs = {}, noescape = None} 
%Cause a redirect to either the URL specified or to the path specified.
%Arguments are the same as those for \texttt{url}.  
%\end{funcdesc}
%
%\begin{funcdesc}{retain}{dict}
%Returns a string containing HTML \texttt{<INPUT TYPE=HIDDEN>} tags
%that contain the contents of \texttt{dict}.  The dictionary keys are
%used for the \texttt{NAME} arguments and the corresponding values are
%used for the \texttt{VALUE} arguments.
%\end{funcdesc}
%
%
%\section{Component Cache Distribution}
%\NOTE{NEEDS ATTENTION}
%\label{ccdist}
%
%%%%%%%%%%%%%%%%%%%%%%%%%%%%%%%%%%%%%%%%%
%\chapter{The \texttt{pars} Service}
%
%\section{The \PAR\ File Format}
%The \PAR\ file format is quite simple.  What it is is a Python
%marshalled format dictionary where the keys are directory/file names
%and the values are a tuple of the offset into the data area (I'll get
%to that in a second) and the information acquired at packaging time
%from calling \texttt{os.stat()} on the file.
%
%After the marshalled dictionary is the data area.  The data area is
%where the contents of the files are stored (directiories, obviously
%don't have anything here).  So, if you've loaded the dictionary into a
%variable named, say \texttt{archDict}, and the data area loaded into
%\texttt{dataArea} you can obtain the contents by:
%\begin{verbatim}
%offset, stat_info = archDict[filename]
%contents = dataArea[offset:offset + stat_info[stat.ST_SIZE]]
%\end{verbatim}
%
%%%%%%%%%%%%%%%%%%%%%%%%%%%%%%%%%%%%%%%%%
%\chapter{The \texttt{basicauth} Service}
%\index{services!\texttt{basicauth}}
%
%The \texttt{basicauth} service provides a basic authentication
%mechanism for browser based authentication.
%
%\section{API}
%The \texttt{basicauth} service adds a few things to the
%\connection\ object.  Namely,
%\begin{memberdesc}[string]{AUTH_TYPE} The type of authentication,
%should always be \verb!'Basic'!.
%\end{memberdesc}
%\begin{memberdesc}[string]{REMOTE_USER} The name that the browser
%specified when authenticating to the server.  May be \None.
%\end{memberdesc}
%\begin{memberdesc}[string]{REMOTE_PASSWORD} The password that the browser
%specified when authenticating to the server.  May be \None.
%\end{memberdesc}
%
%\section{The \texttt{swpasswd} Utility}
%The \texttt{swpasswd} utility is a simple program to maintain
%password files.  
%
%Its general form is:
%\begin{verbatim}
%swpasswd [-cb] password_file username [password]
%\end{verbatim}
%
%The two options are:
%\begin{argdesc}
%\item[-c] Create a new password file
%\item[-b] Specify the password on the command line as opposed to being
%asked for it interactively.
%\end{argdesc}
%
%
%%%%%%%%%%%%%%%%%%%%%%%%%%%%%%%%%%%%%%%%%
%\chapter{Writing A Service}
%
%uri rewriting likely in web.HaveConnection
%
%cookie auth schemes (probably also in web.HaveConnection)
%
%document handlers
%
%daisy chaining functions \'al\`a basicauth
%
%
%an example easter egg service
%
%a remote swpython example?  if so, warn why \textbf{really} bad
%
%\textbf{Explain the \texttt{SkunkWeb.Configuration} ``module'' and
%it's methods and that the namespace above is in this object}
%
%\verb!_mergeDefaultsKw!
%
%Hooks
%
%The \verb!Server! interface
%
%A request recording service
%
%
\appendix
\input ae.tex

\chapter{Database APIs}
\label{Database APIs}

The \service{mysql}, \service{oracle} and \service{postgresql}
services exist to merely import their pylib counterparts
(\texttt{MySQL}, \texttt{Oracle} and \texttt{PostgreSql}
respectively).  The services themselves provide no api, they just
provide configuration options to setup the connection caches that the
pylib modules hold.  

The APIs to the modules are essentially identical.  There is really
only one function of interest (though some contain other function
which you should \emph{NOT} rely on to exist in future releases), and
that is \texttt{getConnection}.  This function takes a connection name
(as defined in the \texttt{sw.conf} file) and returns the connection
associated with it.  The connections are lazily made, i.e. they are
not created when the SkunkWeb child process starts, but when the first
call of \texttt{getConnection} with the connection name is made in a
process.

Here's an example of typical usage using PostgreSQL:

\begin{verbatim}
import PostgreSql
db = PostgreSql.getConnection('test') # <--- as defined in sw.conf
cur = db.cursor()
cur.execute('select * from testTable')
result=cur.fetchall()
cur.close()
\end{verbatim}

\cleardoublepage

\printindex

\end{document}

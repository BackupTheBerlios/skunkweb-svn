\documentclass{article}
\author{Frank Tegtmeyer, fte@pobox.com}
\title{Sending Mail with SkunkWeb}
\begin{document}
\maketitle

\begin{abstract}
This paper describes the use of the mail services included in SkunkWeb.
Sending mail seems to be easy but there are some technical and conventional
requirements that should be fullfilled. The incredible amount of web services
sending technically illegal mail or creating annoyances for users and mail
administrators is unbelievable.

You can do the world a favor by reading this document, following it's
guidelines and thus avoiding all the trouble with mail services and
people.
\end{abstract}

\newpage
\tableofcontents
\newpage


\section{Social behaviour}

\subsection{The SPAM problem}

As you may know, SPAM is the most visible annoying thing regarding mail.
Of course your intention ist to get and keep loyal site users, so you
should do anything possible to avoid the slightest suspicion that your
messages sent through SkunkWeb are SPAM.

Here are some guidelines to keep your reputation of not being a spammer:

\begin{itemize}
  \item Always use ,,opt in''. Never use an address that comes from an
        untrusted or unverified source. The \textbf{intention} of the
	address owner to receive your mails \textbf{has to be verified}.
	An exception
	are mails that are technically ore procedural required in the
	case that there is already a relationship between the user and
	you.

	A short note here regarding opt-in: Some people, espacially those
	belonging to the ,,direct marketing guild'', use the term ,,double
	opt-in''. That's simply nonsense~- either the user opts in (with
	verification) or not. Don't trust anyone using the term double
	opt-in, most likely it's a spammer (of course this will be denied
	by him).
  \item Never trade addresses - not by providing them to others, not by
	accepting them from others.
  \item Always respect the users wish to not longer receive your mails.
        Provide a way for them to unsubscribe without your manual
	intervention and immediately.
  \item Handle complaints manually, not by an automated answer.
  \item Make your mails identifyable as coming from you. Possibly
        provide a reference on the web where the user may check the
	authenticity of your mail.
  \item Provide a contact address for complaints. Mails coming in at
        that address have to be handled of course.
  \item Handle bounces (non deliverable mails) - you should unsubscribe
        addresses, that generate bounces everytime, after a while.
\end{itemize}

\subsection{Privacy and style}

Even in cases where the mail is not SPAM, people easyly may get annoyed by it.
Always remember that your mails require attention and time of the receiver.
This time and attention is valuable, so don't waste it!

\begin{itemize}
  \item Use a good subject line. It's the key to users attention. It
        should give a summary of what the user has to expect from this mail.
  \item Use ,,inverse writing'' style - start with a summary and then
        give the details for interested users (see Nielsen).
  \item Use clear and concise language. Make your intention for sending
        this mail clear and, if possible, the mails value for the receiver.
	In any case there should be a value for the receiver or you will
	be recognized as an annoying factor.
  \item Provide a privacy statement. Explain what the users email address
        is used for and how it will be handled. Of course you must keep
	yourself bound to this statement. Never trade addresses.
  \item In case of HTML mails don't include web bugs. People don't like
        to be tracked. If you need such data, provide a way to get it from
	the users with their agreement. In general HTML mails should be
	avoided, they hurt security/privacy aware users and those with
	technically limited equipment (for example PDAs). Users that read
	mail offline will not get objects that are provided online
	anyway.
  \item Use an existing address as envelope sender\footnote{envelope senders
        are explained later}.
  \item In case of bounces make the user aware of it in your web
        application. There may be a problem with its mail provider that
	the user should know about.
\end{itemize}

The usability expert Jakob Nielsen wrote about newsletters in his
,,alertbox''. This provides some valuable
insights into peoples behaviour and best practices for sending mail.
You can find the alertbox issue at\\
\texttt{http://www.useit.com/alertbox/20020930.html}.

\subsection{Technical correctness}

The number of technical incorrect implemented mailservices within web
applications is frustrating and it seems it gets worse with every new one.

Typically the ones that run these services don't know about it or worse,
they don't care. It's very frustrating for mail administrators at the
receiving sites to handle such situations.

In many cases the offending
site is completely blocked, thus loosing the possibility to send any
mail to this mailserver again. Possibly the bad site is also reported
to some databases that collect information about such behaviour and
provide more administrators the possibility to block the site too.

It may be very hard to remove such database entries after they are created.
Once the services reputation is damaged, it takes a long time to improve
it again.

While Python and SkunkWeb save you from some errors that are common,
they still require you to put attention to some others.

Here are some errors that SkunkWeb/Python avoid:

\begin{itemize}
  \item bare linefeeds (see http://cr.yp.to/docs/smtplf.html for more
        information)
  \item missing Date: header
  \item missing, not required but very useful, message id
  \item MIME encoding for ASCII (not a problem but not necessary)
  \item 8bit data without proper MIME headers
  \item sending with empty envelope sender
\end{itemize}

Here are some problems that you have to avoid yourself:

\begin{itemize}
  \item sending with invalid envelope sender\\
        SkunkWeb uses the FromAddress if no other envelope sender
	is given. The FromAddress is possibly taken from the default
	value that is set in the sw.conf file or from the MailService module
	itself. The default value is ,,root@localhost'' which is not
	a valid envelope sender. You \textbf{have to} change this
	default value to a reachable address that is under your control.
  \item encoding of multipart messages\\
        currently this is only supported via the ,,raw'' parameter
  \item encoding of character sets besides us-ascii and iso-8859-1
        is not supported (except through the ,,raw'' parameter)\\
        Because of compatibility reasons the email module is not
	used by SkunkWeb. The current limitations stem from the use
	of the rfc822, mimetools and mimify modules.
  \item use of invalid recipient addresses\\
        SkunkWeb does some basic syntax checks for the recipient addresses
	but this is a kind of ,,last fallback''. Always use valid,
	verified addresses.
  \item sending to addresses that do not or no longer exist\\
        Always ensure that you handle bounces, either by hand or, much
	better, automated. Addresses that are not rechable for a longer
	period must be removed from your database. If possible, the
	user should be made aware of that fact inside the web application.
\end{itemize}


\section{Usage patterns}
\subsection{Administrative mails}
\subsection{Newsletters}
\subsection{Feedback}
\subsection{,,Send to a Friend''}

\section{Sanity checks before sending}

\section{The sendmail function}
\subsection{Basic parameters}
\subsection{Extended parameters}
\subsection{Exceptions}

\section{The sendmail tag in STML}
\subsection{Parameters}
\subsection{Formatting}

\section{Self formatted mails (raw parameter)}
\subsection{Preparation with email module}
\subsection{Required headers}
\subsection{Sending attachments}

\section{Handling bounces}
\subsection{Qmail}
\subsection{Postfix}
\subsection{Sendmail}
\subsection{Exim}
\subsection{Manual handling}

\section{Example for a feedback form}
\subsection{STML}
\subsection{Python}

\end{document}

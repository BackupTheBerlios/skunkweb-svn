%fix obeylines things
\documentclass{manual}
\usepackage{makeidx}
\usepackage[nottoc]{tocbibind} % make so bib and ind are in toc
\title{The SkunkWeb STML Reference}
\author{Drew Csillag}
\release{3.0}
\setshortversion{3}
\makeindex

\begin{document}
%function argument list environment
\newcommand{\argdescitem}[1]{\hspace\labelsep
                                \normalfont\ttfamily #1\ }
\newenvironment{argdesc}{\begin{list}{}{
        \renewcommand{\makelabel}{\argdescitem}
}
}{\end{list}}

\newcommand{\doref}[1]{(see section \ref{#1}, page \pageref{#1})}
%begin titlepage stuff
\maketitle
\ 
\vfill 
\noindent
This file documents the SkunkWeb Web Application Framework.

\noindent
Copyright (C) 2001 Andrew Csillag

\noindent
Permission is granted to make and distribute verbatim copies of
this manual provided the copyright notice and this permission notice
are preserved on all copies.

%%end titlepage stuff

\tableofcontents

\chapter{Introduction to STML}
\label{stmlrefintro}
The SkunkWeb Enviroment gives you two ways to do programming for your
web applications: with \emph{code modules} written in the programming
language Python, and with \emph{templates}, which are text documents
with embedded programming code. In SkunkWeb, these templates' embedded
programming code is written in a language called STML, or Skunk
Template Markup Language. STML is like other markup languages such as
HTML or XML, in that it uses ``tags'' to express its instructions.

This guide introduces the
syntax and conventions of STML, and also includes a reference
for all STML ``tags''.

STML documents are text (usually HTML) documents with 
embedded ``programming'' tags. These programming tags are 
written in the special ``tag language'' we call STML 
(Skunk Template Markup Language).

At first glance, an STML document looks just like 
a regular text document. Its filename ends in 
\code{.html} (or \code{.txt} or whatever you want the 
output to be). Once you look at the document, however, 
you will notice the STML tags, which look like HTML 
tags with colons (:) inside them:

\begin{verbatim}
<:tag_name:>
\end{verbatim}


SkunkWeb can recognize these STML tags, and follow the instructions
you give in the tags.

STML tags are more complex and powerful than HTML tags. 
In HTML, a tag has a name (such as \texttt{BR}), 
and optionally some \emph{attributes}, which are pairs 
of strings with a ``name'' and a ``value'':

\begin{verbatim}
<A HREF="index.html">
\end{verbatim}


In the above HTML tag, \texttt{A} is the tag's name, 
and the tag has one attribute. The attribute's name is 
\texttt{HREF}, and its value is the string \texttt{"index.html"} 
(quotes not included). In HTML, attribute values are always 
\emph{strings}, and the quotation marks around a string
are required only if there are spaces or tabs in the string.

STML tags are similar. However, STML uses Python! 
So the values of attributes can be things other than strings. 
Consider this STML tag:

\begin{verbatim}
<:happy_tag foo=hi bar="Hello there" 
     baz=5 dib=`myVar`:>
\end{verbatim}

The name of the STML tag is \texttt{happy_tag}. 
(This tag is not a real STML tag, of course.) 
The tag has four attributes, with keys \texttt{foo}, 
\texttt{bar}, \texttt{baz}, and \texttt{dib}. 
Looks like an HTML tag so far, correct? 
But look at the attribute \emph{values}.

The value of attribute foo is the \texttt{"hi"} 
(quotes not included). It's a string; if you type something 
without any quotes, STML assumes the value is a string.
The value of attribute \texttt{bar} is \texttt{"Hello there"} 
(quotes not included in the value). It's also a string, 
of course. When you quote string values in STML, 
you can use Python's way of quoting strings with single 
or double quotes.

The value of attribute \texttt{baz} is \texttt{5}. 
In HTML, this would be the string \texttt{"5"}, and so it 
is in STML. But STML, which uses Python, is smart enough 
to know about Python's real ``data types'', such as integers 
and floating-point numbers.

Therefore, if you want an attribute value to be 
something other than a string, use STML's best feature: 
the \emph{backticks}. Any value which you enclose in backticks 
\texttt{`like this`} will be taken to be a Python expression. 
Thus when you say \texttt{bar=5}, STML sees the string \texttt{"5"} 
for \texttt{bar}, but when you say \texttt{bar=`5`}, STML sees 
the Python expression \texttt{5}, which it knows is an integer.

The attribute \texttt{dib} is an example of more powerful 
uses of backticked Python expressions: the use of Python variables.
The \texttt{dib} argument has a value \texttt{`myVar`}, which in Python means
the Python object in the variable named \texttt{myVar}. Now you have
a real programming enviroment in your STML document!  You can use certain
STML tags to assign strings and other Python objects to variable names,
and then use those variable names in later STML tags:

\begin{verbatim}
<:set name=myVar expr=`56.7`:>
<:val expr=`myVar`:>
\end{verbatim}


The STML \texttt{<:set:>} (see section \ref{tagset}, page
\pageref{tagset}) tag  
lets you assign the result of a Python expression in the attribute
named \texttt{expr} to a variable name in the attributed named \texttt{name}.
In the example, you assign the floating point number 
\texttt{56.7} to the variable \texttt{myVar}. Then the
\texttt{<:val:>} (see section \ref{tagval}, page
\pageref{tagval})
tag displays
the result of a Python expression in the attribute named \texttt{expr}
as HTML output for the user.

You can do a lot inside the backticks. Any Python expression is legal:
make calls to functions, make instances of Python classes, do arithmetic,
do boolean logic with \texttt{and} and \texttt{or}. 
Python has a built-in function \texttt{int()}, for example, which takes 
an object (like a floating-point number) and turns it into an integer: 

\begin{verbatim}
<:set name=myVar expr=`56.7`:>
<:set name=myOtherVar expr=`int(myVar)`:>
\end{verbatim}

STML sees the Python expression \texttt{int(myVar)} and executes it; 
then it puts the result to the variable \texttt{myOtherVar}.
\texttt{myOtherVar} now contains the integer \texttt{56}.

STML has one other feature that is very different from HTML: 
in most cases, attributes do not have to be specified by name,
because the order in which the tag expects the attributes is always the
same. Thus you can specify just values without names, in the order
expected by the tag. (This reference guide tells you the expected order
of attributes of each and every STML tag.)

For instance, the \texttt{<:set:>} tag in STML expects
an attribute named \texttt{name} followed by and attribute named
\texttt{expr}. Instead of typing 

\begin{verbatim}
<:set name=myVar expr=`56`:>
\end{verbatim}


all the time, you can just type the values for \texttt{name}
and \texttt{expr} in that order:

\begin{verbatim}
<:set myVar `56`:>
\end{verbatim}

If you do use attribute names in STML tags, the order of attributes
does not matter, because STML will see them by name and disregard
the order of the attributes. The following example works, although
you would probably not do it in practice:

\begin{verbatim}
<:set expr=`56` name=myVar:>
\end{verbatim}

A few STML tags do not allow you to drop the attribute names
in certain cases. Their documentation in this manual will always tell
you when you cannot.




\chapter{Setting and Showing Things}
\label{stmlrefstandard}

\section{\texttt{<:val:>}}
\label{tagval}

\begin{verbatim}
<:val expr [fmt="plain"]:>
\end{verbatim}

Causes the Python expression \texttt{expr} to be evaluated, 
and the resulting Python object is converted to a string and 
displayed at that place in your document. 
(All Python objects can be turned into strings.) 

For example, if you would like to display the result of 5 + 2:

\begin{verbatim}
<:val `5 + 2`:>
\end{verbatim}


The \texttt{<:val:>} tag is just like calling the \texttt{str()} 
function in Python on the resulting object, with one exception: 
if the object is the \texttt{None} object, \texttt{<:val:>} 
prints an empty string \texttt{""} instead of the string 
\texttt{"None"}. The following two tags show the value of
the variable named \texttt{myVar} and show nothing, respectively:

\begin{verbatim}
<:val `myVar`:>
<:val `None`:>
\end{verbatim}

\texttt{<:val:>} only accepts Python expressions as its attribute, 
not full Python statements. That means that you cannot assign 
objects to variable names with this tag. Use the 
\texttt{<:set:>} \doref{tagset} tag to assign 
objects to variable names. You can, however, call functions 
and classes, and make boolean expressions with \texttt{and}, 
\texttt{or}, and \texttt{not}. For example, if you want to call
a function \texttt{myFunction}, which returns some HTML to include
in your document:

\begin{verbatim}
<:val `myFunction()`:>
\end{verbatim}

The string printed by \texttt{<:val:>} is just a plain 
string by default; no special formatting takes place. 
Sometimes you want the string to be formatted in a special way: 
you want HTML special characters escaped, or you want all 
characters escaped so that the string can be made part of a URL. 
The \texttt{fmt} argument lets you specify one of many common 
formatting or escaping operations on the string:

\begin{itemize}
\item \texttt{"plain"} or \texttt{"plaintext"}: 
No escaping. This is the default.

\item 
\texttt{"latin"} or \texttt{"latinquote"}: 
Escape only extended characters to ISO Latin1 
entities: \texttt{\'i} become \texttt{\&iacute;}, etc.


\item 
\texttt{"html"} or \texttt{"htmlquote"}: 
Escape only things that would break HTML 
formatting: \texttt{< > \& `` '}.
These characters are escaped to the named HTML entities
for them: \texttt{\&lt;}, etc.


\item 
\texttt{"uri"}, \texttt{"url"}, \texttt{"uriquote"} 
or \texttt{"urlquote"}: 
Escape all URL-unsafe characters to \texttt{\%XX} format.


\item \texttt{"fulluri"} or \texttt{"fullurl"}: 
Escape every character into URL-safe \texttt{\%XX} format.


\item \texttt{"base64"}: base64 encodes the string.

\end{itemize}


Examples of \texttt{fmt}, with the output shown after each example:

\begin{verbatim}
<:set myVar "Click here >>>>":>

<:val `myVar`:>
Click here >>>>

<:val `myVar` fmt="html":>
Click here &gt;&gt;&gt;&gt;

<:val `myVar` fmt="url":>
Click%20here%20%3E%3E%3E%3E

<:val `myVar` fmt="fullurl":>
%43%6C%69%63%6B%20%68%65%72%65%20%3E%3E%3E%3E
\end{verbatim}


The \code{<:val:>} tag is equivalent to the following code executed
in a Python component:

\begin{obeylines} \ttfamily
import DT.DTCommon
print DT.DTCommon.ValFmtRgy[\emph{fmt}](\emph{expr})
\end{obeylines}



\section{\texttt{<:set:>}}
\label{tagset}

\begin{verbatim}
<:set name value:>
\end{verbatim}

Allows you to assign the object in \texttt{value} to the
local variable named in \texttt{name}. Use this tag to do
all of your assignments to variables.

You can also use the
\texttt{<:call:>} \doref{tagcall} tag to
make assignments, by putting a regular Python assignment
statement in backticks: \code{<:call `foo = 567`:>}.
STML still lets you do that. Use the \texttt{<:set:>} tag
to assign objects to simple variable names. Use the 
\texttt{<:call:>} for assignments that are more complex;
see its reference page for details.

These two tags assign the integer 567 to the variable name \texttt{myVar},
and then assign a new object returned by calling \texttt{module.function}
to the same variable named \texttt{myVar}.

\begin{verbatim}
<:set myVar `567`:>
<:set myVar `module.function()`:>
\end{verbatim}




\section{\texttt{<:call:>}}
\label{tagcall}

\begin{verbatim}
<:call expr:>
\end{verbatim}


Allows you to execute the Python expression in 
\texttt{expr}. Any object returned by the expression is ignored,
and \emph{no output is included in your template}. Think of this tag
as being just like the \texttt{<:val:>} tag, except that
\texttt{<:call:>} produces no output.

Examples:

\begin{verbatim}
<:call `function()`:>
<:call `module.function(var1, var2)`:>
\end{verbatim}


You can also use this tag to do complex assignments that the
\texttt{<:set:>} \doref{tagset} tag cannot handle,
such as assignments to items in a list or dictionary, or to attributes
in an object. Here are some examples, with \texttt{<:set:>} used
for the simple assignments to variable names:

\begin{verbatim}
<:set myList `[1,2,3,4]`:>
<:call `myList[0] = function()`:>
<:set myDict `{}`:>
<:call `myDict["foo"] = "Hey there"`:>
<:set myObj `newObject()`:>
<:call `myObj.someAttribute = 0`:>
\end{verbatim}




\section{\texttt{<:spool:>}}
\label{tagspool}

\begin{verbatim}
<:spool name:><:/spool:>
\end{verbatim}

Takes the content generated between the open and close
\code{<:spool:>} tags and  
assigns it to the local variable named \texttt{name}. 
The name argument cannot be a Python expression, only
a string with the variable name.

This tag is deceptively powerful: you can execute
any STML you wish, collect all the HTML output into a single
string, and do anything to that string after ending the
\texttt{<:spool:>}. In this example, the variable
\texttt{spoolVar} would end up containing the string
\verb!"\n<A HREF=index.html>Click here</A>\n"!:

\begin{verbatim}
<:set myUrl "index.html":>
<:spool spoolVar:>
<:url path=`myUrl` text="Click here":>
<:/spool:>
\end{verbatim}


Spooling is equivalent to the following Python code in a Python component:
\begin{obeylines}
\ttfamily
import sys, cStringIO
\emph{tempname} = sys.stdout
sys.stdout = cStringIO.StringIO()
\emph{\# stuff between <:spool:> tags }
\emph{name} = sys.stdout.getvalue()
sys.stdout = \emph{tempname}
\end{obeylines}



\section{\texttt{<:args:>}}
\label{tagargs}

\begin{verbatim}
<:args [argname] [argname1=expr] 
     [argname2=expr...]:>
\end{verbatim}


This tag makes short and quick work of handling arguments
submitted by the user of your website through a GET or POST
request. It allows you to copy CGI arguments into local
variables of the same name, convert the CGI arguments (which
are always strings) into Python data types, and assign default 
values to those variables if the conversion fails or if
the caller failed to pass them in.

Each argument of this tag specifies a name, and optionally
a Python expression value. The name of each argument indicates
the name of the user argument \emph{and also} the name of 
the local variable to hold the value. If you just specify this
name, it will copy the user argument of that name
into a local variable of that name. If the user passed no argument
by that name, it assigns the \texttt{None} object
to the variable.

If you do specify a Python expression as a value, it
will examine the expression, and take the following actions:

\begin{enumerate}

\item If the expression results in a function or other "callable"
object, it is a "conversion" function. It will take
the user argument of that name, call the function with it, and
assign the returned value to the local variable of that name.
If the function raises an exception, or if the user passed no
argument of that name, it assigns \texttt{None} to
the local variable:

\begin{verbatim}
<:args this=`int`:>
\end{verbatim}

In the above STML, the user argument named \texttt{this}
will be converted into an integer with the built-in Python
function \texttt{int}. Then the result will be assigned to
a local variable \texttt{this}. If anything goes wrong, 
or if the user passed no \texttt{this} argument, the local
variable \texttt{this} will be \texttt{None}.


\item If the expression results in a regular, or "non-callable",
object, the it considers this object the default value.  If the user
passed no argument of that name, the templating engine assigns this
object to the local variable of that name.

\item If the expression is a list or tuple of two items, the
templating engine considers the first item to be the conversion
function, and the second item to be the default value:

\begin{verbatim}
<:args this=`(int, 5)`:>
\end{verbatim}

In this example, SkunkWeb looks for the user argument named \texttt{this}.
If it's there, SkunkWeb calls the function \texttt{int} with the value,
and then assigns the rsult to the local variable \texttt{this}. If 
the user passed no arugment named \texttt{this}, or if the call to
\texttt{int} fails in some way, SkunkWeb assigns the integer 5 
to \texttt{this}.


\end{enumerate}


This tag only needs to be used once, near the beginning of the STML
document which serves the Web request. However, you may use it as many
times as you like. Each time this tag executes, it looks in the
SkunkWeb dictionary object \texttt{CONNECTION.args}, extracts values
from it, and performs the necessary conversion and variable
assignments.  You cannot use this tag in an STML component, however;
STML components do not have access to any part of the
\texttt{CONNECTION} object (unless of course, the \texttt{CONNECTION}
object was passed to it explicitly).

Also note that you are not limited to the built-in Python functions
\texttt{int}, \texttt{float}, \texttt{long}, etc. for your conversion
functions. Any function, or any callable object for that matter, can
be used. Here's a large example of an \texttt{<:args:>} tag
that does a lot of work:

\begin{verbatim}
<:import Date:>
<:args foo
       bar=`0`
       baz=`(int, 5)`
       dib=`(float, 0.0)`
       gad=`(Date.LocalDate, Date.LocalDate())`
:>
\end{verbatim}

As of SkunkWeb 3.2, the \texttt{CONNECTION} object has a method,
\texttt{extract_args}, which can be used for the same purpose whenever
the texttt{CONNECTION} object is present (e.g., in python top-level
documents).  \texttt{CONNECTION.extract_args} returns a dictionary of
extracted values; to achieve the same result as above in pure Python,
you would do the following:

\begin{verbatim}
import Date
d=CONNECTION.extract_args(foo, 
                          bar=0, 
                          baz=(int, 5),
                          dib=(float, 0.0),
                          gad=(Date.LocalDate, Date.LocalDate()))
locals().update(d)
\end{verbatim}

\section{\texttt{<:compargs:>}}
\label{tagcompargs}

\begin{verbatim}
<:compargs [argname] [argname1=expr] 
     [argname2=expr...] [**kwargname]:>
\end{verbatim}


%\textbf{Note:} This tag is for advanced use only, for developers
%who have worked extensively with STML components.

This tag allows you to create a component signature, i.e. 
specify which arguments the component expects and provide some
default values for them. When someone calls your component, SkunkWeb
will make sure that the arguments it is called with match the 
signature. The arguments specified by the \code{<:compargs:>} 
tag will appear in the local namespace. The tag is optional, if you do 
not specify it, all of the arguments passed to the component will
appear in the local namespace. You can optionally specify default 
values for the arguments. If an argument is not passed by the 
caller and it has a default value, the default value will be used for the
local variable

If your component accepts variable number of arguments, you can
use the \emph{**kwargs} notation to specify a container for any 
additional arguments. Any arguments not explicitely listed in the 
\code{<:compargs:>} tag will be stored in the 
\emph{kwargs} variable. 
If you do not specify such a variable, and the component is called
with arguments not listed in the \code{<:compargs:>} tag, 
an exception will be raised.

Suppose your component has the following signature:

\begin{verbatim}
<:compargs x y=`10` z='hello' **kwargs :>
\end{verbatim}


Suppose someone calls the component with following call: 

\begin{verbatim}
<:component comp x='hi' y=`11` extra=`12`:>
\end{verbatim}


In this case the following variables will appear in local namespace:
\begin{argdesc}
\item[x] The value is string 'hi'
\item[y] The value is integer 11
\item[z] The value is string 'hello', taken from default 
\item[kwargs] The value is a dictionary: \code{\{ 'extra' : 12 \}}
\end{argdesc}

 

\section{\texttt{<:default:>}}
\label{tagdefault}

\begin{verbatim}
<:default name [value]:>
\end{verbatim}

If in the current STML document, there exists no local variable called
\texttt{name} (or it's value is \texttt{None}), create it and assign
it the object in \texttt{value}. The \texttt{value} argument defaults
to the empty string.

This tag is often used in STML ``components'', which are STML
documents that can be embedded in other STML documents. Components
are called with ``argument attributes'', which end up as local
variables in the component document. If the calling STML document
does not pass a variable you need, it won't be in your component
document, and when you try to use it, you will get a Python
\texttt{NameError}, meaning that there is no variable under that
name. This tag makes sure that the variable name exists and that
it has a value.


\section{\texttt{<:date:>}}
\label{tagdate}

\begin{verbatim}
<:date [fmt] [date] 
     [lang] [to_zone] [from_zone]:>
\end{verbatim}


Displays a string representation of a date/time object. This is 
an STML tag equivalent of the \texttt{DateString} function
in the Skunk pylibs \texttt{Date} module.

Note that all attribute arguments to this tag are optional.

\texttt{\textbf{fmt}} must be a string, and specifies the format in 
which to display the date/time. If not specified, \texttt{fmt} defaults to
the \texttt{Date} module's default format, which is the ISO format
\texttt{yyyy-mm-dd hh:mi:ss}.

    The \texttt{fmt} string may contain any combination of the 
    following format elements. Anything not recognized 
    as a format is included in the output as static text:

\begin{argdesc}
\item[yyyy] four-digit year
\item[yy] two-digit year
\item[mm] two-digit month
\item[dd] two-digit day
\item[hh24] 24-hour hour
\item[hh12] 12-hour hour
\item[hh] equivalent to \texttt{hh24}
\item[mi] or min minutes
\item[SS] seconds
\end{argdesc}

    For all of the above, leading zeroes are included 
    as necessary: ``01'' instead of 1, etc. However, 
    if the format element is in UPPERCASE, leading zeroes 
    are not included: ``MM'' $\rightarrow$ ``1'', ``mm'' $\rightarrow$ ``01''.

\begin{argdesc}
\item[mon] abbreviated name of month
\item[month] full name of month
\item[dy] abbreviated weekday name
\item[day] full weekday name
\item[am or pm] whether time is AM or PM
\end{argdesc}


    For all of the above, if the element is in UPPERCASE, 
    the result will be in ALL CAPS: ``MONTH'' $\rightarrow$ ``JANUARY''. 
    If the element is Initcap, the result will be initcapped: 
    ``Month'' $\rightarrow$ ``January''. Any other combination of case will 
    be considered lowercase: ``month'' or ``moNth'' $\rightarrow$ ``january''.
    
\begin{argdesc}
\item[date] is a Python expression resulting in a
date/time (strings are \emph{not} allowed). 
If not specified, it defaults to the current local date/time of the machine.

\item[lang] must be a string specifying the language in which
to format the date/time, using the Skunk convention of \texttt{'eng'
'esp' 'por'} for English, Spanish, and Portuguese, respectively.
If not given, it defaults to \texttt{'eng'}.

\item[to_zone] is a timezone argument, meaning, ``Please convert
the \texttt{date} argument to this timezone before displaying.'' It defaults
to the SkunkWeb server machine's local timezone. See the \texttt{Date} module
for a full discussion of timezone support. (\textbf{Note:} This argument
formerly had the name \texttt{timezone}. STML still understands the old argument
name. If you specify both \texttt{to_zone} and a \texttt{timezone} argument,
STML will use \texttt{to_zone} and ignore \texttt{timezone}.)

\item[from_zone] is a timezone argument, meaning, ``Please convert
the \texttt{date} argument from this timezone into the SkunkWeb server
machine's local timezone before doing anything else.'' See the
\texttt{Date} module for a full explanation. \textbf{Note:} This argument
used to be named \texttt{srctimezone}. STML still understands the old
argument name. If you specify both \texttt{from_zone} and a
\texttt{srctimezone} argument, \texttt{srctimezone} will be ignored.
\end{argdesc}

Here are some examples, if the current date is 2000-08-01 20:34:56
(a Tuesday), and your server is running on Eastern Daylight Time, which
is four hours behind UTC (Greenwich Time). Each tag is followed by
the output which would be displayed:

\begin{verbatim}
<:date:> 
2000-08-01 20:34:56

<:date "dd mon yy":>
01 aug 00

<:date "hh12:mi am":>  
08:34 pm

<:date "HH12:mi am":>  
8:34 pm

<:date "hh12:mi am" to_zone="UTC":>  
12:34 am

<:date "Month DD, yyyy, hh12:mi am" to_zone="UTC":>  
August 2, 2000, 12:34 am

<:date "DD month yyyy, hhhmi" lang="esp" to_zone="UTC":>  
2 agosto 2000 00h34
\end{verbatim}




To do the equivalent in Python components, if unspecified arguments are \texttt{None}, is:
\begin{verbatim}
import Date
temp_date = Date.Convert ( date, to_zone or 'LOCAL', from_zone or 'LOCAL' )
print Date.DateString ( temp_date, fmt, lang )
\end{verbatim}






\section{\texttt{<:del:>}}
\label{tagdel}

\begin{verbatim}
<:del name:>
\end{verbatim}

This tag deletes the local variable named by \texttt{name}
from your STML document. It's identical to the Python statement
\code{del \emph{name}}. You will use this tag very rarely,
but it's there if you need it.

In this example, the variable named \texttt{myVar} is set to
the integer \texttt{5}, then deleted from the variable namespace.
The third tag, which tries to access the variable name, results in
a \texttt{NameError}, meaning that nothing exists under that name:

\begin{verbatim}
<:set myVar `5`:>
<:del myVar:>
<:val `myVar`:>
\end{verbatim}



\chapter{Flow Control and Error Handling}
\label{stmlrefcontrol}


\section{\texttt{<:if:>}, \texttt{<:elif:>}, \texttt{<:else:>}}
\label{tagif}

\begin{verbatim}<:if expr:>
   [ <:elif expr:> ]
   [ <:else:> ]
<:/if:>
\end{verbatim}

If the Python expression \texttt{expr} evaluates to true, 
render until an \texttt{<:/if:>}, \texttt{<:else:>} 
or \texttt{<:elif:>} tag is encountered.  The
\texttt{expr}arguments in  
\texttt{<:if:>} and \texttt{<:elif:>} tags \emph{must} be 
enclosed in backticks, of course.

If it finds an \texttt{<:elif:>} tag and nothing has yet 
been rendered within this \texttt{<:if:>} block and 
\texttt{expr} evaluates to true, renders until it runs into 
another \texttt{<:elif:>}, \texttt{<:else:>} or 
\texttt{<:/if:>} tag. If an \texttt{<:else:>} is 
encountered and nothing has yet been rendered in this 
\texttt{<:if:>} block, renders until it hits the 
\texttt{<:/if:>}, \texttt{<:elif:>} or another \texttt{<:else:>} 
(if you put an \texttt{<:else:>} before an \texttt{<:elif:>} 
or another \texttt{<:else:>}, which while syntactically valid, 
is erroneous since the contents following the 
\texttt{<:elif:>} or \texttt{<:else:>} will never be rendered).

This is just a fancy way of saying that these tags work just
like \texttt{if}, \texttt{elif}, and \texttt{else} in Python.

Using \texttt{<:elif:>} outside of an \texttt{<:if:>} block will raise
an error. Using \texttt{<:else:>} outside of an \texttt{<:if:>},
\texttt{<:try:>} or \texttt{<:for:>} block will also raise an error.

In the following example, the output will be \texttt{Shaft is the Man.}

\begin{verbatim}
<:set myVar `1`:>
<:if `myVar > 0`:>
    Shaft is
    <:if `myVar == 6`:>
    the black private dick.
    <:elif `myVar == 4`:>
    a sex machine with all the chicks.
    <:else:>
    the Man.
    <:/if:>
<:else:>
    Why is myVar not greater than 0?
<:/if:>
\end{verbatim}


\section{\texttt{<:for:>}}
\label{tagfor}

\begin{verbatim}<:for expr [name=sequence_item]> 
    [ <:break:> ]
    [ <:continue:> ]
    [ <:else:> ] 
<:/for:>
\end{verbatim}

The \texttt{<:for:><:/for:>} tag implements a ``loop'': 
it expects the expression in \texttt{expr} to return a 
sequence of items (a tuple or list, for example). 
For each item in the sequence, \texttt{<:for:>} 
assigns the item to the variable name given 
in the name argument, and does whatever is 
inside the \texttt{<:for:>} tag. The name parameter 
is optional; if not specified, \texttt{<:for:>} 
will assign each item to a local variable 
called \texttt{sequence_item}.

If the sequence in \texttt{expr} is empty or false, 
\texttt{<:for:>} will do nothing, unless you put 
an \texttt{<:else:>} tag inside the \texttt{<:for:>} block. 
The \texttt{<:for:>} tag will execute everything after 
an \texttt{<:else:>} tag if the sequence in \code{expr}
is empty.

Examples:

\begin{verbatim}
<:for `(1,2,3)`:>
current loop value: <:val `sequence_item`:><BR>
<:/for:>

<:for `[1,2,3]` foo:> 
current loop value: <:val `foo`:><BR> 
<:/for:>

<:for `[]` foo:>
This wouldn't show
<:else:>
Loop is empty
<:/for:>
\end{verbatim}

There are two special tags which allow you to force either
an exit from the loop or an immediate return to the top of the loop:
\texttt{<:break:>} and \texttt{<:continue:>}. 
The \texttt{<:break:>} tag will
cause an immediate exit from the \texttt{<:for:>} tag.
The \texttt{<:continue:>} tag will cause an immediate return
to the top of the \texttt{<:for:>} tag, where it will
assign the next item in the sequence to the variable, or
if there are no items left in the sequence, it will exit
the \texttt{<:for:>} tag. For example, this STML block
will print the numbers \texttt{1 2 3}:

\begin{verbatim}
<:set myList `[1,2,3,4,3,2,1]`:>
<:for `myList` item:>
   <:if `item > 3`:>
       <:break:>
   <:/if:>
   <:val `item`:>
<:/for:>
\end{verbatim}

And this STML block will print the numbers 
\texttt{1 2 3 3 2 1}:

\begin{verbatim}
<:set myList `[1,2,3,4,3,2,1]`:>
<:for `myList` item:>
   <:if `item > 3`:>
       <:continue:>
   <:/if:>
   <:val `item`:>
<:/for:>
\end{verbatim}



\section{\texttt{<:while:>}}
\label{tagwhile}

\begin{verbatim}<:while expr:>
   [ <:break:> ]
   [ <:continue:> ]
<:/while:>
\end{verbatim}

This tag evaluates the expression \texttt{expr}. If the expression
is true, it executes all of the things inside the tag. Then it
repeats this process until the expression is false. It is functionally
identical to the Python \texttt{while} statement.

This tag is vulnerable to ``infinite loops'', unless you do something
inside the tag to change the value of the expression, the tag will
execute indefinitely, until the SkunkWeb server calls a ``timeout'' and
raises an error. For example, this STML will cause an infinite loop:

\begin{verbatim}
<:set myVar `1`:>
<:while `myVar`:>
   hello, i'm looping
<:/while:>
\end{verbatim}

There are two special tags which work the same way they do inside
a \texttt{<:for:>} tag: \texttt{<:break:>} and 
\texttt{<:continue:>}. The \texttt{<:break:>} tag will
cause an immediate exit from the \texttt{<:while:>} tag.
The \texttt{<:continue:>} tag will cause an immediate return
to the top of the \texttt{<:while:>} tag, where it will
evaluate the expression again and continue executing if the expression
is true. For example, either of these STML blocks will print 
``looping'' 100 times:

\begin{verbatim}
<:set myVar `1`:>
<:while `myVar <= 100`:>
   looping
   <:set myVar `myVar + 1`:>
<:/while:>
\end{verbatim}

\begin{verbatim}
<:set myVar `1`:>
<:while `1`:>
   looping
   <:if `myVar > 100`:>
       <:break:>
   <:set myVar `myVar + 1`:>
   <:/if:>
<:/while:>
\end{verbatim}


\section{\texttt{<:break:>}}
\label{tagbreak}

\begin{verbatim}<:break:>\end{verbatim}

When inside a ``looping tag'', tells the templating engine to exit
the loop immediately. Not allowed outside of a looping tag.
See the \texttt{<:for:>} \doref{tagfor} tag
and the \texttt{<:while:>} \doref{tagwhile} tag
for examples.


\section{\texttt{<:continue:>}}
\label{tagcontinue}

\begin{verbatim}<:continue:>\end{verbatim}

When inside a ``looping tag'', tells the templating engine to return
immediately to the top of the loop and continue executing.
Not allowed outside of a looping tag.
See the \texttt{<:for:>} \doref{tagfor} tag
and the \texttt{<:while:>} \doref{tagwhile} tag
for examples.


\section{\texttt{<:try:>}, \texttt{<:except:>}, \texttt{<:else:>}, \texttt{<:finally:>}}
\label{tagtry}

\ 

\begin{verbatim}
<:try:>
    <:except:>
    [ <:else:> ]
<:/try:>
\end{verbatim}

or 

\begin{verbatim}
<:try:>
    <:except [exc]:>
    [ <:except [exc]:> ]
    [ <:except:> ]
    [ <:else:> ]
<:/try:>
\end{verbatim}

or 

\begin{verbatim}
<:try:>
    <:finally:>
<:/try:>
\end{verbatim}


This tag executes everything inside of it up until it first sees an
\texttt{<:except:>} or \texttt{<:finally:>} tag. If an exception
(error) occurs during that execution, this tag will then take one of the following
actions:

\begin{enumerate}
\item If there is an \texttt{<:except:>} tag which specifies the
exception in its \texttt{exc} argument, it will execute the stuff occurring
after that \texttt{<:except:>} tag until it meets another 
\texttt{<:except:>} tag, an \texttt{<:else:>} tag, or
the end of the \texttt{<:try:>} tag.

\item Otherwise, if there is an \texttt{<:except:>} tag with
nothing specified in \texttt{exc}, it executes the stuff after that
\texttt{<:except:>} tag.

\item Otherwise, there is nothing to catch the exception, so it ``propagates''
upward to another containing \texttt{<:try:>} block. If there
is no containing \texttt{<:try:>} tag, SkunkWeb generates
a page error.

\end{enumerate}


If you have an \texttt{<:else:>} tag after all of your 
\texttt{<:except:>} tags, the stuff after \texttt{<:else:>}
will be executed if no exception has occurred.

If you wish to specify particular exceptions in an
\texttt{<:except:>} tag, use a backticked expression
with the exception object. If you want to specify more than
one exception for an \texttt{<:except:>} tag, specify them
as a tuple expression. 

Example:

\begin{verbatim}
<:try:>
    <:call `something_which_will_blow_up()`:>
<:except `KeyError`:>
    We got a KeyError. 
<:except `(TypeError, ValueError)`:>
    We got a TypeError or ValueError.
<:except:>
    We got some other error. 
    But I know it wasn't a KeyError, TypeError, or ValueError!
<:else:>
    Hey, no error!!!!
<:/try:>
\end{verbatim}

All of this behavior is identical to the way that \texttt{try:},
\texttt{except:} and \texttt{else:} work in regular Python. 
Read the Python documentation about exceptions for more info.

Note the last way to use the \texttt{<:try:>} tag: with
a \texttt{<:finally:>} tag. The tags behave the same
way as Python's \texttt{try:} and \texttt{finally:} statements.


\section{\texttt{<:raise:>}}
\label{tagraise}

\begin{verbatim}<:raise [exc]:>\end{verbatim}

Raises the exception \texttt{exc}, or if \texttt{exc} isn't
specified, the current exception if
one was previously caught. If you specify \texttt{exc} without backticks,
the tag will raise the exception as the string you indicated:

\begin{verbatim}
<:raise BadBadError:>
\end{verbatim}

will raise the \emph{string} \texttt{"BadBadError"} as the exception.
If you want to raise a ``real exception'', such as one of Python's built-in
exceptions or an exception class from a module, you must put it in backticks.
For example, if you want to reject the value of some variable passed to you
with the Python built-in exception \texttt{ValueError}:

\begin{verbatim}
<:raise `ValueError`:>
\end{verbatim}

Please note that if you do not specify \texttt{exc}, and an exception
has not previously occurred, SkunkWeb will raise a \texttt{TypeError}
exception with the message ``exceptions must be strings, classes, or
instances''.

This tag behaves identically to the \texttt{raise} statement in Python.




\section{\texttt{<:halt:>}}
\label{taghalt}
\begin{verbatim}<:halt:>\end{verbatim}

Causes execution of the current STML document or component to stop
entirely. If the templating engine is executing a component or
include, it stops executing that component or include and returns
execution to the template which did the include or component call.  If
there is no calling template (you're at the ``top level'' template),
then the templating engine stops executing any STML, and then returns
the response to the user as it currently stands.


\chapter{Including or Calling Other Documents}
\label{stmlrefexecute}

The biggest benefit of STML documents is that you can 
reuse them inside other STML documents. We call these
``embeddable'' STML documents components. They have a different
file extension: while your regular STML documents end with
the extension of the kind of output they generate (like
\texttt{.html}), all STML component documents end with the file
extension
\texttt{.inc} for include components, \texttt{.comp} for 
regular components, (or \texttt{.dcmp} for data components; more about
those later).

If you have a block of HTML that must be included on 
every page of your website, make it an STML component, 
and then your other STML templates can ``embed'' the template 
in their own output. This lets you use an efficient, modular 
design when building your website.

There are two very different ways to ``embed'' an STML document 
in another one: by ``including'' the embedded template's STML code, 
or by ``calling'' it as an STML component. Let's examine the 
differences between ``include'' and ``component'' in the next sections.


\section{\texttt{<:include:>}}
\label{taginclude}

\begin{verbatim}<:include name:>\end{verbatim}

The STML tag \texttt{<:include:>} executes an
STML component document named \texttt{name}
in the namespace of the currently executing component.

The format of the string for \texttt{name} is a path, such as
\texttt{/foo/bar.inc} or \texttt{foo.inc}. If the path is
\emph{absolute}, starting with a slash, the templating engine will
look for the component at that path underneath your
\texttt{documentRoot} of STML documents. If the path is
\emph{relative}, not starting with a slash, the templating engine
looks for the component inside the current document's directory. You
can also use `\texttt{..}' to move up a directory, just as you do on a
Unix or DOS command line (but you cannot escape the
\texttt{documentRoot}).

For example, you have these three files in your 
documentRoot directory:

\begin{verbatim}/header.inc
/myDir/index.html
/myDir/foo.inc
\end{verbatim}

Inside \texttt{/myDir/index.html}, you want to include both
component files. You can use these tags, the first using an absolute
path and the second using a relative path:

\begin{verbatim}<:include /header.inc:>
<:include foo.inc:>
\end{verbatim}

Unlike the \texttt{<:component:>} \doref{tagcomponent}
tag, there is no caching, nor can you pass any attributes 
in the tag. Any variables that the included template creates, 
deletes, or changes take effect in the including template! 
For example, you have a template \texttt{/index.html} with 
the following code:

\begin{verbatim}<:set myVar `5`:>
<:include foo.inc:>
<:val `myVar`:><BR>
<:val `myOtherVar`:>
\end{verbatim}

and the STML component document \texttt{foo.inc} 
contains this STML code:

\begin{verbatim}<:set myVar `66`:>
<:set myOtherVar "hey man":>
Hey I'm done!
\end{verbatim}

then the output of \texttt{/index.html} is:

\begin{verbatim}Hey I'm done!
66
hey man
\end{verbatim}

A way to do includes from Python code is as follows:
\begin{verbatim}
import AE.Component
print AE.Component.callComponent(name, argdict, 
        compType=AE.Component.DT_INCLUDE)
\end{verbatim}

\strong{Re file extensions:} The \texttt{.inc} and \texttt{.pyinc} file
extensions were added in SkunkWeb 3.2; previous versions used the
\texttt{.comp} and \texttt{.pycomp} file extensions for includes, and
these will still work.  However, since it is rather important to
distinguish the two types of components, using the new include
extensions is recommended.


\strong{NOTE:} The main thing to keep in mind here is that if you do
the include from a Python component, things work as you would expect,
but if you call it from Python code that isn't a component (or an
include), the included component runs in the namespace of the
currently running component, not the namespace of the Python module
you call it from.


\subsection{Python component files (\texttt{.pyinc} and \texttt{.pycomp})}

SkunkWeb also allows you to write includable components in pure Python
code. This is sometimes more convenient when you wish to do a lot of
programming and generate no (or little) HTML output for the
user. These files are simple Python scripts and end in the file
extensions \texttt{.pyinc}  or  \texttt{.pycomp}.  Use them in the 
\texttt{<:include:>} or \texttt{<:component:>} \doref{tagcomponent}
tags, respectively,  just as you would include \texttt{.inc} or regular
\texttt{.comp} component files.  See Python Components,
section \ref{pythoncode}, page \pageref{pythoncode})




\section{\texttt{<:component:>}}
\label{tagcomponent}

\texttt{<:component name [\emph{named_args}] [cache=no|yes|defer|force|old] [__args__]:>}

This tag ``calls'' an STML component much as you would call a function
in Python, and displays any resulting output of the component. It is
\emph{not} like ``including'' the STML component in your document. A
component called with this tag does not get access to your document's
local variables; you must pass any variables explicitly as named
attributes in the \texttt{<:component:>} tag.

Since the component being called does not receive access to the
calling document's local variables, it also does not have any access
to the \texttt{CONNECTION} object, as included STML documents do. The
only variables a called STML component has are the ones passed as
attributes of the \texttt{<:component:>} tag, and any variables which
the component creates for itself.

\begin{verbatim}<:component foo.comp this=`56.7` that=`[5,6,7]`:>
\end{verbatim}

executes the component \texttt{foo.comp} in the current directory,
passing it the variables \texttt{this} and \texttt{that}.  When
\texttt{foo.comp} executes, it sees \texttt{this} and \texttt{that} as
local variables, with the integer 56.7 and the list \texttt{[5,6,7]}
assigned to them.

The \texttt{__args__} argument is a way to be able to pass a
dictionary of arguments to a component programatically.  For example,
\begin{verbatim}
<:component foo.comp __args__=`{'this':56.7, 'that': [5,6,7]}`:>
\end{verbatim}
is exactly equivalent to the previous example. 

And as usual for STML tags, the \texttt{__args__} could be any
expression which results in either a dictionary or \texttt{None}.

Arguments specified via \texttt{__args__} are overridden by values
explicitly specified in the tag.  For example,
\begin{verbatim}
<:component foo.comp a="3" __args__=`{'a':"5", 'b': 5}`:>
\end{verbatim}
The \texttt{foo.comp} would get \texttt{"3"} as the value of the
variable \texttt{a}.

One thing of note, specifying \texttt{'cache'} as keys in the
dictionary argument to \texttt{__args__} does \textbf{NOT} affect
either the deferral or caching semantics of the component, they get
passed through as arguments to the component.

A way to call components from Python code is (see \ref{pythoncode},
page \pageref{pythoncode} for details):
\begin{verbatim}
import AE.Component
print AE.Component.callComponent(name, argdict, 
	cache = whatever)
\end{verbatim}
The \texttt{whatever} values are from the \texttt{AE.Component} module constants of the following:
\label{whatevervals}
\begin{argdesc}
\item[NO] do use the cache
\item[YES] analogous to \texttt{cache=yes} in the \texttt{<:component:>} tag
\item[DEFER] analogous to \texttt{cache=defer} in the
\texttt{<:component:>} tag
\item[FORCE] analogous to \texttt{cache=force} in the
\texttt{<:component:>} tag
\item[OLD] analogous to \texttt{cache=old} in the
\texttt{<:component:>} tag
\end{argdesc}

\subsection{Python component files (\texttt{.pycomp})}

SkunkWeb also allows you to write components in pure Python
code. This is sometimes more convenient when you wish to do a lot
of programming and generate no HTML output for the user. These files
are simple Python scripts and end in the file extension \texttt{.pycomp}.
Use them in the \texttt{<:component:>} or
\texttt{<:include:>} \doref{taginclude}
tags just as you would regular \texttt{.comp} component files.


\subsection{Making components ``cacheable''}
\label{tagcomponentcaching}

There is a special attribute for the \texttt{<:component:>} 
tag, called \texttt{cache}. If you set cache to the string ``yes''
or ``true'', the called component will have its output cached 
to the SkunkWeb server's disk. Setting it to the string ``no'' or
``false'', or not setting it at all, turns caching off.

The next time you call the same component with the same attribute
names and values (that is, the same arguments), the SkunkWeb server
won't bother to execute the component. Instead, it will just read the
cached output from disk. It's a great speed increase for your
application.

If you call the component with different attributes names and values,
the SkunkWeb server creates a \emph{different} cached output file for
those new attributes. If you say:

\begin{verbatim}<:component foo.comp cache=yes arg="hi":>
\end{verbatim}

and then later

\begin{verbatim}<:component foo.comp cache=yes arg="blah" arg2=`5`:>
\end{verbatim}

then SkunkWeb will keep two different cached output files 
for \texttt{foo.comp}: one file for when it
is called with \texttt{arg="hi"}, and one file for when 
it's called with \texttt{arg="blah"} and \texttt{arg2=`5`}. Get it?

As to setting when a cached component will expire, \doref{tagcache}.

\subsection{``Deferred'' component rendering}

The \texttt{<:component:>} and \texttt{<:datacomp:>} also accept as an
argument value to the \texttt{cache} argument the value
\texttt{"defer"}.  If this has been specified, then SkunkWeb will
handle the caching of the component in a graceful, and often more
efficient, way.

When deferral is activated, SkunkWeb will look in the cache to see if
there is output cached for the component under the set of arguments
you passed. If it finds cached output, SkunkWeb uses it and does not
execute the component, just as if it weren't specified.  If SkunkWeb
does not find any cached output, it executes the component and caches
the result, just as if deferral was not requested. However, if
SkunkWeb finds cached output, but it is \emph{expired}, SkunkWeb will
use the cached output anyway, and ``defer'' the execution of the
component until \emph{after} all of the output has been sent back to
the client in the response.  Then, after sending all of the output
back to the user, SkunkWeb executes the component and caches its
output.

This gives you a great performance gain: if any component's cached
output expires, no individual user will have to wait for the component
to be executed before receiving her response. The component's cache
will be updated ``offline'', behind the scenes.

You can make all of your component calls automatically ``deferred'' by
setting the configuration variable
\texttt{DeferByDefault} to 1.

\subsection{``Forced'' component rendering}
Like deferred component rendering, there are times when you want to
render (ignore any available cached version) and cache a component
regardless of whether or not it has expired.  In which case, specify
the value \texttt{"force"} to the \texttt{cache} attribute in the tag.

\subsection{``Old'' Component Rendering}
Almost the complete opposite of forced component rendering.  In this
case, you're effectively saying, ``gimme whatever is in the cache, I
don't care how old it is.''  The only time when component rendering
will actually occur when \texttt{old} is specified is if there is
\strong{NO} cached version at all.

\subsection{Restrictions on arguments to cacheable components}
\label{tagcomponentrestrictions}

If you want to call a component with \texttt{cache=yes} (or
\texttt{cache=defer} or \texttt{cache=force}), there is one
restriction you must obey:

\begin{itemize}
\item \emph{The attributes passed to the cached component must be, 
as Python says, ``hashable''.}
\end{itemize}

The reason for this restriction is that the SkunkWeb server 
needs to take all of the attributes you pass and ``flatten''
them into a short, unique value. The SkunkWeb server uses 
this value to look up the cached output it has stored in 
its disk cache.

So, what does ``hashable'' mean? Well, just consider it 
to mean ``able to be flattened'' until you learn more Python. 
Here's a rule of thumb:

\begin{itemize}
\item Numbers, strings, dates, and the \texttt{None} object are hashable
\item Tuples (read-only sequences) are hashable if all of its items are hashable
\item Lists (changeable sequences) are not hashable, but SkunkWeb makes
an exception and lets you use them as arguments to cacheable 
components, as long as the list's items are either hashable or
are lists or dictionaries. However, \emph{you must be very careful} that the called
component does not try to change the list in any way; if it does,
you will find that the list changes only the first time the component
is called, becuase on subequent calls, the output is cached and the 
component will not be executed.

\item Dictionaries are not hashable, but SkunkWeb makes an exception for them
just as it does for lists, with the same cautions and restrictions.

\item Any combination of acceptable objects is acceptable: 
a tuple of lists of dates, etc.

\item Instances of Python classes which you write yourself are not
acceptable, unless you implement a \texttt{__hash__} or \texttt{__cachekey__}
method in the class. Consult the Skunk mailing list if you want
to make instances of your own classes acceptable arguments to 
cacheable components.

\end{itemize}


%\textbf{Technical note:} As of Python 1.5.2, it is not accurate 
%to say that an object must be hashable to be a proper argument
%to a cacheable component. Python 1.5.2 makes *all* objects hashable,
%and that is a little stupid. It's more accurate to say that
%SkunkWeb accepts all basic data types, plus instances of classes
%which implement either a \texttt{__cachekey__} or a \texttt{__hash__}
%method that 
%returns an integer. (See the Python Library Reference about the \texttt{hash}
%function under "Built-in Functions", section 2.3.)



\section{\texttt{<:datacomp:>} and \texttt{<:return:>}}
\label{tagdatacomp}

\texttt{<:datacomp var name [\emph{named_args}] [cache=no] [defer=no] [__args__]:>}

Data components are like regular STML components, with these differences:

\begin{itemize}
\item Data component file end with the file extension \texttt{.dcmp}
instead of \texttt{.comp}.

\item 
Any ``output'' of a data component is ignored. Instead, the data
component must return a Python object with the \texttt{<:return:>}
tag.

\end{itemize}


Think of a data component as a Python function. 
You call it with attributes instead of arguments, you write it 
in STML instead of straight Python, but it returns a Python 
object just like a Python function.

The \texttt{<:datacomp:>} tag is used like the 
\texttt{<:component:>} \doref{tagcomponent}
tag, except for the first argument, \texttt{var}. 
\texttt{var} must be a string which names the local 
variable that will hold the return value of the 
data component:

\begin{verbatim}<:datacomp returnObj foo.dcmp arg1="foo":>
\end{verbatim}

In the above tag, the variable named \texttt{returnObj} 
will hold the object returned by data component \texttt{foo.dcmp}.

The \texttt{<:return:>} tag inside the data component tells
the component to stop executing and return immediately
the result of the Python expression in the tag. Examples:

\begin{verbatim}<:return `someVar`:>
<:return `5`:>
<:return `{'hey': 'dude', 'i': 'made', 'a': 'dictionary'}`:>
\end{verbatim}

If you ``forget'' to put a return tag in your data component, 
the object returned is the \texttt{None} object, just like a Python 
function that forgets to return something.

The \texttt{__args__} argument works exactly like its
\texttt{<:component:>} counterpart.

Like calling regular components from Python code, a way to get the
value of a datacomponent is:
\begin{verbatim}
import AE.Component
val = AE.Component.callComponent(name, argdict, 	
        cache = whatever, compType=AE.Component.DT_DATA)
\end{verbatim}
See section \ref{whatevervals}, page \pageref{whatevervals} for valid
values of \texttt{whatever}.

\subsection{Cacheable data components}
\label{tagdatacompcaching}

As with regular components, you may call data components with
\texttt{cache=yes} (or \texttt{cache=defer} or
\texttt{cache=force}). Data components use the \texttt{<:cache:>}
\doref{tagcache} tag to specify how long to cache, just like
components do.

Arguments to cached data components must obey the same 
restrictions as arguments 
to cached regular components \doref{tagcomponentrestrictions}.

Cached data components must also obey another restriction: 
the object returned by a data component must be \emph{pickleable}. 
That's Python's way of saying ``serializable''. If you don't 
know much about Python's \texttt{pickle} module, don't worry. 
The rule of thumb is: practically any kind of object you work 
with is pickleable. All Python basic data types can be pickled, 
as can instances of any classes you design. Therefore there is 
truly very little restriction on the kinds of objects you can 
return from a data component.




\section{\texttt{<:type:>}}
\label{tagtype}

\texttt{<:type \emph{named_args}:>}

\textbf{Note:} This tag is for advanced use by developers
who understand STML components and the Python type system.

The \texttt{<:type:>} tag allows you to to 
run-time type checking on variables.
Only keyword arguments are allowed.  The \texttt{<:type:>} 
tag may appear anywhere in an STML document. The argument name 
is the name of the variable that you wish to check.  
The value is a Python expression that is either a type, 
a class, or a tuple of types and/or classes.


The named variables are then checked (in no particular order) to see
if they pass an 
\texttt{isinstance} check with the value, or in the case where the
value is a tuple, check to see if \texttt{isinstance} succeeds on any
of the types/classes in the tuple.  If there is no match, a
\texttt{TypeError} will be raised.

For convenience, the type objects from the \texttt{types} module
are available in the global document namespace.  See the \emph{Python
Library Reference} section on the \texttt{types} module for details
on the type objects contained in the \texttt{types} module.


Passing values other than classes or types is erroneous and will raise
a \texttt{TypeError} at run-time.  If a named variable doesn't exist, a
\texttt{NameError} will be raised.

Some examples:
\begin{verbatim}
<:type x=`IntType` y=`StringType`:>
\end{verbatim}  
Check that \texttt{x} is an integer and that \texttt{y} is a string.

\begin{verbatim}
<:type foo=`(IntType, MyClass)`:>
\end{verbatim}
Checks that \texttt{foo} is either an integer, or an instance of
\texttt{MyClass}.

\begin{verbatim}
<:type foo=`y.__class__`:>
\end{verbatim}
Checks to see that \texttt{foo} is an instance of the same class as
\texttt{y}.



\section{\texttt{<:cache:>}}
\label{tagcache}

\begin{verbatim}<:cache [ until | duration ]:>\end{verbatim}

This tag, located in STML components and data components
(it's not allowed in non-components), tells SkunkWeb how long,
or until when, to cache the component's output (or
data component's return object) if it is called with
\texttt{cache=yes}.

This tag accepts one argument, either \texttt{until}
or \texttt{duration}. You must pass this argument by name:
\texttt{<:cache until=foo:>} or
\texttt{<:cache duration=foo:>}.

If you do not use either argument, the component will cache for around
30 seconds by default. The default cache length is configurable in
your SkunkWeb configuration file using the
\texttt{defaultExpiryDuration} configuration variable.

\subsection{Using the \texttt{duration} argument}

If you specify \texttt{duration}, the value should be 
a string specifying the length of time to cache. It's a 
string with no whitespace, and a sequence of integer-letter
pairs specifying untis of time:

\begin{argdesc}

\item[\emph{integer} d]
Cache for \emph{integer} number of days.
\item[\emph{integer} h]
Cache for \emph{integer} number of hours.
\item[\emph{integer} m]
Cache for \emph{integer} number of minutes.
\item[\emph{integer} s]
Cache for \emph{integer} number of seconds.
\end{argdesc}

You may use any combination of these integer-letter pairs,
in any order; SkunkWeb will add them all up and cache for the sum
period of time:

\begin{verbatim}
# cache for 10 minutes
<:cache duration=10m:>
# cache for 10 hours, 10 minutes
<:cache duration=10h10m:>
# cache for 2 days, 40 seconds
<:cache duration=40s2d:>
# cache for 60 imnutes (silly)
<:cache duration=40m20m:>
\end{verbatim}

\subsection{Using the \texttt{until} argument}

If you specify \texttt{until}, the value can be one
of many things, all intended to indicate a point in time
until which to cache. The \texttt{until} argument accepts
any one of the following:

\begin{itemize}
\item A string of the form \texttt{"hh:mi"} or
\texttt{"hh:mi:ss"}, specifying a time in hours, minutes,
and optionally seconds, on a 24-hour clock. This tells SkunkWeb
to cache until the next occurrence of that time. If you say
\texttt{until=06:00}, and SkunkWeb executes the component at 
05:30, it will cache for 30 minutes; but if SkunkWeb executes the 
component at 07:00, it will cache for 23 hours until the next
day at 06:00.

\item A string of the form \texttt{":mi"}, meaning 
``minutes after the hour''. This tells SkunkWeb to cache until
that many minutes after the current hour, or after the
next hour if those minutes have already passed.

\item A \texttt{DateTime} object, as produced by functions in the
\texttt{Date} or \texttt{DateTime} modules, or retrieved
from a database using skunk.org's \texttt{SQL} module
or SQL personality of SkunkWeb.

\item A string of the form \texttt{"yyyy-mm-dd"},
\texttt{"yyyy-mm-dd hh:mi"}, or \texttt{"yyyy-mm-dd hh:mi:ss"}.
These strings specify dates according to the ISO 
format supported by the \texttt{Date} module.

\item A string of the form \texttt{"mm/dd/yyyy"},
\texttt{"mm/dd/yyyy hh:mi"}, or \texttt{"mm/dd/yyyy hh:mi:ss"}.
These strings specify dates according to a common
U.S. date format. These string forms are only supported
for backward compatibility; if you must use a string
specifying an explicit date, use the ISO format above.

\item A \texttt{RelativeDateTime} object from the \texttt{DateTime}
module. these objects allow you to specify very complex
relative dates, such as ``the first Sunday of next month
at noon''. Be careful with these objects: unless you
read the documentation fully, you can end up constructing
a \texttt{RelativeDateTime} that moves you \emph{backwards} in time.
If you do that, the \texttt{<:cache:>} will raise
an exception telling you of your mistake.

\item A tuple or list containing ANY combination of the above
items. In this case, SkunkWeb will evaluate all items, and choose
the result that is nearest the current date/time for 
the cache expiration. This means you can do neat complex
cache control:

\begin{verbatim}# expires the cache at 0, 15, 30, and 45 minutes after the hour
<:cache until=`(":00", ":15", ":30", ":45")`:>

# expires the cache at 15 minutes after the hour,
# and also at 06:00 every day, and also
# at 5pm (17:00) on Sundays.
<:import DateTime "RelativeDateTime":>
<:cache until=`(":15", "06:00", 
   RelativeDateTime(
        weekday=0, hour=17, 
        minute=0, second=0))`:>

\end{verbatim}
\end{itemize}




\section{\texttt{<:import:>}}
\label{tagimport}

\begin{verbatim}<:import name [items] [as]:>\end{verbatim}

Imports a Python code module into the STML template. This
is practically identical to the \texttt{import}
statement in Python.

If the \texttt{items} parameter is specified, it should be 
either \texttt{'*'} or, a space delimited list of names 
in the module to copy into the STML template's namespace. 
If items is '*', all names in the module (that don't begin
with '_') are copied into the STML template's namespace.

If the \texttt{as} parameter is specified, it acts just like the
\texttt{as} clause to the Python \texttt{import} statement.
Some examples, with their equivalents in straight Python:

\begin{argdesc}

\item[<:import string:>]
\texttt{import string}
\item[<:import string find:>]   \texttt{from string import find}
\item[<:import string "find replace join":>]   \texttt{from string import find, replace, join}
\item[<:import string *:>]   \texttt{from string import *}
\item[<:import string find string_find:>] \texttt{from string import find as string_find}
\end{argdesc}

\textbf{Note on where modules come from:} When you
do an \texttt{<:import foo:>} in STML, the SkunkWeb server
first looks for the module \texttt{foo} in Python's standard
module directories. If it does not find the module in any
of those directories, SkunkWeb then looks in your custom SkunkWeb
module library, which is usually \texttt{/usr/local/skunk/site-libs}
(but could be changed by changing the \texttt{ImportPath}
configuration variable)
for a file \texttt{foo.py}.
If SkunkWeb still cannot find the module \texttt{foo}, it raises an 
\texttt{ImportError}.



\chapter{HTML Helper Tags}
\label{stmlrefurl}

The STML tags in this section help you construct URLs and common HTML
tags which use URLs for links, images, forms, and redirects.  These
tags are only enabled if using the \texttt{AE} library under the
SkunkWeb server or under \texttt{swpython}.  Chances are exceptionally
fair that this is the case.

The default behavior of these tags is very simple.  Your SkunkWeb
administrator or lead developer may customize the behavior of these
tags to fit the needs of your website.  If these STML tags behave
differently or strangely in your web application, it's probably
because they've been customized; ask your local SkunkWeb guru.

\section{\texttt{<:url:>}}
\label{tagurl}

\begin{verbatim}<:url path [queryargs] [text] [noescape] [extra_args]:>\end{verbatim}

Produces a URL or an HTML \texttt{<A>} tag. 
All arguments except \texttt{path} are optional.

\begin{itemize}
\item \texttt{queryargs} can be a dictionary of arguments which you
would like added to the URL as a query string, for example, the
dictionary \texttt{\{'foo':5, 'bar':'yes'\}} gets translated to
appending \texttt{?foo=5\&bar=yes} to the url. Defaults to an empty
dictionary.

\item \texttt{text} is an optional string. If it is present, the tag
will not just produce a URL, but a full HTML \texttt{<A>}
tag with an \texttt{HREF} attribute holding the URL and the text in
between. 

\item The \texttt{noescape} argument, if present,
tells SkunkWeb \emph{not} to escape URL-unsafe characters in the \texttt{path}
argument, such as spaces or extended characters. \emph{By default,
SkunkWeb will escape all non-safe characters except for the \texttt{/}
character, which it leaves intact. It will escape the \texttt{:}
character in your \texttt{path} argument, however.}


\item \texttt{extra_args} is not an argument name: it just means 
that any other arguments you pass by name will be put in the
\texttt{<A>} tag if \texttt{text} is not empty. Example:

\begin{verbatim}<:url /index.html text="Home Page" target="_blank" :>
# produces
<a href="/index.html" target="_blank">Home Page<a>
\end{verbatim}



\end{itemize}

The equivalent Python component code would be:
\begin{obeylines}\ttfamily
import templating.UrlBuilder
print templating.UrlBuilder.url(path, queryargs, text,
     \textit{extra_args_as_dict}, noescape)
\end{obeylines}



\section{\texttt{<:img:>}}
\label{tagimg}

\begin{verbatim}<:img path [queryargs] [noescape] [extra_args]:>\end{verbatim}

creates an appropriate HTML \texttt{<IMG>} tag. 
Arguments similar to \texttt{<:url:>} \doref{tagurl}. 

The \texttt{queryargs} argument, if present, allows you to pass a
dictionary of arguments as a querystring to the image document, as
is useful for dynamically generated images (thumbnails, graphs,
etc.). Defaults to an empty dictionary.

The \texttt{noescape} argument, if present,
tells SkunkWeb \emph{not} to escape URL-unsafe characters in the \texttt{path}
argument, such as spaces or extended characters. \emph{By default,
SkunkWeb will escape all non-safe characters except for the \texttt{/}
character, which it leaves intact. It will escape the \texttt{:}
character in your \texttt{path} argument, however.}

The \texttt{extra_args} parameter is not actually a real argument name, 
but is a pass through to let you pass through arguments to 
the \texttt{<img>} tag.  Keyword arguments only.

Examples:

\begin{verbatim}<:img path=this.gif border=1:>
# produces <IMG SRC="this.gif" border="1">
\end{verbatim}

The equivalent Python component code would be:
\begin{obeylines}\ttfamily
import templating.UrlBuilder
print templating.UrlBuilder.image(path, queryargs,
     \textit{extra_args_as_dict}, noescape)
\end{obeylines}


\section{\texttt{<:form:>}}
\label{tagform}

\texttt{<:form path [noescape] [\emph{extra_args}]:>}

Creates an appropriate HTML \texttt{<FORM>} ``start'' tag. 
Arguments are similar to the \texttt{<:url:>} \doref{tagurl}.
The \texttt{extra_args} parameter is not actually a real argument name, but
is a pass through to let you pass through arguments to the
\texttt{<FORM>} 
tag such as \texttt{METHOD}. These arguments must be passed by name only.

\textbf{IMPORTANT NOTE!} This tag only generates the ``start'' HTML
form tag, but \emph{not} the closing \texttt{</FORM>} tag.
Always remember to close your HTML forms with the raw HTML tag
\texttt{</form>}.

The equivalent Python component code would be:
\begin{obeylines}\ttfamily
import templating.UrlBuilder
print templating.UrlBuilder.form(path,
     \textit{extra_args_as_dict}, noescape)
\end{obeylines}


\section{\texttt{<:hidden:>}}
\label{taghidden}

\texttt{<:hidden \emph{named_args}:>}

Creates a series of HTML hidden input fields for each named attribute
in the tag, escaping html entities as appropriate.

Example:

\begin{verbatim}
<:set that "Hello//world":>
<:hidden this=`that` count=`5`:>
\end{verbatim}

produces

\begin{verbatim}
<INPUT TYPE=HIDDEN NAME="this" VALUE="Hello%2F%2Fworld">
<INPUT TYPE=HIDDEN NAME="count" VALUE="5">
\end{verbatim}


The Python component code to mimic this tag is:
\begin{obeylines}\ttfamily
import templating.UrlBuilder
print templating.UrlBuilder( \emph{named_args_as_dict} )
\end{obeylines}

\section{\texttt{<:redirect:>}}
\label{tagredirect}

\begin{verbatim}
<:redirect url:>}
<:redirect path [queryargs] [noescape]:>
\end{verbatim}

If called with the named argument \texttt{url}, the argument
should contain an absolute URL. Example:

\begin{verbatim}<:redirect url="http://www.starmedia.com/":>\end{verbatim}

Alternately, you may call this tag with arguments
similar to those for 
\texttt{<:url:>} \doref{tagurl}. 
\texttt{path} is the path, \texttt{queryargs} an optional
dictionary of arguments to put in the query string,


The \texttt{noescape} argument, if present,
tells SkunkWeb \emph{not} to escape URL-unsafe characters in the \texttt{path}
argument, such as spaces or extended characters. \emph{By default,
SkunkWeb will escape all non-safe characters except for the \texttt{/}
character, which it leaves intact. It will escape the \texttt{:}
character in your \texttt{path} argument, however.}


\textbf{Please note} that either \texttt{url} or \texttt{path}
must be passed by name, like \texttt{url=foo} or 
\texttt{path=foo}.

%Extra note about the using \texttt{path} instead of \texttt{url}:
%SkunkWeb must find the "server name" and prepend it to \texttt{path},
%and it finds the server name in the \texttt{REQUEST} object. The
%\texttt{REQUEST} object is not available in components called 
%with \texttt{<:component:>}, however. You \textbf{must} use
%\texttt{url} inside these components, or you will get an error.


The equivalent Python component code is: 
\begin{obeylines}\ttfamily
import templating.UrlBuilder
templating.UrlBuilder.redirect(path, url, queryargs, noescape)
\end{obeylines}



\chapter{Internet Services}
\label{stmlrefnet}
Originally, STML planned to have many tags
which performed operations on common Internet services.
The skunk.org team soon realized, however, that
it was a better approach to use Python's standard
modules for Internet services. If you need to perform
some operation, such as fetching a web page from a server,
or talking to a UseNet newsgroup, or fetching mail
from a POP mail server, write your own ``wrapper module''
for use in SkunkWeb, and make use of Python's fantastic
standard modules, such as \texttt{httplib} and
\texttt{urllib}.

STML gives you only one tag for Internet services:
a tag for sending electronic mail, since that is
a common task performed by web applications.

\section{\texttt{<:sendmail:>}}
\label{tagsendmail}
\texttt{<:sendmail to_addrs subject msg [from_addr]:>}

The \texttt{to_addrs} parameter is a list of strings 
(or a single string) containing the email address(es). The
\texttt{subject} argument is a 
string containing the subject line. The \texttt{msg} argument contains
the entire message body as a single string. The \texttt{from_addr} argument
is an optional string with the email address of the sender, 
and it defaults to the address in the
\texttt{FromAddress} configuration variable.


To do the equivalent in Python components:
\begin{verbatim}
import templating.MailServices
templating.MailServices.sendmail ( 
    to_addrs, subject, msg, from_addr )
\end{verbatim}


\chapter{Text Messages with Message Catalogs}
\label{stmlrefmsg}
In many web applications, you will find that you need
to display text messages to the web user, such as 
error messages, alert messages, ``success'' messages,
and so forth. Yet you do not wish to type these messages
directly into your STML templates; what if you use them
in many places on your Web site, and then you have to
change the text of a message? You do not want to have
to hand-edit each STML document to make the change.

STML provides a nice way to manage text messages
and avoid ``hard-coding'' them into your STML documents:
\textbf{message catalogs}. An STML message catalog acts somewhat
like a Python dictionary: you give it a \emph{key},
which is a string, and the mesage catalog looks up
the message for that key. Unlike a dictionary, an STML
message catalog does not raise an error if it cannot
find a message under the key you request; instead,
the message catalog returns an empty string \texttt{""}.

Unlike a dictionary, a message catalog is not an object
which you work with directly. Instead,
you use the \texttt{<:msg:>} tag to access messages
in the catalog.

There are two kinds of message catalogs in STML:

\begin{description}
\item[Regular, or ``simple'', catalogs] Dictionary-like
message catalogs. Each message ``name'' is like a dictionary
key, with a single message stored for each key.

\item[Multi-lingual, or ``complex'', catalogs:] Like a
simple catalog, except that each message key stores
multiple messages, one message for each language in the
catalog. (Currently supported languages are \texttt{'esp'} (Spanish),
\texttt{'por'} (Portuguese), and \texttt{'eng'} (English). 
\end{description}

The following sections explain the three tags which
provide message catalog features.


\section{\texttt{<:catalog:>} for message catalogs}
\label{tagcatalog}

\texttt{<:catalog catname name:>}

Loads a  message catalog 
named \texttt{catname} into the local variable 
\texttt{name}. Once the catalog is loaded, you may
use the \texttt{<:msg:>} \doref{tagmsg}
tag to access messages in the catalog.

The catalog name in \texttt{catname} is of the path to the message
catalog (simple message catalogs have the extension \texttt{.msg}, complex, or multi-lingual ones have an \texttt{.cat} extension)
, just like names for includes,
components, and data components.

The equivalent Python code is:
\begin{verbatim}
import AE.Cache
name = AE.Cache.getMessageCatalog ( catname )
\end{verbatim}


\section{\texttt{<:msg:>} for retrieving messages}
\label{tagmsg}

\texttt{
<:msg \emph{catname msgname} [lang] 
    [fmt] [substitution_args]:>
}

This tag accesses a message in the catalog object
previously loaded into the local variable \texttt{catname}
with a \texttt{<:catalog:>}
tag. If the message does not
exist, the message is an empty string. The returned message
is then displayed in the template output, as in a
\texttt{<:val:>} \doref{tagval} tag.

For example, you load a simple message catalog file
\texttt{foo/bar/mycatalog.msg} with the following tag:

\begin{verbatim}
<:catalog foo/bar/mycatalog.msg thisCat:>
\end{verbatim}

The catalog object is now loaded into the local variable
\texttt{thisCat}. Then, to access the message under the name
\texttt{welcome_user}, you use the tag:

\begin{verbatim}
<:msg thisCat.welcome_user:>
\end{verbatim}

If the catalog object is a multi-lingual message catalog,
you may pass the optional argument \texttt{lang}. If passed,
it should be one of the strings \texttt{"esp"}, 
\texttt{"por"} or \texttt{"eng"}, for Spanish, Portuguese,
or English respectively. (For backward compatibility, the
string \texttt{"spa"} is acceptable for Spanish.)
If you do not specify \texttt{lang},
the SkunkWeb server will look for a global variable in your template
called \texttt{lang}, and use that.  You can configure your SkunkWeb
server to set this global variable automatically.  If SkunkWeb cannot
find this global variable, it will raise an error.

For example, you load a multi-lingual message catalog file
\texttt{foo/bar/mycatalog.cat} with the following tag:

\begin{verbatim}
<:catalog foo/bar/mycatalog.cat thisCat:>
\end{verbatim}


The catalog object is now loaded into the local variable
\texttt{thisCat}. Then, to access the message under the name
\texttt{welcome_user}, for Portuguese, you use the tag:

\begin{verbatim}
<:msg thisCat.welcome_user lang=por:>
\end{verbatim}

Alternately, if you know that the global variable \texttt{lang}
is set properly, you can leave out the \texttt{lang} argument.

\texttt{fmt} is an optional argument, and works just like 
the \texttt{fmt} argument in the 
\texttt{<:val:>} \doref{tagval}
tag.

The \texttt{substitution_args} marker is not an argument, but instead
means that any named arguments you specify will be used to do
``variable substitution'' if the message contains placeholders
for variable substitution. See the 
variable subtitution \doref{msgcatvarsub}
section for details.

The equivalent Python code is:
\begin{obeylines}\ttfamily
import AE.MsgCatalog
import DT.DTCommon
print AE.MsgCatalog.getMessage ( 
    catalog, msgname, lang, DT.DTCommon.ValFmtRgy[ fmt ], 
    \emph{substitution_args_as_dict} )
\end{obeylines}

\section{Variable Subsitution in Messages}
\label{msgcatvarsub}

You can, if you wish, put ``placeholders'' in
messages and then, in the 
\texttt{<:msg:>} \doref{tagmsg}
tag, subtitute the values of variables for those
placeholders.

To make a placeholder, choose the name of the variable, 
and then insert that variable between double brackets in
the message, for example:

\begin{verbatim}
"This is a message with a variable substitution [[thisVar]]."
\end{verbatim}

Then, when using this message with the
\texttt{<:msg:>} \doref{tagmsg}
tag, pass a named argument as a tag attribute:

\begin{verbatim}
<:msg thisCatalog.thisMessage thisVar="here":>
\end{verbatim}

which will produce the output:

\begin{verbatim}
This is a message with a variable substitution here.
\end{verbatim}


\section{Editing Message Catalogs}
\label{msgcatedit}

Creating message catalog files is a snap. You can
use skunk.org's SkunkDAV software, which has a built-in
message catalog editor. Or you can create an edit
these files directly using your favorite text editor.

Message catalogs are just like Python dictionaries.
And in message catalog files, the message names and
values are expressed exactly as a big Python dictionary.
For example, the contents of the simple message catalog
\texttt{my_messages.msg}:

\begin{verbatim}
{
    'welcome': 'Welcome to Skunk.',
    'unauthorized': 'Go away!'
}
\end{verbatim}


As you can see, the file's contents are a
Python dictionary. The keys of the dictionary
are the message names (always strings), and
the values are the messages themselves (always
strings).

Multi-lingual message catalogs are ``two-level''
Python dictionaries. The language is the top level,
and each language key has a full dictionary of 
message names and values. Here's the above simple
message catalog as a multilingual catalog in
the file \texttt{my_messages.cat}:

\begin{verbatim}
{
    'eng': {
        'welcome': 'Welcome to Skunk.',
        'unauthorized': 'Go away!'
    },
    'esp': {
        'welcome': 'Bienvenido a Skunk.',
        'unauthorized': 'Que se vaya...'
    }
}
\end{verbatim}


When you edit message catalog files directly,
you may type anything that is legal in a Python
dictionary literal: any whitespace or indenting you wish,
even comments. Please note, however, that editing
message catalogs in the SkunkDAV editor results in 
all of your comments being lost; SkunkDAV
reformats the dictionary whenever it saves the
message catalog file.

\chapter{Documenting Your STML Code}
\label{stmlrefmisc}



\section{\texttt{<:doc:>}}
\label{tagdoc}

\begin{verbatim}<:doc:>
    [ template, component, or datacomp documentation ]
<:/doc:>
\end{verbatim}


This tag is a great tool for your development projects!
At the beginning of any STML document,
place a single \texttt{<:doc:>} tag. Inside the tag,
write text documentation about the document (any STML tags will
be ignored). Describe the role of the document in your application.
Also describe the user arguments it looks for (if it's a template)
or the arguments it expects (if it's a component or data component).
If your STML document raises exceptions with the \texttt{<:raise:>}
tag, describe which exceptions it can raise.
For data components, also describe what kinds of objects are returned.

The \textbf{skunkdoc} tool, which comes with the SkunkWeb software, can
then extract all of your STML documents' \texttt{<:doc:>} 
contents, and construct a surfable, cross-referenced HTML
documentation set for your website. (It also will include all of
your custom SkunkWeb Python modules and their documentation.)

\textbf{\emph{Use the \texttt{<:doc:>}
tag; you will be forever glad you did.}}

Note that this tag considers all of its contents to be text; STML tags
are not parsed. The resulting documentation will be in ``plaintext'', unless
you use the magic \textbf{skunkdoc} XML format: make the first two
non-whitespace characters in the tag \texttt{**} and the skunkdoc tool
will parse them as XML. See the manual ``Documentation Tools'' on the skunk.org
website to learn all about the skunkdoc XML format.


\section{\texttt{<:comment:>} or \texttt{<:\#:>}}
\label{tagcomment}

\begin{verbatim}
<:comment:>
    [ stuff you want commented out ]
<:/comment:>
\end{verbatim}

Similar to the HTML \texttt{<!-- -->} comment. Everything until
the closing \texttt{</comment>} tag is ignored by the SkunkWeb server,
and will neither be executed nor show up in the output. 
Comments do not nest,
and whatever occurs inside a \texttt{<:comment:>} tag must
be legal STML. 

As a shortcut, this tag can also be written
\texttt{<:\#:><:/\#:>}.

   


\section{\texttt{<:* ....... *:>}}
\label{fullcomment}
\begin{verbatim}
<:*
   [ stuff you want commented out ]
*:>
\end{verbatim}

This form of comment is different than the previous kind.  The
previous kind is good when you want to either disable the contained
code/text, but stinks when the contained tag structure isn't
syntactically correct.  The contents of this form of comment are
\emph{totally} ignored and contain anything, except a \texttt{*:>}.


\chapter{Python Components}
\label{pythoncode}
Python components (we will use the term ``component'' here to mean
regular components, data components or top-level documents) 
are a way to write components and/or data components using straight
Python.  They are useful for when the operation of the component is
code-heavy and/or not terribly markup-intensive.

To call other components from within Python components, there is a
function in the \texttt{AE.Component} module called, aptly enough,
\texttt{callComponent} whose definition looks like this:

\begin{verbatim}
def callComponent (name, argDict, cache = 0,
                   defer = None, compType = DT_REGULAR)
\end{verbatim}

The arguments are:
\begin{argdesc}
\item[name] the path to the component (can be relative)
\item[argDict] dictionary of arguments to the component
\item[cache] equivalent to the \texttt{cache} parameter in the
component tags, but see section \ref{whatevervals}, page
\pageref{whatevervals} for actual values to use.
\item[compType] either the default (a regular
component), \texttt{AE.Component.DT_REGULAR},
\texttt{AE.Component.DT_DATA} for a data component or
\texttt{AE.Component.DT_INCLUDE}
\end{argdesc}



The return value is either the output of the component, in the case of
a regular component, or the return value of the component, in the case
of a data component.





\section{Top-Level Python Documents}
\label{pythontl}
Top-level Python documents \emph{are} url accessible with the file
extension of \texttt{.py}.  They recieve
the \texttt{CONNECTION} object in their global
namespace and standard output goes to the browser (for the body text, if
you wish to set response headers, they must be set using the
\texttt{CONNECTION} object).  For example, the following is a valid
top-level python script: 

\begin{verbatim}
print "The current URI is: <B>%s</B>" % CONNECTION.uri
\end{verbatim}




\section{Python Include Components}
\label{pythonincl}
The Python include component extension is \texttt{.pyinc}.
Python include components have similar semantics as their STML
counterparts.  All of the variable accessible in the calling documents
namespace will appear in the global namespace of the Python component
being called, and all changes (additions, changes, deletions) to those
variables will appear in the calling document.  

Producing output works the same as in top-level python components.




\section{Regular Python Components}
\label{pythonreg}
Regular Python components get the default global namespace (as would
an STML component) and the component arguments in the global
namespace.  The catch is that there is no \texttt{<:cache:>} tag
in Python,  unfortunately, but this can be done by setting the global variable
named \texttt{__expiration} to the number of seconds past the Epoch
(12:00am 1/1/1970 UTC) at which this component will expire.  This may sound
nasty, but of course there is a simple way to compute this value!  In
the \texttt{Date.TimeUtil} module) there are two functions which mimic STML's
\texttt{<:cache:>} tag, \texttt{convertDuration} and
\texttt{convertUntil}. 

The \texttt{convertDuration} function takes the same argument as the
\texttt{duration} argument to the \texttt{<:cache:>} tag as a string.  So
to say your component should cache for five minutes, you'd say:

\begin{verbatim}
import Date.TimeUtil
__expiration = Date.TimeUtil.convertDuration("5m")
\end{verbatim}


The \texttt{convertUntil} function takes the same argument as the
\texttt{until} argument to the \texttt{<:cache:>} tag.

Producing output works the same as in top-level python components.



\section{Python Data Components}
\label{pythondata}
Python data components extension is \texttt{.pydcmp}.  Python data
components are similar to regular Python components as STML data
components are similar to regular STML components.  Unfortunately,
there is no \texttt{<:return:>} tag in Python, nor can the Python
\texttt{return} statement be used (since it would be at the top
level, where it is a syntax error).  So, in order to return a value
from the data component, you 
raise an exception of the proper type, \texttt{ReturnValue}, with the
exception value being the return value.  The ReturnValue exception is
preloaded into the global namespace in which the component is
executed, so it doesn't need to be imported.

For example, to return the value of the variable foobar from a data
component:
\begin{verbatim}
raise ReturnValue, foobar
\end{verbatim}


The neat thing about this method of returning the value is that you
can raise the exception from anywhere in the data component, i.e. from
within functions, classes, wherever.

Raising other exceptions will, of course, invoke the normal error
handling routines.  Raising no exception returns the value \texttt{None}

Setting the caching parameters of a data component is identical to that of
regular Python components.

As with normal STML data components, any generated output is ignored.



\chapter{Namespaces}
\strong{ATC--FIXME}
\label{namespaces}


\section{The Global Namespace}
\label{globalns}

The global namespace contains:
\begin{itemize}
\item The contents of the \texttt{types} module that don't start with a
leading underscore. 
\item The \texttt{SkunkWeb} module.
\end{itemize}

\section{The \texttt{CONNECTION} Object}
\label{connectionobj}

\chapter{User Modules}
\label{usermodules}
The \texttt{site-libs} directory contains application-specific modules
that (depending on the state of the \texttt{UserModuleCleanup}
\texttt{templating.conf} configuration option) get cleaned out on a
per-request basis.  The upshot of this is that this way you can do
development on these modules without having to restart SkunkWeb each time
you make a change.  Since \texttt{site-libs} is normally already in
the import path, all you need to do is put stuff there and it is
importable by the standard rules of Python.

\cleardoublepage

\printindex

\end{document}


\documentclass{manual}
\usepackage{makeidx}
\usepackage[nottoc]{tocbibind} % make so bib and ind are in toc
\title{The SkunkWeb Developers Guide}
\author{Drew Csillag and Jacob Smullyan}
\release{3.0}
\setshortversion{3}
\makeindex

\begin{document}

%my commands here
\newcommand{\Location}{\texttt{Location}}
\newcommand{\PAR}{\texttt{par}}
\newcommand{\mskunkroot}[1]{\textit{SkunkRoot}\texttt{#1}}
\newcommand{\swpython}{\texttt{swpython}}
\newcommand{\connection}{\texttt{Connection}}
\newcommand{\NOTE}[1]{\textbf{\Large #1}}
\newcommand{\None}{\texttt{None}}
\newcommand{\nolocation}{[N]}


%function argument list environment
\newcommand{\argdescitem}[1]{\hspace\labelsep
                                \normalfont\ttfamily #1\ }
\newenvironment{argdesc}{\begin{list}{}{
        \renewcommand{\makelabel}{\argdescitem}
}
}{\end{list}}

%begin titlepage stuff
\maketitle
\ 
\vfill 
\noindent
This file documents the SkunkWeb Web Application Framework.

\noindent
Copyright (C) 2001 Andrew T. Csillag, Jacob Smullyan

\noindent
Permission is granted to make and distribute verbatim copies of
this manual provided the copyright notice and this permission notice
are preserved on all copies.

%%end titilepage stuff

\tableofcontents

\chapter{Introduction}
This is the intro

%%%%%%%%%%%%%%%%%%%%%%%%%%%%%%%%%%%%%%%%
\chapter{Core Services}
The core services that SkunkWeb supplies are (intentionally) fairly minimal. 
The services provided are:
\begin{itemize}
\item The reading of configuration file(s)
\item The loading of Python modules or packages
\item A logging service
\item A multi-process TCP server framework
\item Two hooks, one for pre-\texttt{fork} initialization and one for
post-\texttt{fork} initialization.
\item An interactive Python prompt (a.k.a. \texttt{swpython})
\end{itemize}

\section{Configuration}
Configuration proceeds by first checking to see if a \verb!-c! or
\verb!--config-files=! option was specified (as a colon delimited list
of configuration files to read), if not, a configuration-time
hardcoded default is used.  A namespace is set up with a small amount
of preloaded stuff \footnote{The \texttt{Scope} ``directives'' is loaded and the default \texttt{SkunkRoot} is set}
(and the configuration files executed (they're
Python programs really) in this namespace in the order given on the
command line (if given).

\subsection{The \texttt{SkunkWeb.Configuration} ``Module''}

The current configuration is available via the
\texttt{SkunkWeb.Configuration} ``module''.  In reality, it is not a
real Python module, but a class instance that has been set up to be
importable.  The implementation of the ``module'' is the
\texttt{ScopeableConfig} class in  \verb!pylibs/ConfigLoader.py!.
(\texttt{ScopeableConfig} is itself a \texttt{scopeable.Scopeable} subclass
with a backwards compatibility layer that will probably be removed soon.)

\begin{methoddesc}{\mergeDefaults}{dict} Function taking any number of dictionary
arguments, and/or keyword arguments. For each key/value in each dictionary, (with 
any keyword arguments comprising a final dictionary), if the current
configuration already has a variable with that name, it is skipped,
otherwise a the configuration variable with that name is set to the
corresponding value. 
\end{methoddesc}

\begin{methoddesc}{\update}{dict} Updates the root configuration with the
contents of the argument (argument should be a dictionary-like
object). 
\end{methoddesc}
\begin{methoddesc}{\push}{} Overlay configuration with the contents
of the argument (should be a dictionary-like object). 
\end{methoddesc}
\begin{methoddesc}{\pop}{} Undo the last overlay.
\end{methoddesc}
\begin{methoddesc}{\trim}{} Undo all overlays, resulting in just the
root configuration. 
\end{methoddesc}
\begin{methoddesc}{\mash}{} No argument function that returns a
dictionary whose contents are the current view of the configuration.
\end{methoddesc}
\begin{methoddesc}{\mashSelf}{} Sets the root configuration to
\verb![self.mash()]!. 
\end{methoddesc}

Since the configuration is read prior to the services being loaded,
the configuration is available to services at their time of import.

The configuration variables show up as attributes in the module
\verb!SkunkWeb.Configuration!.  Thus, the configuration
variable \texttt{debugFlags} would appear as
\verb!SkunkWeb.Configuration.debugFlags!.

\section{Service Loading}
The loading of services is done by the normal Python import facility.
This being the case, the directory(ies) that the service
modules/packages live in \strong{must} be in \texttt{sys.path}.  When
SkunkWeb is installed, the services shipped with SkunkWeb are placed
in a directory that is added to \texttt{sys.path} by the bootloader
automatically.  If you decide to move them, or add services of your
own (which should be located elsewhere so that future upgrades won't
accidentally wipe it out), you can add paths to \texttt{sys.path} in
the configuration file like so:
\begin{verbatim}
import sys
sys.path.append('/some/other/path')
\end{verbatim}

Even though a service sounds like a big deal, services are merely
Python modules and/or packages that are loaded 
just after the configuration has been read in.  Conventionally though,
service packages/modules assume presence of the SkunkWeb environment
(i.e. configuration information, hooks, etc.) and plug into the
SkunkWeb evironment in some way.


\section{Logging}
Logging is done by the \verb!SkunkWeb.LogObj! module.  The main
functions of interest are:

\begin{funcdesc}{DEBUG}{debugFlag, message} Takes two arguments, a debug flag (explained below) and a
message string.  If the flag masked against \verb!Configuration.debugFlags!
results in a non-zero value, the message is logged to the debug flag.
\end{funcdesc}
\begin{funcdesc}{DEBUGIT}{debugFlag} Take a single argument, a debug flag.  If the flag
masked against \verb!Configuration.debugFlags! 
results in a non-zero value, returns 1, else 0.  Useful for doing
computation for debug messages that you don't want to do if you aren't
going to log said message.
\end{funcdesc}
\begin{funcdesc}{LOG}{message} Takes one argument, the message to log.  Used for logging
normal, but significant events.
\end{funcdesc}
\begin{funcdesc}{WARN}{message} Takes one argument, the message to log.  Used for logging
events that require attention.  Logs to the error log file. 
\end{funcdesc}
\begin{funcdesc}{ERROR}{message} Takes one argument, the message to log.  Used for logging
erroneous events.  Logs to the error log file.
\end{funcdesc}
\begin{funcdesc}{ACCESS}{message} Takes one argument, the message to log.  Used mainly by
the \texttt{AE} library to log access events, but is available for
general use.  Logs to the access log.
\end{funcdesc}
\begin{funcdesc}{logException}{} Logs the current exception to the error log.
\end{funcdesc}


If, when SkunkWeb is started, standard out and/or standard error are
not TTY's (i.e. hooked to a terminal), they will be redirected to the
\texttt{LOG} nad \texttt{ERROR} functions, respectively.

\section{The \texttt{Server} Object}
The \texttt{Server} object handles the routing of TCP connections to
appropriate handlers, presumably installed during the loading of
Python modules, to handle whatever protocol is spoken over that TCP
connection. 

In the \texttt{SkunkWeb.Server} module there is a single function of
interest: 
\begin{funcdesc}{addService}{sockaddr, func}
Opens up a socket and listens for connections, and when connections
arrive calls \texttt{func} (with the \texttt{accept()}ed socket to
handle them. 
\begin{argdesc}
\item[sockaddr] An address specfier of one of two forms:
\begin{enumerate}
\item \texttt{('TCP', 'hostname', port)} --- bind to the TCP address
\texttt{host}/\texttt{port} where \texttt{port} is an integer.  To
bind to all addresses on host, specify the empty string as the
hostname.
\item \texttt{('UNIX', '/path/to/socket' [,
\textit{socket_permissions} ])} --- bind to the Unix socket located at
\verb!/path/to/socket! and optionally set the permissions of the
socket to \texttt{\textit{socket_permissions}}.
\end{enumerate}
\item[func] A function to call (with the socket) when a connection
comes in.
\end{argdesc}
\end{funcdesc}

\section{Hook Objects}
Hook objects are used in SkunkWeb to provide extensibility.  They
allow you to setup functions to be called at specific points in
processing.  They can be viewed as a list of functions to call 
(the \texttt{Hook} class is actually a subclass of \texttt{UserList})
that calls the functions in the list, in order, until one of them
returns a non-\None\ value.  The return value of calling a hook then,
is the return value of the function that returned a non-\None\ value,
or if all of the calls returned \None, the return value is \None.

\section{Initialization Hooks}

\begin{datadesc}{ServerStart}  Called after services have been imported but
before the server \texttt{fork()}s.
\end{datadesc}
\begin{datadesc}{ChildStart}
Called in the child process just after the \texttt{fork()}
\end{datadesc}


%%%%%%%%%%%%%%%%%%%%%%%%%%%%%%%%%%%%%%%%
\chapter{The \texttt{web} Service}
\index{services!\texttt{web}}
The \texttt{web} service, is short provides base request phase
handling (extendable/configurable via hooks), an \connection\ object
and socket connectivity to 
Apache (via \texttt{mod\_skunkweb}).


\section{Protocol}
The protocol between the web listener (presumably Apache, but could be
something else) is as follows:
\begin{enumerate}
\item Client connects to server at some prearranged port
(\texttt{WebListenPorts}).
\item Server sends one byte (the byte value should be 0)
\item Client sends 10 byte ascii length of request (e.g. a length of
twenty would be sent as the ascii string \verb*!        20! (a 
\verb*! ! is a blank space)).
\item Client sends Python marshalled form of a dictionary whose
key/value pairs are:
\begin{description}
\item[\texttt{'stdin'}] the request body (in CGI-land, it's standard
input)
\item[\texttt{'headers'}] a dictionary of HTTP headers
\item[\texttt{'env'}] the CGI environment variables.
\end{description}
\item Server sends 10 byte ascii length of request --- as in (3).
\item Server sends full response (headers and body).
\item Both sides close connection.
\end{enumerate}

\section{Hooks Exposed}

\begin{datadesc}{InitRequest} Called with the raw request data as received from
Apache.  If the return value is a tuple of
(\textit{HTTP\_RESPONSE\_CODE}, string),  request processing stops and
the string (which is assumed to contain headers also) is sent back as
the response.
\end{datadesc}
\begin{datadesc}{HaveConnection} Called after the \connection\ object has been
constructed but before the location specific configuration has been
overlayed.
Argument is the \connection\ object.  If the
return value is an HTTP response code, the \connection\  object
is assumed to be complete and a response is sent.
\end{datadesc}
\begin{datadesc}{PreRequest} Called after the location specific configuration has been
overlayed.  Argument is the \connection\ object.  If the
return value is an HTTP response code, the \connection\  object
is assumed to be complete and a response is sent.
\end{datadesc}
\begin{datadesc}{HandleRequest} Called after the \texttt{PreRequest}
hook. Argument is the \connection\  object.  If the
return value is an HTTP response code, the \connection\  object
is assumed to be complete and a response is sent.  If no response code
is returned in the calling of this hook, a \texttt{'NoHandler'} string
exception is raised and processing proceeds to the
\texttt{RequestFailed} hook.
\end{datadesc}
\begin{datadesc}{PostRequest} Called after the response has been sent. Arguments
are the \connection\  object and the raw request data.  If the
exception occurs before the \connection\  has been constructed,
the connection argument is \texttt{None}, otherwise the raw request
data argument is \texttt{None}.
\end{datadesc}
\begin{datadesc}{CleanupRequest} Called after the \texttt{PostRequest} hook. Arguments
are the \connection\  object and the raw request data.  If the
exception occurs before the \connection\  has been constructed,
the connection argument is \texttt{None}, otherwise the raw request
data argument is \texttt{None}.
\end{datadesc}
\begin{datadesc}{RequestFailed} Called if there is an exception before a response
has been sent.  It should return the text of a full response
(i.e. headers and body).  After this hook returns (even if by a raised
exception), the \texttt{PostRequest} and \texttt{CleanupRequest}
handlers are called.
\end{datadesc}

\section{The \connection\  Object}
The \connection\ object is used to obtain information about the
request and to produce a response.  As such, it contains things like
the CGI arguments sent from forms, and things like the request and
response headers.  The attributes of the \connection\ object are as
follows:

\begin{memberdesc}[dictionary]{\_initRequestData}
 the raw data that the \connection\ object is
constructed from.  Is the same data that the \texttt{InitRequest} hook
is called with.
\end{memberdesc}
\begin{memberdesc}[integer]{\_status}
 The HTTP response code to send back to Apache.  If not
otherwise set via call to \texttt{setStatus()}, is 200.
\end{memberdesc}
\begin{memberdesc}[dictionary]{env}
 The CGI environment set by Apache.
\end{memberdesc}
\begin{memberdesc}[string]{\_stdin}
The request body text sent by the browser to Apache.
\end{memberdesc}
\begin{memberdesc}[HeaderDict]{requestHeaders} The HTTP headers of the request.
\end{memberdesc}
\begin{memberdesc}[Cookie]{requestCookie} The cookie sent by the browser.
\end{memberdesc}
\begin{memberdesc}[HeaderDict]{responseHeaders} The HTTP headers to be
sent in the response. 
\end{memberdesc}
\begin{memberdesc}[Cookie]{responseCookie} The cookie to be sent in
the response.
\end{memberdesc}
\begin{memberdesc}[File]{\_output} A file-like object that contains
the body of the response. 
\end{memberdesc}
\begin{memberdesc}[Browser]{browser} An object with three attributes
that describes the user agent as sent in the \texttt{User-Agent} HTTP header.  Those attributes are:

\begin{memberdesc}[string]{version} the version of the client browser
\end{memberdesc}
\begin{memberdesc}[string]{lang} the language of the client browser
\end{memberdesc}
\begin{memberdesc}[string]{name} the name of the client browse (e.g. Mozilla)
\end{memberdesc}
\end{memberdesc}
\begin{memberdesc}[dictionary]{args} A dictionary of CGI variables.
\end{memberdesc}
\begin{memberdesc}[string]{uri} The URI of the request.
\end{memberdesc}

\begin{methoddesc}{redirect}{url}  A function taking a URL that sets the appropriate response
variables to enact a redirect.  
\end{methoddesc}
\begin{methoddesc}{setContentType}{contentType}  Convienience function taking a single argument
to set the \texttt{Content-Type} header.  Equivalent to:
\begin{verbatim}
connObj.responseHeaders['Content-Type'] = arg
\end{verbatim}
\end{methoddesc}
\begin{methoddesc}{write}{s} Function taking a string to be written into the request body.
\end{methoddesc}
\begin{methoddesc}{setStatus}{statusCode} Set the HTTP response code to the integer argument.
\end{methoddesc}
\begin{methoddesc}{response}{} Returns a string containing the full text of the response.
\end{methoddesc}

%%%%%%%%%%%%%%%%%%%%%%%%%%%%%%%%%%%%%%%%
\chapter{The \texttt{templating} Service}
\index{services!\texttt{templating}}
The main raison d'\^etre of the \texttt{templating} service is to
integrate the \texttt{AE} library (see Appendix \ref{aelib}, page
\pageref{aelib}).  The \texttt{AE} library, in short, is responsible
for the component rendering, caching and message catalog facilities
that are generally the reason for using SkunkWeb in the first place.


\section{API}
The API to the \texttt{templating} service is actually quite minimal,
as most of the interesting bits are in the \texttt{AE} library proper.

\subsection{\texttt{templating.MailServices.sendmail}}

In the \texttt{templating.MailServices} module, there is one function
of interest:

\begin{funcdesc}{sendmail}{to_addrs, subj, msg, from_addr =
Configuration.FromAddress} 
The general-purpose sendmail function, which is called
    by the STML \texttt{<:sendmail:>} tag, or directly by Python code.

\begin{argdesc}
\item[to\_addrs] should be a list or tuple of email address
    strings. 
\item[subj] must be a string, although it may be empty.
\item[msg] is a string containing the body of the message; it can
    be empty. 
\item[from\_addr] is a single mail address string; it 
    defaults to the value of the \texttt{FromAddress} configuration variable.
\end{argdesc}
This function returns nothing on success, and raises a
\texttt{MailError} on any mail failure. 

\end{funcdesc}



\subsection{\texttt{templating.UrlBuilder} Functions}

\begin{funcdesc}{url}{path, queryargs, text = None, kwargs = {},
noescape = None, url = None}

Returns either the url part of the request (\texttt{http://path}) or
(if \texttt{text} is not \None) the complete link:
\verb!<A href="http://path">text</A>!. 

If the \texttt{url} argument is given, it is used in place of
\texttt{path} and \texttt{query}.

\begin{argdesc}
\item[path] Path to the target document.
\item[queryargs] Any HTTP GET style CGI arguments.
\item[text] Text to appear between the open and close \texttt{<A>} tags.
\item[kwargs] Any additional arguments to the open \texttt{<A>} tag
(e.g. \texttt{onClick}).
\item[noescape] Boolean specifying that the path should not be
properly escaped.
\item[url] As an alternate to \texttt{path}, the full absolute URL to the
target document.
\end{argdesc}
\end{funcdesc}


\begin{funcdesc}{image}{path, queryargs = None, kwargs={}, noescape = None}

Generate an HTML \texttt{<IMG>} tag.  Arguments are the same as those
for \texttt{url}.
\end{funcdesc}


\begin{funcdesc}{form}{path, kwargs = {}, noescape = None}
Generate an HTML \texttt{<FORM>} tag.  Arguments are the same as those
for \texttt{url}.
\end{funcdesc}


\begin{funcdesc}{redirect}{path = None, url = None, queryargs = {},
kwargs = {}, noescape = None} 
Cause a redirect to either the URL specified or to the path specified.
Arguments are the same as those for \texttt{url}.  
\end{funcdesc}

\begin{funcdesc}{retain}{dict}
Returns a string containing HTML \texttt{<INPUT TYPE=HIDDEN>} tags
that contain the contents of \texttt{dict}.  The dictionary keys are
used for the \texttt{NAME} arguments and the corresponding values are
used for the \texttt{VALUE} arguments.
\end{funcdesc}


\section{Component Cache Distribution}
\NOTE{NEEDS ATTENTION}
\label{ccdist}

%%%%%%%%%%%%%%%%%%%%%%%%%%%%%%%%%%%%%%%%
\chapter{The \texttt{pars} Service}

\section{The \PAR\ File Format}
The \PAR\ file format is quite simple.  What it is is a Python
marshalled format dictionary where the keys are directory/file names
and the values are a tuple of the offset into the data area (I'll get
to that in a second) and the information acquired at packaging time
from calling \texttt{os.stat()} on the file.

After the marshalled dictionary is the data area.  The data area is
where the contents of the files are stored (directiories, obviously
don't have anything here).  So, if you've loaded the dictionary into a
variable named, say \texttt{archDict}, and the data area loaded into
\texttt{dataArea} you can obtain the contents by:
\begin{verbatim}
offset, stat_info = archDict[filename]
contents = dataArea[offset:offset + stat_info[stat.ST_SIZE]]
\end{verbatim}

%%%%%%%%%%%%%%%%%%%%%%%%%%%%%%%%%%%%%%%%
\chapter{The \texttt{basicauth} Service}
\index{services!\texttt{basicauth}}

The \texttt{basicauth} service provides a basic authentication
mechanism for browser based authentication.

\section{API}
The \texttt{basicauth} service adds a few things to the
\connection\ object.  Namely,
\begin{memberdesc}[string]{AUTH_TYPE} The type of authentication,
should always be \verb!'Basic'!.
\end{memberdesc}
\begin{memberdesc}[string]{REMOTE_USER} The name that the browser
specified when authenticating to the server.  May be \None.
\end{memberdesc}
\begin{memberdesc}[string]{REMOTE_PASSWORD} The password that the browser
specified when authenticating to the server.  May be \None.
\end{memberdesc}

\section{The \texttt{swpasswd} Utility}
The \texttt{swpasswd} utility is a simple program to maintain
password files.  

Its general form is:
\begin{verbatim}
swpasswd [-cb] password_file username [password]
\end{verbatim}

The two options are:
\begin{argdesc}
\item[-c] Create a new password file
\item[-b] Specify the password on the command line as opposed to being
asked for it interactively.
\end{argdesc}


%%%%%%%%%%%%%%%%%%%%%%%%%%%%%%%%%%%%%%%%
\chapter{Writing A Service}

uri rewriting likely in web.HaveConnection

cookie auth schemes (probably also in web.HaveConnection)

document handlers

daisy chaining functions \'al\`a basicauth


an example easter egg service

a remote swpython example?  if so, warn why \textbf{really} bad

\textbf{Explain the \texttt{SkunkWeb.Configuration} ``module'' and
it's methods and that the namespace above is in this object}

\verb!_mergeDefaultsKw!

Hooks

The \verb!Server! interface

A request recording service


\appendix
\input ae.tex


\cleardoublepage

\printindex

\end{document}

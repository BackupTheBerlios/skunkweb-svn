\documentclass{manual}
\usepackage{makeidx}
\usepackage[nottoc]{tocbibind} % make so bib and ind are in toc
\title{Creating a SkunkWeb Application: A Gentle Tutorial}
\author{Jacob Smullyan}
\release{1.1}
\setshortversion{1}
\makeindex
\begin{document}
\maketitle

% \ 
\vfill 

\noindent
Copyright (C) 2002 Jacob Smullyan

\noindent
Permission is granted to make and distribute verbatim copies of this
manual provided the copyright notice and this permission notice are
preserved on all copies.

\tableofcontents

\chapter{Introduction}
\label{introduction}

The purpose of this tutorial is to give a web developer an idea of how
a fairly typical web application might be written in SkunkWeb. The
focus is not on explaining all of the separate facilities that
SkunkWeb provides, which are better explained in other documents, but
on how those facilities (among them being the STML templating and
component system, services for session handling, authentication,
database connection caching, and url rewriting, and the PyDO object
relational layer) can be used together.  This will done by tracing the
development of a reasonably elaborate example application that happens
to be one the author wanted an excuse to write anyway.  The process
will be goal-oriented; not on explaining SkunkWeb and its architecture
per se, except as necessary, but on explaining how to use it. 

\chapter{The Application: A Writing Game}
\label{application}

\section{The Rules of the Game}

The application being developed here is an multi-player, turn-based,
writing game.  The rules are very simple:

\begin{itemize}

\item A user, henceforth the game admin, creates a new game instance,
to which other users, the players, may join.  The game admin is also
the first player; there must be, by default, at least two players for
the game to proceed, and no more than six.

\item At any point when at least two players but less than six have
joined the game, the game admin may decide that the game is no longer
open for new players, at which point it can begin.  If six players
have joined, the join phase is closed automatically, and play can
begin.

\item The players take turns in the order in which they originally
joined the game.  The game admin, having joined the game automatically
by creating it, is therefore always the first player.  Each player is
prompted to append text to a single document; the document may be
viewed in its entirety by the players at any time, but the only
modification it is possible to make is to append to it.  A player's
turn is not over until he or she adds some non-blank text or punctuation
to the document; an empty submission is not allowed.

\item At any point, the game admin may decide that the document is
complete, at which the game is over, and the admin may decide whether
to edit the document, to publish it, or to trash it.  

\item If a game admin wishes to edit the document, the entire document
text is available to him or her for modification, without restriction.
(An enhancement would be to give the game admin the ability to specify
other users as having editorial privileges, but that isn't implemented
here.)  After editing, the document can be saved for further editing,
published, or trashed.

\item If the document is published, the game admin is prompted for a
title, and the document is published on the game website for everyone
to read.

\item If the document is trashed, it is no longer available for
editing and will not appear any longer on the website.

\end{itemize}

\section{The Game Interface}

An unauthenticated user, coming to the game website, would see, in
addition to descriptive text and site navigation:

\begin{enumerate}

\item A list of links to published items, in reverse order of
publication.  If the list is too long, it will show only the most
recent publications, and then link to a page with a fuller listing.
\item A form for user login.
\item A link to a sign-up page.

\end{enumerate}

Once a user had logged in, he or she would be able to go to his or her
home page, which would contain a summary of the user's current and
past game-playing activities:

\begin{enumerate}

\item A list of the player's active games, leading off with links to
the games where it is the player's turn (which are highlighted). 

\item A list of games it is possible to join.

\item A link to create a new game.

\item A list of any publications that resulted from previous games in
which the player participated.

\end{enumerate}

The game page during game play would show a heading giving some
metadata about the game (its players, the start date of the game, time
of the last move), and then present the document in its current state
in read-only form.  At bottom, there would be a form with a textarea
in which text to be appended could be entered.

\section{A Project Plan}

The following pages have been described above:

\begin{enumerate}

\item{the main page}
\item{user home page}
\item{the game page}
\item{create-a-game page}
\item{story page}
\item{story index page}

\end{enumerate}

In order to create these, we will need to meet these technical goals 
on the back end:

\begin{enumerate}

\item{user authentication}
\item{persistence of user data, game state, and texts}
\item{user session management}

Persistence will here be managed in the most straightforward and
conventional manner, namely, the use of a relational database (I'll be
using PostgreSQL).  
\end{enumerate}
%TO BE DONE:
%A Project Plan
%The Back End
%Authentication
%etc.

\end{document}





